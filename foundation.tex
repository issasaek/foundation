\documentclass[leqno]{ltjsarticle}
\renewcommand{\headfont}{\bfseries}
\usepackage{luatexja}
\usepackage{amsmath,amssymb,amsthm}
\usepackage{newpxtext,newpxmath}
\usepackage{enumitem} % 箇条書きのカスタマイズ
\usepackage[absolute,overlay]{textpos} % 要素を絶対座標で指定
\usepackage{url}
\newcommand{\version}{v2.0.2} % バージョン情報
\usepackage[luatex,unicode,pdfencoding=auto]{hyperref}
\hypersetup{
  bookmarksnumbered=true,
  colorlinks=true,
  allcolors=blue,
  pdftitle={規制の概念とその論理的還元},
  pdfauthor={佐伯 一冴},
  pdfsubject={\version},
  pdfkeywords={}
}
% 定理環境
\newtheoremstyle{mystyle}%   % スタイル名
    {5pt}%                   % 上部スペース
    {5pt}%                   % 下部スペース
    {}%                      % 本文フォント
    {}%                      % 1行目のインデント量
    {\bfseries}%             % 見出しフォント
    {}%                      % 見出し後の句読点
    {9.247pt}% 全角の幅        % 見出し後のスペース
    {\thmname{#1}\thmnumber{\hspace{2pt}#2}\thmnote{\hspace{7pt}[\gtfamily#3]}}%                      % 見出しの書式
\theoremstyle{mystyle}
\newtheorem{df}{D}[]% 定理環境(定義)
\newtheorem{axim}{A}[]% 定理環境(公理)
\newtheorem{thm}{T}[]% 定理環境(定理)
\newtheorem*{pf}{証明.}% 定理環境(証明) \textbf{照明.}
\newtheorem*{pfx}{証明の概略.}% 定理環境(証明)
\newtheorem*{nom}{正規性論証.}% 定理環境(証明)
\newtheorem*{dem}{論証.}% 定理環境(証明)
% log型関数
\DeclareMathOperator{\func}{Func}
\DeclareMathOperator{\cor}{Cor}
\DeclareMathOperator{\seq}{SEQ}
\DeclareMathOperator{\trcl}{trcl} % 下部構造
% 0項演算子
\newcommand{\univ}{\mathrm{V}} % 普遍クラス
\newcommand{\img}{{\mspace{1mu}{``}\mspace{1mu}}} % 関係の像
\newcommand{\fap}{{\mspace{1mu}{`}\mspace{1mu}}} % 関数適用
\newcommand{\resl}{\mathbin{\mid}} % 関係の積
\newcommand{\uphl}{\mathbin{\upharpoonleft}} % 左域の制限
\newcommand{\uphr}{\mathbin{\upharpoonright}} % 右域の制限
\newcommand{\exten}{\mathbin{\between}} % モデルに相対的な論理式の外延
% 1項演算子
\newcommand{\union}[1]{{\textstyle\bigcup}\mspace{3mu}#1} % クラス合併
\newcommand{\intersect}[1]{{\textstyle\bigcap}\mspace{3mu}#1} % クラス共通部分
\newcommand{\classab}[1]{\{\mspace{2mu}#1\mspace{2mu}\}} % クラス抽象
\newcommand{\barl}[1]{\bar{\:\:}#1} % 補クラス
\newcommand{\dotl}[1]{\dot{\:\:}#1} % 関係部分
\newcommand{\brevel}[1]{\breve{\:\:}#1} % 関係の逆
\newcommand{\tildel}[1]{\tilde{\:\:}#1} % 系列の反転
\newcommand{\ance}[1]{{^*}#1} % 祖先関係
\newcommand{\orp}[1]{\langle #1 \rangle}% 順序対
\newcommand{\kagi}[1]{「~#1~」}% 半角引用
\newcommand{\trgl}[1]{{^\triangleleft}#1} % 左域の唯一の要素
\newcommand{\indx}[1]{\mathfrak{h}#1} % 有意指標
\newcommand{\app}[1]{\mathrm{app}_{#1}} % 適用条件
\newcommand{\exe}[1]{\mathrm{exe}_{#1}} % 構成要件
\newcommand{\enf}[1]{\mathrm{enf}_{#1}} % 執行可能性
\newcommand{\cs}[1]{\mathrm{cs}_{#1}} % 制御構造
\newcommand{\cty}[1]{\mathrm{cty}_{#1}} % 制御可能性
\newcommand{\bkg}[1]{\mathbin{\vartriangleright_{#1}}} % 背景条件
% 2項演算子
\newcommand{\iter}[2]{#1^{\mid #2}} % 関係の反復
\newcommand{\timex}[2]{#1^{\times #2}} % 直積の反復
\newcommand{\prob}[2]{#1^{:#2}} % 蓋然性
\newcommand{\msec}[2]{#1^{\circledcirc #2}} % メンバーの系列要素
\newcommand{\mser}[2]{#1^{\otimes #2}} % メンバーの反転系列要素
% 真理関数の点
\newcommand{\md}[1]{{\;#1\;}}
\newcommand{\ld}[1]{{\;#1\!}}
\newcommand{\rd}[1]{{\!#1\;}}
\newcommand{\tdot}{{:\mspace{-3.5mu}.}}
\newcommand{\tdos}{{.\mspace{-3.5mu}:}}
\newcommand{\sdot}{{:\mspace{2mu}:}}
% 真理関数
\newcommand{\con}[1]{%
	\ifcase #1%
    \or \md{.} \or \md{:} \or \md{\tdot} \or \md{\sdot}
	\fi%
}%
\newcommand{\case}[3]{%
	\ifcase #2%
    \or \ld{.} \or \ld{:} \or \ld{\tdos} \or \ld{\sdot}
	\fi%
  \ifcase #1%
    \or \supset \or \lor \or \equiv \or 
  \fi%
  \ifcase #3%
    \or \rd{.} \or \rd{:} \or \rd{\tdot} \or \rd{\sdot}
  \fi%
}%

\begin{document}

% !TeX root = foundation.tex

\title{規制の概念とその論理的還元}
\author{佐伯 一冴\\\small\url{https://github.com/issasaek}}
\date{}
\maketitle

\begin{abstract}
工学的設計の対象としての規範が実はいかなる対象なのか明らかでない.規範の同一性の基準も物理的因果的世界における規範の位置も不明であるため,規範的システムの解析と設計は最適化が十分でない.その解決策として,規範をある種の因果的な構造とみなして,その構造を数学的意味での集合に還元する.
まず,因果の概念について,$a\in x$であることが$b\in y$であることを因果的に決定するような$\orp{a,x}$の$\orp{b,y}$に対する関係を,ある解釈空間に相対的に定義する.
次に,行動の環境的条件$y_1$,$y_1$における行動惹起$y_2$,行動の後続条件$y_3$を一義的に特定するような構造$y$について,
①$y$の執行可能性と②$y$の制御可能性という二つの因果的な構造を定義する.そして,$y_1$を充たす領域$x$について①と②が成立しているとき,順序対$\orp{x,y}$を1個の規制とみなす.工学的設計の対象としての規範はこの意味での規制である.
\end{abstract}

\section{序論}

規範的システムの設計は典型的には法システムにおいて現れるが,応用倫理学が倫理的ルールが表現する構造の設計に多分関わっているように,倫理的システムも設計対象になり得る.しかし,規範的システムを設計・構築する際,実はいかなる対象が設計・構築されているのかが明らかでないことが問題となる.つまり,規範の同一性の基準が不明であるから,1個の規範の内部構造を解析することができず,規範の集合が持つ全体的構造(例えば法規範の集合が持つであろう階層構造)の把握も覚束ない.また,物理的因果的世界における規範の位置が不明であり,規範がそれの成否による物理的相違を示唆するように記述されることもない.そのため,設計される規範的システムの因果的影響を予測して,その工学的妥当性を検証するという戦略が進展しない.

この問題の解決策として,規範をそれを表現するルール等の言語的対象と同一視することが考えられる.言語的トークンは通常の物理的対象であり,言語的タイプはトークンの集合かそのような集合の系列と見做せるから,言語的対象それ自体の同一性に問題はない.しかし,いかなる文または文の集合が1個の規範であるのか,という別の個別化の問題が直ちに生じる.仮にルールの記述的な性格を否定しつつ,その構文論的または意味論的な特徴によってこの個別化に対処しようとするなら,その困難は手に負えないものになるだろう\footnote{例えば,民法の約1000個の条文を眺めて,それらが何個の規範なのかを確定する作業を想像するとよい.}.むしろ必要なことは,規範をルール等の言語的対象から独立した構造として位置づけること,そして,ルールは端的にそのような構造に関する事実を記述すると認めることである.

本稿では,規範をある種の(主として物理的な)因果的構造として捉えた上で,その構造を数学的意味での集合に還元するという解決策を提案する.この因果的構造を表す理論的概念を拵えるが,その定式化は標準的な公理的集合論の言語に還元可能な形式で行われる.それによって,規範的システムの工学的妥当性の基準を構築することに方向づけを与え,規範的システムを解析的に記述する枠組を提供する.また,法的または倫理的規範から完全に個人的な規範に至るまで,それらが共通に持つ構造を特徴づけることによって,統合的な理解を促進する.

\section{描像}

これ以降,工学的な解析と設計の対象としての規範を特に「規制」と呼ぶ.第 \ref{sec:論理}節以降で定式化される概念は規制の概念である.それは規範の解析と設計を最適化する工学的理論を準備するために開発される人工的な言語的装置であり,既存の規範表現の概念分析ではない.したがって,日常的な「規制」等との用法の一致は,規制の概念に付与された機能を果たすのに必要な範囲でのみ要求される.また,規制の概念の定式化は,日常言語の規範演算子を識別してそれに関わる論理を定式化しようとするものではない.それは,工学的設計の対象としての物理的システムを記述する言語的装置であり,規範表現の論理形式とは関係がない.単にそのような記述を明晰化するために形式化された言語が使用されるにすぎない.
以上を確認した上で,ここでは定式化の前段階として,規制の概念の漠然としたイメージを提示する\footnote{規制の概念はそれなりに込み入っており,冗長にならずに日常言語で完全に展開することは困難である.}.

\subsection{一般形式}

規範的システムは,人やその他のシステムの行動を制御すること,すなわちそれを促進したり抑制したりする機能を持つ.犯罪に対する刑罰の仕組みや,道徳的な行動を賞賛する(または不道徳を非難する)等の事例では明白であるが,そこでは学習理論において時に弁別学習と呼ばれるようなプロセスが利用されている.
弁別学習とは,一定の条件$F$における行動$G$に報酬または罰$H$を(何度か)後続させると,同一の行動主体について$F$における$G$の出現確率が高まる/低下する,というプロセスである\footnote{レイノルズ~\cite[pp.\,9--12]{レイノルズ}.}.行動主体が言語的能力を持つ場合,$F+G$に$H$が後続するという経験を積まなくても,「$F+G$の実現によって$H$が実現される」という因果関係の情報を言語的な経路で取得すれば,$F$における$G$の出現確率が同様に変化し得る\footnote{長谷川~\cite[pp.\,4]{長谷川}.}.

規制の概念は,この弁別学習のプロセスに含まれる因果的構造からその一般形式を抽象して得られる.
まず,条件$F$,行動$G$,後続条件$H$を一義的に特定する何らかの構造(規制類型)$X$を考える.今の段階では単に3つ組$\orp{F,G,H}$でもよい.
次に,規範的システムの対象は特定の有機体の再現性を持つ行動だけではないため(同一の個体について$F$や$G$の再現性がないケースがあり得る),$F$における$G$の出現確率の変化の代わりに,次のいずれかの制御構造(の蓋然性)に言及する.
\begin{enumerate}[label=(\arabic*)]
    \item $F$の実現によって$G$が実現する,という因果的関係($X$の正の制御構造),
    \item $F$の実現によって$G$の実現が阻止される,という因果的関係($X$の負の制御構造).
\end{enumerate}
また,$F+G$に$H$が(何度か)後続するという事態は,$F+G$($X$の構成要件)が実現し,かつ,次の因果的関係($X$の執行可能性)が成立しているという事態に置き換える.
\begin{align*}
    \text{$F+G$の実現によって$H$(の蓋然性)が実現される,という因果的関係.}
\end{align*}
さらに,$X$の執行可能性を充たす領域で$X$の構成要件が共に実現すること($X$の執行随伴性)によって,同一の行動主体について$X$の正/負の制御構造(の蓋然性)が生じる,という因果的関係を$X$の正/負の制御可能性と言う\footnote{この因果的関係を担うメカニズムは行動主体の種類や個体ごとに異なり得る.メカニズムの相違は行動科学ではともかく規範的システムの解析と設計においては必ずしも重要ではない.行動主体が人間等の有機体であっても,法人のようなシステムであっても,規制の概念の一般形式は等しく適用できる.}.

$F$を充たす領域$s$について$X$の執行可能性と$X$の制御可能性が成立しているとき,順序対$\orp{s,X}$は1個の規制である.このとき,
$s$について成立する制御可能性が正の制御可能性であるなら,$s$における$H$の実現は$s$における$X$の構成要件の実現に対して報酬の機能を持つ.他方,負の制御可能性なら罰の機能を持つ.
そして,前者の$\orp{s,X}$をP規制,後者のそれをS規制と呼ぶ.P規制は伝統的なカテゴリーで言う許可や権利の領域を,S規制は禁止や義務の領域を主としてカバーする.

\subsection{言語的統制}

引き続き$X=\orp{F,G,H}$と置き,行動主体$a$を含む領域$s$について,$F$と$X$の制御可能性が成立していると仮定する.このとき,$s$について$X$の執行可能性を構築しておけば,$s$で$G$が実行された場合に,将来の$a$について$X$の制御構造(の蓋然性)を形成することができる\footnote{規制類型$X$に関する規制を設計し構築するということは,$X$の執行可能性を構築することを意味する.}.
しかし,$s$における$G$の実行がなくても,$s$における$X$の執行可能性を記述する($a$の言語の)文$\phi$について,$a$において$\phi$を肯定する言語的傾向性があれば,同一の制御構造を形成できる可能性がある.このような言語的統制が機能するための典型的なモデルを以下のように構成できる.

まず,他の規制の執行可能性を後続条件(報酬/罰)とする規制を考える.この場合,その構成要件実現によって他の規制の執行可能性(の蓋然性)が構築される.すなわち,$X'=\orp{F',G',H'}$について,$s'$における$H'$の実現が$s$における$X$の執行可能性の実現であるような規制$ \orp{s',X'} $を考える.$s'$において$X'$の構成要件が実現している場合,$ \orp{s',X'} $は$ \orp{s,X} $に対して制定関係を持つ(制定規制).
これに対して,$s'$における$H'$の実現が$s$における$X$の執行可能性の阻止である場合は,廃止関係を持つ(廃止規制).
$s'$における制定類型$X'$の構成要件の実現が,$X$の執行可能性を記述する規制ルール\footnote{ここでは,$F$を充たす任意の領域$s$について$X$の執行可能性が成立する,ということを記述する文を意味する.}を当事者に提示する行動である場合,言語的統制との関係で特に重要なカテゴリーに属する(規制表明型)\footnote{例えば,議会の立法権や契約により私法的権利義務を創出する権利は,制定規制であるP規制であり,かつ,規制表明型である.他方,不法行為により損害賠償義務を負う関係は,制定規制であるS規制の執行可能性であるが,規制表明型ではない.}.

次に,ある規制集合$\alpha$に相対的な正規集合の概念を構成する.すなわち,$\orp{s,X}$が制定関係を反復して$\alpha$のメンバーに遡及可能であり,かつ,その反復のどの段階についても,それに対して廃止関係を持つ正規集合のメンバーが存在しないとき,$\orp{s,X}$は$\alpha$の正規集合に属する\footnote{$\alpha$のメンバーとして何を選ぶかは文脈依存的であるが,差し当たって,憲法その他の基本法の一群かまたはそれらを制定する権限を想定すればよい.}.

さて,$r_1\in\alpha$であり,かつ,系列の前者が後者に対して制定関係を持つような規制の系列$r_1,r_2,\dots,r_n$を考える.ただし,$r_n =\orp{s,X}$とする.
そして,$s$の行動主体$a$について,次の①②を仮定する.
\begin{enumerate}
    \item[①] $\alpha$の正規集合の主要部分が規制表明型である等により,$a$は,$r_1,r_2,\dots,r_n$及び関連する正規集合メンバーの構成要件該当性を記述する文を肯定する傾向性を持つ.
    \item[②] $a$は,$r_1\in\alpha$の執行可能性を記述する文を肯定する傾向性を持つ.
\end{enumerate}
このとき,以下のようにして,$s$における$X$の執行可能性を記述する文$\phi$を肯定する$a$の言語的傾向性を形成できる.
すなわち,②により,$r_1$の執行可能性は前提される.したがって,$r_1$の構成要件該当性を証拠として,$r_2$の執行可能性を認定できる.次に,$r_2$の構成要件該当性を証拠として,$r_3$の執行可能性を認定できる.以下同様にして,$r_n$の執行可能性を認定できる.ただし,各$r_i(1\leq i\leq n)$の執行可能性が,$\alpha$の正規集合に属する規制によって廃止されていないことも前提しなければならない.
 % 概要,序論,描像
% !TeX root = foundation.tex

\section{論理}
\label{sec:論理}

規制の概念を定式化するために,\ref{ssec:原初的言語}において形式化された
% 述語論理に適合するように統制された
原初的言語を,\ref{ssec:クラス}において集合またはクラスの概念に関わる一連の省略記法を導入する\footnote{記号法は大体において,クワイン~\cite{クワインa}とクワイン~\cite{クワインb}に依拠している.}.
規制の概念は,定義を通じてこの原初的言語の記法に還元可能なように定式化される.それによって,(因果の概念が特定されない解釈空間に相対的である点を除けば)この概念に曖昧さがないことが示される.
さらに,\ref{ssec:集合論の体系}においては,標準的な公理的集合論の体系を導入して基本的な存在論を確定する.

\subsection{原初的言語}
\label{ssec:原初的言語}

本稿の原初的言語は,形式化された文またはその図式である論理式のクラス(量化言語)を規定することを通じて導入される.
まず原子式を構成する要素として,以下の無限個の変項と述語記号を導入する.
\begin{enumerate}
    \item 変項:\kagi{$x$},\kagi{$x'$},\kagi{$x''$},$ \dots $
    \item 述語記号:\kagi{$ (F,) $},\kagi{$ (F,)' $},\kagi{$ (F,)'' $},$\dots$,\kagi{$ (F,\!,) $},\kagi{$ (F,\!,)' $},\kagi{$ (F,\!,)'' $},$\dots$
\end{enumerate}
\kagi{$,$}の数が$n$,\kagi{$'$}の数が$i$であるとき,$n$項述語記号の$i$番目を意味する.
次に,複合表現を構成する手段として,以下の記号を導入する.
\begin{enumerate}
    \item 否定及び条件法の真理関数記号:\kagi{$\neg$}及び\kagi{$\supset$}
    \item 普遍量化子:\kagi{$(x)$},\kagi{$(x')$},\kagi{$(x'')$},$ \dots $
\end{enumerate}
そして論理式は,次の3つの規則により再帰的に記述される.
\begin{enumerate}[label=(\arabic*)]
    \item \kagi{$ R\alpha_1\dots\alpha_n $}の\kagi{$ R $}に$0\neq n$項述語記号の$i$番目を,\kagi{$ \alpha_1\dots\alpha_n $}に$n$個の変項を並べて代入した結果は論理式である.
    \item \kagi{$\neg P$}と\kagi{$(P\supset Q)$}において,\kagi{$P$}と\kagi{$Q$}に任意の論理式を代入した結果は論理式である.
    \item \kagi{$(\alpha)P$}において,\kagi{$\alpha$}に任意の変項を,\kagi{$P$}に任意の論理式を代入した結果は論理式である.
\end{enumerate}
他の真理関数は否定と条件法から,存在量化子は普遍量化子から定義される \footnote{
    否定と連言等,否定と条件法以外の組み合わせを原始的として,他の真理関数をそこから定義することもできる.また,存在量化子を原始的として普遍量化をそこから定義することもできる.
}.つまり,
\begin{gather*}
    \text{\kagi{$ (P\lor Q) $}は\kagi{$ (\neg P\supset Q) $}を表わす,}\\
    \text{\kagi{$ (P\con{1}Q) $}は\kagi{$ \neg(\neg P\lor\neg Q) $}を表わす,}\\
    \text{\kagi{$ (P\equiv Q) $}は\kagi{$ (P\supset Q)\con{1}(Q\supset P) $}を表わす,}\\
    \text{\kagi{$(\exists\alpha)P$}は\kagi{$\neg(\alpha)\neg P$}を表わす.}
\end{gather*}
なお,\kagi{$(\alpha)P$}という文脈における\kagi{$\alpha$}の位置に来る変項のすべての出現は「束縛出現」と呼ばれる.論理式における変項の出現が束縛出現ではないとき,その出現は「自由出現」と呼ばれる.そして,変項が自由出現しない論理式を「閉鎖式」,閉鎖式でない論理式(少なくとも1個の変項の自由出現を持つ論理式)を「開放式」と言う.また,原初的言語の文である閉鎖式は「閉鎖文」,文である開放式は「開放文」と言う.

次に,1個の形式言語は論理式のクラスの部分クラスであり,それは当該言語の述語として使用する述語記号を指定することによって規定できる.本稿の原初的言語では,$2$項述語記号の$0$番目\kagi{$ (F,\!,) $}が,要素関係を表わす\kagi{$ \in $}の形式的な表現であるとみなされる.すなわち,
\begin{itemize}
    \item \kagi{$(\alpha\in\beta)$}の\kagi{$\alpha$}と\kagi{$\beta$}に任意の変項を代入した結果は,\kagi{$ (F,\!,)\alpha\beta $}に同一の代入をした結果を表わす.
\end{itemize}
この文脈的定義により,\kagi{$ (F,\!,) $}は,原初的言語における実質的な原始的述語として(実際上は\kagi{$ \in $}で代用して),使用される.そして,これ以外の述語記号が出現しない論理式が原初的言語の文である.他方,文でない論理式は文の論理構造を表わす図式(量化図式)となる.

この他,実用的な便法として,いくつかの記法を明示的な定義によらずに採用する.まず,変項は\kagi{$x$},\kagi{$y$},\kagi{$z$}等々で,原始的述語以外の述語記号は\kagi{$F$},\kagi{$G$},\kagi{$H$}等々で代用される.
また,\kagi{$\notin$}や\kagi{$\neq$}における打ち消しを\kagi{$\neg$}の代わりに用いたり,連言を短縮して,\kagi{$x,y\in\alpha$},\kagi{$x=y=z$}等と書くことがある.次に,文を単独で表示する場合の一番外側の括弧を省略する.その他の括弧は真理関数に点を付加することによって適当に省略する.すなわち,①点が付加された連言は,それより少数の点が(その連言の側に)付加された真理関数よりも大きな区切りを表わす.また,②点が付加された連言以外の真理関数は,点が付加された側にある,それより少数の点が(その真理関数の側に)付加された真理関数よりも大きな区切りを表わし,連言を表わす同数以下の点集団よりも大きな区切りを表わす\footnote{
    クワイン~\cite[pp.\,26--28]{クワインb}を参照.
}.この点記法によると,例えば,\kagi{$ ((P\supset(Q\lor R))\con{1}S) $}は\kagi{$ P\case{1}{0}{1}Q\lor R\con{2}S $}となり,
\[
    (S\lor(((P\con{1}(Q\supset R))\equiv((P\lor Q)\con{1}R))\con{1}T))
\]
は次のように書かれる.
\[
    S\case{2}{0}{2}P\con{1}Q\supset R\case{3}{1}{1}P\lor Q\con{1}R\con{2}T.
\]

\subsection{クラス}
\label{ssec:クラス}

集合は何らかのクラスの要素であるクラスであり,いかなるクラスの要素でもないクラス(究極クラス)と区別される.集合以外のクラスを持たない体系ではこの区別は消えるが,\ref{ssec:集合論の体系}で導入されるのもそうした体系である.したがって,本稿では「集合」と「クラス」を可換的に使用する.

最初に,$Fx$である$x$のクラスを指示しようとするクラス抽象体\kagi{$\classab{x:Fx}$}を文脈的に定義する.

\begin{df}
\label{df:クラス抽象A}
\kagi{$
   y\in\classab{x:Fx}
$}は\kagi{$
   Fy
$}を表わす.
\end{df}

\noindent 本稿で定義を述べる際,代用変項\kagi{$x$},\kagi{$y$}等は変項の位置を,文型\kagi{$Fx$},\kagi{$Fy$}等は原初的言語の文の位置を表わしている.また,ギリシア文字\kagi{$\alpha$},\kagi{$\beta$}等は,任意の変項またはクラス抽象体の位置を表わす型文字である.
つまり,定義文は,被定義項の\kagi{$x$},\kagi{$y$}等に変項を,\kagi{$F$}に原初的言語の開放文を,\kagi{$\alpha$},\kagi{$\beta$}等に変項またはクラス抽象体を代入した結果は,定義項に同一の代入をした結果を表わす,ということを意味している.
開放文$\phi$の代入は\kagi{$F$}とそれに続く$n$個の変項出現を文で置き換えることを意味するが,その文は,$\phi$における変項$v_0,v_1\dots v_n$の自由出現のすべてを\kagi{$F$}に続く変項で順次置換することにより得られる. ただし,$v$ は$\phi$に自由出現する変項を\kagi{$'$}の数が少ない順に$n$個並べた系列である\footnote{
    さらに,変項の新たな束縛を防ぐために,次の①②のケースでは$\phi$は代入不可とする.①$v_0,v_1\dots v_n$以外で$\phi$に自由出現する変項$k$について,$\phi$が代入される文脈で$k$が量化子の変項であり,かつ,その射程内に代入するケース.②\kagi{$F$}に続く$i$番めの変項が$\phi$で量化子の変項であり,かつ,その射程内に$v_i$が自由出現するケース.クワイン~\cite[pp.\,154--156]{クワインb}を参照.
}.

次に,クラスのブール代数に属するよく知られた概念群を導入する.

\begin{df}[部分クラス]
\label{df:部分クラス}
\kagi{$
    \alpha\subseteq\beta
$}は\kagi{$
    (x)(x\in\alpha\case{1}{1}{1}x\in\beta)
$}を表わす,
\end{df}

\begin{df}[真部分クラス]
\label{df:真部分クラス}
\kagi{$
    \alpha\subset\beta
$}は\kagi{$
    \alpha\subseteq\beta\not\subseteq\alpha
$}を表わす,
\end{df}

\begin{df}[合併]
\label{df:合併}
\kagi{$
    \alpha \cup \beta
$}は\kagi{$
    \classab{x:x\in\alpha\case{2}{1}{1}x\in\beta}
$}を表わす,
\end{df}

\begin{df}[共通部分]
\label{df:共通部分}
\kagi{$
    \alpha \cap \beta
$}は\kagi{$
    \classab{x:x\in\alpha\con{1}x\in\beta}
$}を表わす,
\end{df}

\begin{df}[補クラス]
\label{df:補クラス}
\kagi{$\bar{\alpha}$}あるいは\kagi{$\barl{\alpha}$}は,\kagi{$
    \classab{x:x\notin\alpha}
$}を表わす.
\end{df}

D \ref{df:部分クラス}〜D \ref{df:補クラス}のように,定義文が定義項にのみ出現する\kagi{$x$},\kagi{$y$}等を持つ場合,被定義項を一義的に変換できない.そこで,定義項にのみ出現する\kagi{$x$},\kagi{$y$}等への代入は,被定義項への代入結果に出現しない変項を,\kagi{$'$}の数が少ない順番に使って,アルファベット順に代入するものと決める\footnote{
    さらに,変項の新たな束縛を防ぐために,代入される変項または代入される抽象体に自由出現する変項$v$について,次の①②のケースでは代入不可とする.①$v$が量化子の変項の位置に代入される場合で,その射程内に代入前から$v$が出現するケース.②代入前から$v$が量化子の変項であり,その射程内に代入するケース.
}.

次に,同一性とその関連概念を導入する.

\begin{df}[同一性]
\label{df:同一性}
\kagi{$
    \alpha = \beta
$}は\kagi{$
    \alpha\subseteq\beta\subseteq\alpha
$}を表わす.
\end{df}
\noindent D \ref{df:同一性}の定義式は,\kagi{$(x)(x\in\alpha\case{3}{1}{1}x\in\beta)$}とも表せる.すなわち,クラスの同一性はその要素の同一性に帰着する.したがって,規範をクラスに還元できれば(クラスの同一性の基準は明確であるから)その同一性の問題は解消される.それに加えて,物理的対象を含むあらゆるものをクラスとみなすことができれば,さらに存在論を単純化できるだろう.この点については後述する.

\begin{df}[空クラス]
\label{df:空クラス}
\kagi{$
    \Lambda
$}は\kagi{$
    \classab{x:x\neq x}
$}を表わす,
\end{df}

\begin{df}[普遍クラス]
\label{df:普遍クラス}
\kagi{$
    \univ
$}は\kagi{$
    \classab{x:x=x}
$}を表わす,
\end{df}

\begin{df}[単一クラス]
\label{df:単一クラス}
\kagi{$
    \classab{\alpha}
$}は\kagi{$
    \classab{z:z=\alpha}
$}を表わす,
\end{df}

\begin{df}[対クラス]
\label{df:対クラス}
\kagi{$
    \classab{\alpha,\beta}
$}は\kagi{$
    \classab{\alpha}\cup\classab{\beta}
$}を表わす.
\end{df}

次の定義は,D \ref{df:クラス抽象A}と相まって,クラス抽象体の文脈的定義を完成させる.

\begin{df}
\label{df:クラス抽象B}
\kagi{$
    \classab{x:Fx}\in\beta
$}は\kagi{$
    (\exists y)(y=\classab{x:Fx}\con{1}y\in\beta)
$}を表わす.
\end{df}
\noindent\kagi{$y\in\classab{x:Fx}$}を肯定することは,D \ref{df:クラス抽象A}により,単に\kagi{$Fy$}を言うことだから,クラス$\classab{x:Fx}$が存在するとの含みはない.これに対して,$ \classab{x:Fx} $がメンバーになることを肯定することは,D \ref{df:クラス抽象B}により,$ \classab{x:Fx} $の存在を肯定することを含意する.

次に,$\alpha$の合併と共通部分の概念.

\begin{df}
\label{df:クラスの合併}
\kagi{$
    \union{\alpha}
$}は\kagi{$
    \classab{x:(\exists y)(x\in y\con{1}y\in\alpha)}
$}を表わす,
\end{df}

\begin{df}
\label{df:クラスの共通部分}
\kagi{$
    \intersect{\alpha}
$}は\kagi{$
    \classab{x:(y)(y\in\alpha\case{1}{1}{1}x\in y)}
$}を表わす.
\end{df}

次に,$ Fx $である唯一の対象$x$を指示しようとする単称記述\kagi{$(\imath x)Fx$}を,そのような$ x $が存在しないとき$ (\imath x)Fx=\Lambda $となるように定義する.

\begin{df}
\label{df:単称記述}
\kagi{$
    (\imath x)Fx
$}は\kagi{$
    \union{\classab{y:(x)(Fx\case{3}{0}{1}x=y)}}
$}を表わす.
\end{df}
\noindent 単称記述に関して,次の2つの法則が量化論理学によって証明可能である.したがって,D \ref{df:単称記述}は意図された通りの意味を与える.
\[
    (x)(Fx\case{3}{0}{1}x=y)\case{1}{0}{1}(\imath x)Fx=y,\qquad \neg(\exists y)(x)(Fx\case{3}{0}{1}x=y)\case{1}{0}{1}(\imath x)Fx=\Lambda.
\]

次に,$Fxy$である$x$の$y$に対する関係を,$Fxy$なるすべての順序対$\langle x,y \rangle$のクラスとして導入する.

\begin{df}[順序対]
\label{df:順序対}
\kagi{$
    \orp{\alpha,\beta}
$}は\kagi{$
    \classab{\classab{\alpha},\classab{\alpha,\beta}}
$}を表わす.
\end{df}
\noindent この定義により,順序対の概念に要求される法則\kagi{$\orp{x,y}=\orp{z,w}\case{3}{1}{1}x=z\con{1}y=w$}が成り立つ.
なお,順序のある三つ組\kagi{$\orp{\alpha,\orp{\beta,\gamma}}$}は\kagi{$\orp{\alpha,\beta,\gamma}$}に,
四つ組\kagi{$\orp{\alpha,\orp{\beta,\gamma,\delta}}$}は\kagi{$\orp{\alpha,\beta,\gamma,\delta}$}に省略されるものとする.五つ以上の組についても同様である.

次の定義によって,クラス抽象体の表現を拡張する一般的な記法手段を導入する.\kagi{$\alpha(v_0v_1\dots v_n)$}を変項$ v_0,v_1,\dots v_n $のみが自由出現するクラス抽象体とすると,
\begin{df}
\label{df:クラス抽象C}
\kagi{$
    \classab{\alpha(v_0v_1\dots v_n):Fv_0v_1\dots v_n}
$}は,\\\hfill\kagi{$
    \classab{z:
        (\exists v_0)(\exists v_1)\dots(\exists v_n)(Fv_0v_1\dots v_n\con{1}z=\alpha(v_0v_1\dots v_n))
    }
$}を表わす.
\end{df}
\noindent したがって,\kagi{$\alpha(v_0v_1\dots v_n)$}を,(クラス抽象体の省略形である)\kagi{$\orp{x,x'}$}にとれば,
\[
    \text{
        \kagi{$
            \classab{\orp{x,x'}:Fxx'}
        $}は\kagi{$
            \classab{x'':
                (\exists x)(\exists x')(Fxx'\con{1}x''=\orp{x,x'})
            }
        $}を表わす
    }
\]
が得られる.

D \ref{df:クラス抽象C}により,関係の諸概念はクラス抽象体の省略形とみなせる.すなわち,

\begin{df}[関係部分]
\label{df:関係部分}
\kagi{$
    \dotl{\alpha}
$}は\kagi{$
    \classab{\orp{x,y}:\orp{x,y}\in\alpha}
$}を表わす.
\end{df}
\noindent$\alpha$が関係であるのは,$\alpha=\dotl{\alpha} $であるときである.

\begin{df}[同一性関係]
\label{df:同一性関係}
\kagi{$
    I
$}は\kagi{$
    \classab{\orp{x,y}:x=y}
$}を表わす,
\end{df}

\begin{df}[要素関係]
\label{df:要素関係}
\kagi{$
    \mathfrak{E}
$}は\kagi{$
    \classab{\orp{x,y}:x\in y}
$}を表わす,
\end{df}

\begin{df}[関係の逆]
\label{df:関係の逆}
\kagi{$\breve{\alpha}$}あるいは\kagi{$\brevel{\alpha}$}は,
\kagi{$
    \classab{\orp{y,x}:\orp{x,y}\in\alpha}
$}を表わす,
\end{df}

\begin{df}[関係の像]
\label{df:関係の像}
\kagi{$
    \alpha\img\beta
$}は\kagi{$
    \classab{x:(\exists y)(\orp{x,y}\in\alpha\con{1}y\in\beta)}
$}を表わす.
\end{df}
\noindent $ \alpha\img\univ $を「~$\alpha$の左域」,$ \breve{\alpha}\img\univ $を「~$\alpha$の右域」と言う.

\begin{df}[関係の積]
\label{df:関係の積}
\kagi{$
\alpha\resl\beta
$}は\kagi{$
    \classab{\orp{x,z}:
        (\exists y)(\orp{x,y}\in\alpha\con{1}\orp{y,z}\in\beta)
    }
$}を表わす,
\end{df}

\begin{df}[直積]
\label{df:直積}
\kagi{$
    \alpha\times\beta
$}は\kagi{$
    \classab{\orp{x,y}:
        x\in\alpha\con{1}y\in\beta
    }
$}を表わす,
\end{df}

\begin{df}[右域の制限]
\label{df:右域の制限}
\kagi{$
    \alpha\uphr\beta
$}は\kagi{$
    \alpha\cup(\univ\times\beta)
$}を表わす,
\end{df}

\begin{df}[左域の制限]
\label{df:左域の制限}
\kagi{$
    \beta\uphl\alpha
$}は\kagi{$
    \alpha\cup(\beta\times\univ)
$}を表わす.
\end{df}

次に,$ \orp{y,x}\in\alpha $である唯一の$y$が存在するとき,$x$を$\alpha$の独立変項と呼び,そのクラスは
\begin{df}
\label{df:独立変項}
\kagi{$
    \arg\alpha
$}は\kagi{$
    \classab{x:(\exists y)(\alpha\img\classab{x} =\classab{y})}
$}を表わす,
\end{df}
\noindent と定義される.また,$y$は$\alpha$による$x$の値$\alpha\fap x$である.

\begin{df}
\label{df:関数適用}
\kagi{$
    \alpha\fap\beta
$}は\kagi{$
    (\imath y)(\orp{y,\beta}\in\alpha)
$}を表わす.
\end{df}

\noindent さらに,どの二つの対象も同じ対象に対して関係$\alpha$を持つことがないならば,$\alpha$は関数である.つまり,

\begin{df}
\label{df:関数}
\kagi{$
    \func\alpha
$}は\kagi{$
    (x)(y)(z)(
        \orp{x,z}\in\alpha\con{1}\orp{y,z}\in\alpha\case{1}{1}{1}x=y
    )\con{1}\alpha=\dotl{\alpha}
$}を表わす.
\end{df}
\noindent 例えば,$ \classab{\orp{x,y}:\text{$x$は$y$の遺伝的な父}} $は関数であるが,その逆(子の父に対する関係)は関数ではない.
$ \alpha $が関数であるとき,$ \arg\alpha =\breve{\alpha}\img\univ $である.

\kagi{$\dots x \dots$}が\kagi{$x$}に代入される変項,または,それが自由出現するクラス抽象体の位置を表わす場合,次の定義は,独立変項$x$の値が$ \dots x \dots $であるような関数を指示する記法(関数抽象)を導入する\footnote{\kagi{$\lambda_x$}の他,\kagi{$\barl{}$},\kagi{$\brevel{}$}等の単項演算子の作用域は,それに続く文法的に可能な文字列の最小部分であると決める.}.

\begin{df}
\label{df:関数抽象}
\kagi{$
    \lambda_x(\dots x \dots)
$}は\kagi{$
    \classab{\orp{y,x}:y=\dots x \dots}
$}を表わす.
\end{df}

次に,個々の自然数を,$ \Lambda $を含み,かつ後続者演算$\mathrm{S}$に関して閉じているような,すべてのクラスの共通のメンバーとして,自然数のクラス$\mathbb{N}$を定義する.

\begin{df}
\label{df:後続者関数}
\kagi{$
    \mathrm{S}
$}は\kagi{$
    \lambda_x(x\cup\classab{x})
$}を表わす,
\end{df}

\begin{df}
\label{df:自然数}
\kagi{$
    \mathbb{N}
$}は\kagi{$
    \intersect{\classab{z:\Lambda\in z\con{1}\mathrm{S}\img z\subseteq z}}
$}を表わす.
\end{df}
\noindent したがって,自然数$ 0,1,2,3,\dots $は,$ \Lambda,\classab{\Lambda},\classab{\Lambda,\classab{\Lambda}},\classab{\Lambda,\classab{\Lambda},\classab{\Lambda,\classab{\Lambda}}},\dots $である\footnote{
    自然数を表わす\kagi{$\Lambda$},
    \kagi{$\mathrm{S}\fap\Lambda$},
    \kagi{$\mathrm{S}\fap(\mathrm{S}\fap\Lambda)$},
    \kagi{$
        \mathrm{S}\fap
        (
            \mathrm{S}\fap(\mathrm{S}\fap\Lambda)
        )
    $}等は,\kagi{$0$},\kagi{$1$},\kagi{$2$},\kagi{$3$}等で省略される.
}.自然数はそれより小さい自然数のクラスであり,その大小関係は\kagi{$\in$}または\kagi{$ \subset $}で表わせる.

次に,対象$ a,b,c $をこの順番で並べた系列は,単に,$ \orp{a,0},\orp{b,1},\orp{c,2} $という$ 3 $個の順序対からなるクラスである.このような有限系列のクラスは次の定義による.
\begin{df}
    「\,$ \mathrm{Seq} $\,」は
    「\,$ \{\,x:
        \func x \con{1} \arg x\in \mathbb{N}
    \,\} $\,」を表わす.
    \label{df:有限系列}
\end{df}
\noindent$ z\in\mathrm{Seq} $であるとき,$ z $はちょうど$ \arg z $個の系列要素を持ち,任意の$ n\in \mathbb{N} $について,$z\uphr n$もまた系列になる.
有限系列の概念を使って,$\alpha$の$\beta$回の反復$\iter{\alpha}{\beta}$を導入できる.

\begin{df}
\label{df:関係の反復}
\kagi{$
    \iter{\alpha}{\beta}
$}は\kagi{$
    \classab{\orp{x,y}:
        (\exists z)(
            \orp{x,\beta},\orp{y,\Lambda}\in z\in\mathrm{Seq}\con{1}z\resl(\mathrm{S}\resl\breve{z})\subseteq\alpha
        )
    }
$}を表わす.
\end{df}
\noindent$\iter{\alpha}{\beta}$は,系列の後者が前者に$\alpha$を持つような有限系列$z$について,$z$の$\beta$番目が$0$番目に対して持つ関係である.

関連して,$\alpha$の祖先関係$\ance{\alpha}$は,$x$が$y$に対して$\alpha$の反復を持つ$\orp{x,y}$のクラスである\footnote{
    先に\kagi{$ \ance{\alpha} $}を\kagi{$ \classab{\orp{x,y}:x\in\intersect{\classab{z:y\in z\con{1}\alpha\img z\subseteq z}}} $}と定義してから,\kagi{$ \mathbb{N} $}を\kagi{$ \ance{\mathrm{S}}\img\classab{\Lambda} $}と定義する方法もある.クワイン~\cite[pp.\,93--95]{クワインa}を参照.
}.すなわち,

\begin{df}
\label{df:祖先関係}
\kagi{$
    \ance{\alpha}
$}は\kagi{$
    \classab{w:(\exists z)(w\in\iter{\alpha}{z})}
$}を表わす.
\end{df}

\noindent $\ance{\alpha}$は,$x=y$であるか,$\orp{x,y}\in\alpha$,または,$\orp{x,y}\in\alpha\resl\alpha$,または,$\orp{x,y}\in\alpha\resl(\alpha\resl\alpha)$,等々であるような$x$の$y$に対する関係である.
例えば,$\alpha$を人の親子関係にとれば,人の祖先関係が得られる\footnote{
$\ance{\alpha}$は$\alpha$を反復して$y$から$x$に遡及可能であるような$\orp{x,y}$のクラスである.定義上$ I\subseteq\ance{\alpha} $であるから$x$の祖先には$x$自身が含まれる.それでは困るケースでは$\alpha\resl\ance{\alpha}$を使う.そもそも$I\subseteq\alpha$のケースを考えると$\ance{\alpha}\cap\bar{I}$では微妙である.}.

\subsection{集合論の体系}
\label{ssec:集合論の体系}

規制はクラスであり,その存在はクラス理論の公理によって決定される.いかなるクラスが存在するかという問題は,いかなるクラス理論を採用するかという問題である.
以下では標準的と思われる体系ZFCを導入する.公理の主要部分はクラスの存在仮定である.クラス$\alpha$が存在することを,\kagi{$ (\exists x)(x=\alpha) $}と等値な\kagi{$ \alpha\in\univ $}で表わす.また,次の省略記法を使う.
\begin{df}[冪集合]
\label{df:冪集合}
\kagi{$
    \mathcal{P}(\alpha)
$}は\kagi{$
    \classab{x:x\subseteq\alpha}
$}を表わす.
\end{df}

\noindent すると,以下のA \ref{axim:外延性}〜A \ref{axim:選択}の(自由出現する変項をすべて普遍量化した)普遍閉鎖体がZFCの公理(型)である.

\begin{axim}[外延性]
\label{axim:外延性}
$
    x=y\con{1}x\in z\case{1}{1}{1}y\in z.
$
\end{axim}

\begin{axim}[一対化,和,冪]
\label{axim:一対化,和,冪}
$
    \classab{x,y},\:\union{x},\:\mathcal{P}(x)\in\univ.
$
\end{axim}

\begin{axim}[無限]
\label{axim:無限}
$
    \mathbb{N}\in\univ.
$
\end{axim}

\begin{axim}[分出]
\label{axim:分出}
$
    x\cap\alpha\in\univ.
$
\end{axim}

\begin{axim}[置換]
\label{axim:置換}
$
    \func\alpha\case{1}{1}{1}\alpha\img x\in\univ.
$
\end{axim}

\begin{axim}[正則性]
\label{axim:正則性}
$
    x\neq \Lambda \ld{.}\supset (\exists y)(y\in x\md{.}x\cap y=\Lambda).
$
\end{axim}

\begin{axim}[選択]
\label{axim:選択}
$
    (\exists y)(
        y\subseteq\mathfrak{E}\con{1}x\cap\barl{\classab{\Lambda}}\subseteq\arg y
    ).
$
\end{axim}

定理は公理から量化論理学の演繹方法を適用して導出される.以下で公理以外の初等的な定理の例をいくつか示す.ただし,そこでの証明は正式な証明のステップを省略して非形式的に展開したものである\footnote{
    T \ref{thm:単一クラス}によって,すべてのクラス$x$について,$x\in\classab{x}\in\univ$.したがって,$(\exists y)(x\in y)$.すなわち,すべてのクラスは集合であり,この体系においては究極クラスは存在しない.
}.

\begin{thm}[同一性の法則]
\label{thm:同一性の法則}
$
    x=y\con{1}Fx\case{1}{1}{0}Fy.
$
\end{thm}
\setcounter{equation}{0}
\begin{pf}

A \ref{axim:一対化,和,冪}により,$\classab{x,x}\in\univ$.つまり,$w = \classab{x,x}$なる$w$が存在する.D \ref{df:同一性}とD \ref{df:対クラス}により,
\[
    (y)(y\in w\case{3}{1}{2}y=x\case{2}{1}{1}y=x).
\]
\kagi{$ y $}を例化して,$x\in w\case{3}{1}{1}x = x$.すると,$x\in w$.他方,第2の仮定とD \ref{df:クラス抽象A}により,$ x\in\classab{z:Fz} $.したがって,
\begin{equation}
    x\in w\cap\classab{z:Fz}.
\end{equation}

次に,A \ref{axim:分出}によって,$w\cap \classab{z:Fz}\in\univ$.それゆえ,$v=w\cap \classab{z:Fz}$である$v$が存在する.すると,D \ref{df:同一性}により,
\begin{equation}
    (u)(u\in v\case{3}{1}{1}u\in w\cap\classab{z:Fz}).
\end{equation}
(1)(2)により$x\in v$.すると,T \ref{thm:同一性の法則}の第1の仮定とA \ref{axim:外延性}により,$y\in v$.
(2)により$y\in w\cap\classab{z:Fz}$.D \ref{df:クラス抽象A}によって,$Fy$.
\end{pf}

\begin{thm}[空集合]
\label{thm:空集合}
$
    \Lambda\in\univ.
$
\end{thm}
\begin{pf}
    A \ref{axim:分出}によって,$x\cap\Lambda\in\univ$だから,$ y = x\cap\Lambda $なる$y$が存在する.D \ref{df:同一性}により,$ \Lambda = x\cap\Lambda $.それゆえ,$ y = \Lambda $.すなわち,$ \Lambda\in\univ $.
\end{pf}

\begin{thm}[単一クラス]
\label{thm:単一クラス}
$
    \classab{\alpha}\in\univ.
$
\end{thm}
\begin{pf}
    $\alpha\in\univ$を仮定すると,$x = \alpha$である$x$が存在する.それゆえ,$ \classab{x}=\classab{\alpha} $.また,A \ref{axim:一対化,和,冪}によって,$\classab{x}=\classab{x,x}\in\univ$.したがって,$\classab{\alpha}\in\univ$.
    $\alpha\notin\univ$を仮定すると,$(x)(x\neq\alpha)$.それゆえ,$\classab{\alpha}=\Lambda$.T \ref{thm:空集合}により,$\classab{\alpha}\in\univ$.
\end{pf}

\begin{thm}[対クラス]
\label{thm:対クラス}
$
    \classab{\alpha,\beta}\in\univ.
$
\end{thm}
\begin{pf}
    T \ref{thm:単一クラス}によって,$\classab{\alpha},\classab{\beta}\in\univ$.すなわち,$ x = \alpha\con{1}y = \beta $である$ x,y $が存在する.すると,A \ref{axim:一対化,和,冪}により,$\union{\classab{\classab{x},\classab{y}}}\in\univ$.D \ref{df:同一性}により,
    \[
        \classab{\alpha,\beta}=\classab{x,y}=\union{\classab{\classab{x},\classab{y}}}.
    \]
    したがって,$ \classab{\alpha,\beta}\in\univ $.
\end{pf}

\begin{thm}[部分クラス]
\label{thm:部分クラス}
$
    \alpha\subseteq\beta\con{1}\beta\in\univ\case{1}{1}{1}\alpha\in\univ.
$
\end{thm}
\begin{pf}
    第2の仮定により,$ x = \beta $なる$ x $が存在する.すると第1の仮定により,$ \alpha = \alpha\cap x $.そして,A \ref{axim:分出}により$ \alpha\cap x\in\univ $.つまり,$ y = \alpha\cap x=\alpha $なる$ y $が存在する.したがって,$\alpha\in\univ$.
\end{pf}

\begin{thm}[合併]
\label{thm:合併}
$
    x\cup y\in\univ.
$
\end{thm}
\begin{pf}
    定義により,$ x\cup y = \union{\classab{x,y}} $.
    T \ref{thm:対クラス}により,$\classab{x,y}\in\univ$.A \ref{axim:一対化,和,冪}により,$ \union{\classab{x,y}}\in\univ $.
    それゆえ,$ x\cup y\in\univ $.
\end{pf}
    
\begin{thm}[直積]
\label{thm:直積}
$
    x\times y\in\univ.
$
\end{thm}
\setcounter{equation}{0}
\begin{pf}
    $ z\in x\times y $と仮定する.するとD \ref{df:直積}により,
    \begin{equation}
        a\in x\con{1}b\in y\con{1}z = \orp{a,b}
    \end{equation}
    となる$a,b$が存在する.それゆえ,$ \classab{a},\classab{a,b}\subseteq x\cup y $であるから,$ \classab{a},\classab{a,b}\in\mathcal{P}(x\cup y)$.したがって,
    \[
        \classab{\classab{a},\classab{a,b}}\subseteq\mathcal{P}(x\cup y).
    \]
    するとD \ref{df:順序対}により,$ \orp{a,b}\in\mathcal{P}(\mathcal{P}(x\cup y)) $.つまり,ある$ z' $が存在して,
    \begin{equation}
        z'=\orp{a,b}\con{1}z'\in \mathcal{P}(\mathcal{P}(x\cup y)).
    \end{equation}
    (1)と(2)から,T \ref{thm:同一性の法則}により,$ z\in\mathcal{P}(\mathcal{P}(x\cup y)) $.

    以上から,$ x\times y\subseteq \mathcal{P}(\mathcal{P}(x\cup y)) $.そして,T \ref{thm:合併}により$x\cup y\in\univ$,A \ref{axim:一対化,和,冪}により$\mathcal{P}(\mathcal{P}(x\cup y))\in\univ$.したがって,T \ref{thm:部分クラス}によって,$x\times y\in\univ$.
\end{pf}

自然科学で必要とされるような数学的法則は,定義を通じてクラス理論の文に還元可能であり,それらは,上記のような証明を積み重ねて行くことで,\ref{ssec:集合論の体系}の公理から演繹可能である\footnote{
    自然科学が要求する数学という目的にとっては,ZFCからA \ref{axim:置換}を除いたZCで足りる.この点,(D \ref{df:階層構造}の記法を使って)その対象領域が$\mathcal{W}\fap 2$で,\kagi{$ \in $}を$ \mathfrak{E}\cap\timex{(\mathcal{W}\fap 2)}{2} $と解釈する任意のモデルにおいて,ZCの公理は真となる.この意味で,自然科学の総体が要求する存在論は$ \mathcal{W}\fap 2 $である.
}.
このことは,それらの定理が真であるために要求される対象が,それらの公理群によって供給されるということも意味している.例えば,実数のクラス$\mathbb{R}$を,$ \mathbb{R}\subseteq\mathcal{P}(\mathbb{N}) $となるように定義することができる\footnote{
    クワイン~\cite[pp.\,113--117]{クワインa}.
}.すると,個々の実数は$\mathbb{N}$の部分クラスであるから,T \ref{thm:部分クラス}によって存在が供給される.$\mathbb{R}$自体についても,A \ref{axim:無限}とA \ref{axim:一対化,和,冪}により,$ \mathcal{P}(\mathbb{N})\in\univ $だから,同様にT \ref{thm:部分クラス}により,$ \mathbb{R}\in\univ $.

ところで,規制$ z $の内部構造,すなわちD \ref{df:クラスの内部構造}に基づく$ \trcl z $のメンバーには,人やその他の物理的対象が含まれる.したがって,物理的対象の存在も供給しなければならない.

\begin{df}
\label{df:クラスの内部構造}
\kagi{$
    \trcl\alpha
$}は\kagi{$
    \union{(
        \ance{
            (\lambda_x\union{x})
        }\img\classab{\alpha}
    )}
$}を表わす.
\end{df}

\noindent この点については,$ \mathbb{R}\in\univ $であることから,T \ref{thm:直積}により,$ \timex{\mathbb{R}}{2},\timex{\mathbb{R}}{3},\timex{\mathbb{R}}{4}\in\univ $.そして,$ \timex{\mathbb{R}}{4} $を四次元時空と同一視すれば,その要素を時空点,その任意の部分クラスを時空領域とみなすことができる.
すると,物理的対象が時空領域を占有する関係$\nu$は,$\nu\img\univ$がすべての物理的対象のクラスであり,$\breve{\nu}\img\univ\subseteq\timex{\mathbb{R}}{4}$であるような関係と考えられる.$\nu$は一対一対応,つまり,$ \func{\nu}\con{1}\func{\breve{\nu}} $であろうから,任意の$x\in\nu\img\univ$はそれが占有する時空領域$\breve{\nu}\fap x$と結局は同一視できる\footnote{
    Quine~\cite{Quine}.
}.任意の時空領域の存在は,$ \timex{\mathbb{R}}{4} $の部分クラスとして,T \ref{thm:部分クラス}によって供給されるから,これとは区別された物理的対象を用意する必要はない.

次に,規制$x$の内部構造$\trcl x$に他の規制類型が含まれ得ることから,規制の存在論を明確化する必要が生じる.そのため\ref{ssec:存在論}である種の階層構造を設定するが,そこでは,$\mathcal{W}\fap 0 = \Lambda$であり,$ 0 \in n \in\mathbb{N}$について$\mathcal{W}\fap n$が,$\mathcal{W}\fap (\breve{\mathrm{S}}\fap n)$から冪集合演算を反復して得られるクラス全ての合併,であるような無限系列$\mathcal{W}$を使う\footnote{
    任意の$n\in\mathbb{N}$について,$ \mathcal{W}\fap n $は,いわゆる累積的階層における水準$ \omega\cdot n $の階層と同一である.
}.

\begin{df}
\label{df:冪集合演算}
\kagi{$
    \mathfrak{P}
$}は\kagi{$
    \lambda_x\mathcal{P}(x)
$}を表わす,
\end{df}

\begin{df}
\label{df:階層構造}
\kagi{$
    \mathcal{W}
$}は\kagi{$
(\imath z)(
    \func{z}\con{1}\arg{z}=\mathbb{N}\con{1}z\fap\Lambda=\Lambda\con{1}z\resl(\mathrm{S}\resl\breve{z})\subseteq\lambda_x\union{(\ance{\mathfrak{P}}\img\classab{x})}
)
$}を表わす.
\end{df}

\noindent 規制の概念が要求する存在論は,$ \mathcal{W}\fap 0,\mathcal{W}\fap 1,\mathcal{W}\fap 2,\dots $のある段階$ \mathcal{W}\fap n $であり,最大でも$\mathcal{W}$の各段階すべての合併$ \union{(\mathcal{W}\img\mathbb{N})} $である.
この点,$\func{\mathcal{W}}$であることが証明可能であり,したがって,A \ref{axim:置換}とA \ref{axim:無限}により,$ \mathcal{W}\img\mathbb{N}\in\univ $.さらに,A \ref{axim:一対化,和,冪}により,$\union{(\mathcal{W}\img\mathbb{N})}\in\univ$.
この場合,$\mathcal{W}$の各段階についても,任意の$\Lambda\neq n\in\mathbb{N}$に対して,$ \Lambda\neq \mathcal{W}\fap n\in\univ $.
この点に関連して一般的に言えば,
\[
    \union{(\ance{\mathfrak{P}}\img\classab{x})}= \union{(\lambda_n(\iter{\mathfrak{P}}{n}\fap x)\img\mathbb{N})}.
\]
すると,$ \func{\lambda_n(\iter{\mathfrak{P}}{n}\fap x)} $であるから,A \ref{axim:無限}とA \ref{axim:置換}により,
\[
    \lambda_n(\iter{\mathfrak{P}}{n}\fap x)\img\mathbb{N}\in\univ.
\]
また,A \ref{axim:一対化,和,冪}により,$ \union{(\lambda_n(\iter{\mathfrak{P}}{n}\fap x)\img\mathbb{N})}\in\univ $.
したがって,$ \union{(\ance{\mathfrak{P}}\img\classab{x})}\in\univ $.

 % 論理
% !TeX root = foundation.tex

\section{因果}
\label{sec:因果}

因果とは何かについての有力な立場として,D.ルイスの反事実的条件法による因果の分析がある.本稿ではこの分析を改造して用いるが,「因果」の日常的な用法との可及的一致を目指すような概念分析は意図しない.本稿での因果の概念は,規制の概念の一部として,(規範の同一性の基準を確立し,それを解析的に記述する枠組みを与えるという)特定の目的のために作られる技術的装置でしかない.日常的用法との一致はその目的を果たすのに必要な範囲でのみ要求される.ルイスの分析を借用する理由は,その直観性と形式化への適応性の他,若干の改造を施せば本稿の目的と前提に適合させられることによる.まず\ref{ssec:反事実的条件法による因果の分析}でルイスの分析を要約する.次に,\ref{ssec:改造}で改造箇所を説明した上で,\ref{ssec:論理式とその解釈}以降で定式化を行う.因果の概念から派生する蓋然性の表現もそこで定義される.

\subsection{反事実的条件法による因果の分析}
\label{ssec:反事実的条件法による因果の分析}

% 因果とは何かについての有力な立場として,D.ルイスの反事実的条件法による因果の分析がある.本稿の因果概念はこの分析を改造したものである.
Lewis~\cite{Lewis}によると,まず,反事実的条件文の真理条件は可能世界の比較類似性によって与えられる.
「もし$A$ならば$B$だろう」を「$ A\text{□→}B$」と書き,これの「$A$」「$B$」にそれぞれ文$p_1,p_2$を代入した結果を$p$と置く.すると,$p$が(世界$w$において)真であるのは,次のいずれかであるときまたそのときに限られる.
\begin{enumerate}[label=(\arabic*)]
    \item $p_1$が真であるような($ w $から)到達可能な世界が存在しない($p$は空虚に真),
    \item $p_1$と$p_2$が共に真である($ w $から)到達可能な世界$w'$が存在して,$ w' $の方が,$p_1$が真で$p_2$が偽であるどの世界よりも,$ w $に類似している.
\end{enumerate}
次に,\kagi{$Oe$}を「(出来事)$e$が生起する」と読むことにすると,出来事間の因果的依存関係は,
\[
    \text{
        「$e_2$は$e_1$に因果的に依存する」は\kagi{$Oe_1\text{□→}Oe_2\con{1}\neg Oe_1\text{□→}\neg Oe_2$}を表わす,
    }
\]
と規定される.そして,因果関係に推移性を持たせるために,$\mathcal{D}=\classab{\orp{e_1,e_2}:\text{$e_2$は$e_1$に因果的に依存する}}$そのものではなく,同一性を除外したそれの祖先関係$\mathcal{D}\resl\ance{\mathcal{D}}$を因果関係とする.

\subsection{改造}
\label{ssec:改造}

ルイスの分析を改造する箇所は以下の通りである.

\begin{enumerate}
    \item \ref{ssec:原初的言語}の原初的言語は反事実的条件法の演算子「\text{□→}」を持たない.また,それを真理関数と量化による標準的な記法に直接還元する方法はない.したがって,反事実的条件文$p$をバイパスして,それを構成する2個の文$p_1,p_2$の間に成立する反事実的依存関係だけを考える.
    \item 反事実的依存関係を定義するのに\ref{ssec:反事実的条件法による因果の分析}の(2)を用いるが,本稿は様相概念を還元する等の動機を持たないため「可能世界」の概念を必要としない.代わりに論理式(文の論理形式を表わす図式)を解釈するモデルを使えば足りる.それに伴い,反事実的依存性が成り立つ対象も文ではなく論理式に置き換える.
    \item 比較類似性の支点となる$w$と類似性の尺度及び到達可能性を解釈空間という関数にまとめて,反事実的依存関係を明示的に解釈空間に相対化する.解釈空間は論理式のモデルに対して数を割り当てる関数であり,それが$0$を割り当てる唯一のモデルが比較類似性の支点$w$の役割を果たす.そして,関数の値である数は$\breve{\epsilon}\fap 0$への類似性の指標となる(小さいほど$\breve{\epsilon}\fap 0$に類似する).
    \item 出来事間の因果ではなく,$a\in x$であることが$b\in y$であることを因果的に決定するような関係を規定する必要があるため,因果的依存関係を出来事ではなく任意の順序対の間の関係として再構成する.そして,このように因果的依存関係を拡張する代わりに,反事実的依存関係を限定する.すなわち,$p_1$が真で$p_2$が偽となるモデル$w\in\breve{\epsilon}\img\univ$が存在するという条件を追加して,これと矛盾する\ref{ssec:反事実的条件法による因果の分析}の(1)を廃棄する.この追加条件は,$p_1$と$p_2$から作られる条件法が(最終的に想定される解釈空間において)論理的/数学的真理となるケースを除外する機能を持つ.
\end{enumerate}

以上を踏まえると,\ref{ssec:論理式とその解釈}以降で定式化される因果概念は次のようなものになる.
まず,解釈空間$\epsilon$に相対的に論理式$p_2$が論理式$p_1$に反事実的に依存するのは,次の両方が成立するときまたそのときに限られる.
\begin{enumerate}[label=(\arabic*)]
    \item $p_1$と$p_2$が共に真であるモデル$w'\in\breve{\epsilon}\img\univ$が存在して,$p_1$が真で$p_2$が偽である任意のモデル$w\in\breve{\epsilon}\img\univ$について,$ \epsilon\fap w'<\epsilon\fap w$,
    \item $p_1$が真で$p_2$が偽であるモデル$w''\in\breve{\epsilon}\img\univ$が存在する.
\end{enumerate}
次に,$\epsilon$に相対的に$\orp{b,y}$が$\orp{a,x}$に因果的に依存するのは,下記の条件※を充たす論理式$p_1,p_2$が存在して,$p_2,\mathbf{N}(p_2)$がそれぞれ$p_1,\mathbf{N}(p_1)$に反事実的に依存するときまたそのときに限られる.ただし,論理式$\zeta$の否定を\kagi{$\mathbf{N}\zeta$}と書く.
\begin{enumerate}
    \item [※] モデル$\breve{\epsilon}\fap 0$において,$p_1$は$a\in x$であることを記述し,$p_2$は$b\in y$であることを記述する.
\end{enumerate}
そして,因果的依存関係を
\[
    \mathcal{D}^{\epsilon}=\classab{\orp{\orp{a,x},\orp{b,y}}:\text{$\epsilon$に相対的に$\orp{b,y}$が$\orp{a,x}$に因果的に依存する}}
\]
とすると,因果関係は$\mathcal{C}^{\epsilon}$である.また,\kagi{$\orp{\orp{\alpha,\beta},\orp{\gamma,\delta}}\in \mathcal{C}^{\epsilon}$}を省略して,
\[
    \orp{\alpha,\beta}\to_{\epsilon}\orp{\gamma,\delta}
\]
と書く.これは「$\epsilon$において,$\alpha\in\beta$であることが$\gamma\in\delta$であることを因果的に決定する」と読まれる.

\subsection{論理式とその解釈}
\label{ssec:論理式とその解釈}

論理式は文の論理形式を表わす図式であり,述語記号に変項を結合して作られる原子式\kagi{$Fxy$},\kagi{$Gy$},\kagi{$Hz$}等と,これらから真理関数と量化によって構成されうる複合的表現が含まれる.
原初的言語と同じ文法構造を持つ言語の文はすべて,ある論理式からその中の述語記号に代入を行うことによって得られる.
今,すべての述語記号について,何らかの方法により
\footnote{
    例えば,述語記号を「 $ (F')\text{,}(F',)\text{,}(F'{,\!,})\text{,}\dots\text{,}(F'')
    \text{,}(F'',)\text{,}(F''{,\!,})\text{,}\dots
    $ 」として,\kagi{$'$}の数が$n$,\kagi{$,$}の数が$i$であるとき,$n$項述語記号の$i$番目とする.
},$\Lambda \neq n$項述語記号の$i$番目等の指定がされていると仮定する(それゆえ,述語記号を$ \orp{n,i} $と同一視してもよい).
また,\kagi{$'$}の数が$\alpha$である変項を\kagi{$\boldsymbol{v}(\alpha)$}と書く.
すると,$ n $項数列$ k $について,$n$項述語記号の$i$番目に,$n$個の変項$\boldsymbol{v}(k {`}0),\dots ,\boldsymbol{v}(k{`}(\Breve{\mathrm{S}}{`}n)) $が後続する原子式は,$ \orp{0,\orp{n,i},k} $と同一視できる.したがって,原子式のクラスは,

\begin{df}
\label{df:原子式のクラス}
\kagi{$
    \mathrm{Atm}
$}は\kagi{$
    \classab{\orp{0,\orp{n,i},k}:
        i\in\mathbb{N}\con{1}\Lambda\neq n=\arg k\con{1}
        k\in\classab{x:x\subseteq\mathbb{N}\uphl\univ}\cap\mathrm{Seq}
    }
$}を表わす,
\end{df}

\noindent と規定することができる.また,任意の論理式$x,y$と$i\in \mathbb{N}$について,$x$の否定は$\langle 1,x \rangle$,前件$x$と後件$y$による条件法は$ \langle 2,x,y \rangle $,変項$ \boldsymbol{v}_i $に関する$x$の普遍量化は,$\langle 3,i,x \rangle$と同一視できる.そして,論理式のクラス$\mathrm{L}$は,原子式を含み,メンバーの否定,条件法,普遍量化もメンバーであるようなすべてのクラスの共通部分であるから,

\begin{df}
\label{df:論理式のクラス}
\kagi{$
    \mathrm{L}
$}は\kagi{$
    \intersect{
        \classab{z:
            \mathrm{Atm}\subseteq z\con{1}
            (x)(y)(i)(
                x,y\in z\con{1}i\in\mathbb{N}\case{1}{1}{1}
                \\\hfill
                \orp{1,x},\orp{2,x,y},\orp{3,i,x}\in z
            )
        }
    }
$}を表わす.
\end{df}

これ以降,以下のより直観的な記法で論理式を指示する.

\begin{df}[原子式]
\label{df:原子式}
\kagi{$
    \mathbf{F}^{\alpha}_{\gamma}\beta
$}は\kagi{$
    \orp{0,\orp{\alpha,\gamma},\beta}
$}を表わす,
\end{df}

\begin{df}[否定]
\label{df:否定}
\kagi{$
    \mathbf{N}\alpha
$}は\kagi{$
    \orp{1,\alpha}
$}を表わす,
\end{df}

\begin{df}[条件法]
\label{df:条件法}
\kagi{$
    \mathbf{C}\alpha\beta
$}は\kagi{$
    \orp{2,\alpha,\beta}
$}を表わす,
\end{df}

\begin{df}[普遍量化]
\label{df:普遍量化}
\kagi{$
    \mathbf{G}_\alpha\beta
$}は\kagi{$
    \orp{3,\alpha,\beta}
$}を表わす.
\end{df}

\noindent 連言\kagi{$\mathbf{K}\alpha\beta$}は\kagi{$\mathbf{N}\mathbf{C}\alpha\mathbf{N}\beta$}であり,
双条件法\kagi{$\mathbf{E}\alpha\beta$}は\kagi{$\mathbf{K}\mathbf{C}\alpha\beta\mathbf{C}\beta\alpha$}である.

論理式はモデル$\orp{u,r}$によって解釈される.対象領域$u$は空でないクラスであり,解釈関数$r$は述語記号に対応する自然数の組$\orp{n,i}$に対して,$ u $上の$ n $項関係を割り当てる関数である\footnote{
    \ref{ssec:論理式とその解釈}は,論理式とその解釈に関する標準的なモデル理論的定義(例えば,清水~\cite[pp.103-107]{清水})を,\ref{ssec:原初的言語}の原初的言語で再記述したものである.
}.
$\beta$が$\alpha$上の$n$項関係であるというのは,$\beta$が$\alpha$のメンバーからなる$n$項順序対のクラスであることを意味するが,これは精密な規定を要する\footnote{
    $\alpha$上の$n$項関係であるという条件だけならば,それは,$m=\breve{S}\fap n$について,$ \iter{(\lambda_z(\alpha\times z))}{m}\fap \alpha $の部分クラスであるということである.しかし,この規定は$\alpha$が存在しないケースでは有効ではない.
}.

まず,順序対の左成分と右成分のそれぞれ一方を抽出する関数を定義する.
\begin{df}
\label{df:左成分}
\kagi{$
    \mathcal{L}
$}は\kagi{$
    \lambda_z(\imath x)(x\in\intersect{z})
$}を表わす,
\end{df}

\begin{df}
\label{df:右成分}
\kagi{$
    \mathcal{R}
$}は\kagi{$
    \lambda_z(\imath x)(x\in\union{z}\cap\barl{\intersect{z}})
$}を表わす.
\end{df}

\noindent 次に,$\alpha{`}\beta$を$\gamma$に交換した$\alpha$である$ \alpha \tbinom{\beta}{\gamma} $を,
\begin{df}
\label{df:系列要素の変換}
\kagi{$
    \alpha\tbinom{\beta}{\gamma}
$}は\kagi{$
    (\alpha\uphr\barl{\classab{\beta}})\cup\classab{\orp{\gamma,\beta}}
$}を表わす,
\end{df}
\noindent と定義する.そして,$ n $ 項系列にそれに対応する $ n $ 項順序対を与える関数$ \mathcal{O} $を導入する.

\begin{df}
\label{df:系列から順序対へ}
\kagi{$
    \mathcal{O}
$}は\kagi{$
    \classab{\orp{y,x}:\Lambda\neq x\in\mathrm{Seq}\con{1}
        (\exists m)(
            m = \breve{S}\fap\arg{x}\con{1}
            x = \lambda_z(\mathcal{L}\fap(\iter{\mathcal{R}}{z}\fap y))\tbinom{m}{\iter{\mathcal{R}}{m}\fap y}
        )
    }
$}を表わす.
\end{df}

\noindent $\func{\mathcal{O}}$であるが,$\func{\breve{\mathcal{O}}}$ではない.$n$項系列$x$の最終要素が順序対である場合,$\mathcal{O}\fap x$は$n$項順序対であり,かつ,$n+1$項順序対である.したがって,$m=\breve{S}\fap n$について,
\[
    x'= x\tbinom{m}{\mathcal{L}\fap(x\fap m)}\cup\classab{\orp{\mathcal{R}\fap(x\fap m),n}}
\]
とすると,$x'$も$\mathcal{O}\fap x$に対して$\breve{\mathcal{O}}$を持つ.
$y$が$n$項順序対であるということは,$(\exists x)(\arg{x}=n\con{1}\mathcal{O}\fap x=y)$ということに帰着する.D \ref{df:系列から順序対へ}は$n=1$の場合でも有意味であるが,その場合,$x$の唯一の系列要素が順序対でない限り$y$は順序対そのものではない.

$\alpha$の要素からなる$\gamma$項順序対をすべて集めたクラスは次のように定義できる.

\begin{df}
\label{df:直積の反復}
\kagi{$
    \timex{\alpha}{\gamma}
$}は\kagi{$
    \classab{\mathcal{O}\fap x:
        x\in\mathrm{Seq}\con{1}\arg x=\gamma\con{1}x\img\univ\subseteq\alpha
    }
$}を表わす.
\end{df}

\noindent $\timex{\alpha}{2}=\alpha\times\alpha$である.$\beta$が$\alpha$上の$n$項関係であることは,$\beta\subseteq\timex{\alpha}{n}$であることを意味する.すると,次の定義によってモデルのクラス$ \mathrm{MD} $が導入される.

\begin{df}
\label{df:モデル}
\kagi{$
    \mathrm{MD}
$}は\kagi{$
    \classab{\orp{x,y}:
        x\neq\Lambda\con{1}\func{y}\con{1}\arg{y}=(\mathbb{N}\cap\barl{\classab{\Lambda}})\times\mathbb{N}\con{1}
        \\\hfill
        (n)(i)(
            \Lambda\neq n\case{1}{1}{1}y\fap\orp{n,i}\subseteq\timex{x}{n}
        )
    }
$}を表わす.
\end{df}

さて,モデルに相対的に論理式を解釈するということは,モデルに相対的にその論理式の真理条件を述べるということを意味する.そして,$\delta\in\mathrm{MD}$において真である論理式の集合$ \mathrm{T}^{\delta} $は,$\delta$における充足関係$\mathrm{SR}^{\delta}$によって規定することができる.$\mathrm{SR}^{\delta}$は,$\mathcal{L}\fap\delta$上の対象列とそれが$\delta$において充足する論理式との関係である.この点,ある対象列が論理式の無限クラスのすべてのメンバーを充足すると言う場合,そこでは無限個の変項が出現し得るから,有限系列では足りなくなる.そこで最初に,$\alpha$上の対象列として,$\alpha$のメンバーからなる無限系列を導入する.
\begin{df}
\label{df:対象列}
\kagi{$
\mathrm{seq}^\alpha
$}は\kagi{$
    \classab{s:\func{s}\con{1}s\img\univ\subseteq\alpha\con{1}\arg s=\mathbb{N}}
$}を表わす.
\end{df}

\noindent 次に,原子式,真理関数,普遍量化のそれぞれについて,充足関係の断片を定義する.

\begin{df}
\label{df:原子式の充足関係}
\kagi{$
    \mathrm{sra}^\delta
$}は\kagi{$
    \classab{\orp{s,\mathbf{F}^n_{i}k}:
        \mathcal{O}\fap(s\resl k)\in (\mathcal{R}\fap\delta)\fap\orp{n,i}
    }
$}を表わす,
\end{df}

\begin{df}
\label{df:真理関数の充足関係}
\kagi{$
    \mathrm{srt}\,\gamma
$}は\kagi{$
    \classab{\orp{s,\mathbf{N}x}:\orp{s,x}\notin \gamma}\cup\classab{\orp{s,\mathbf{C}xy}:\orp{s,x}\in\gamma\case{1}{1}{1}\orp{s,y}\in\gamma}
$}を表わす,
\end{df}

\begin{df}
\label{df:普遍量化の充足関係}
\kagi{$
    \mathrm{srg}^\delta\,\gamma
$}は\kagi{$
    \classab{\orp{s,\mathbf{G}_ix}:
        (a)(a\in \mathcal{L}\fap\delta\case{1}{1}{1}\orp{s\tbinom{i}{a},x}\in\gamma)
    }
$}を表わす.
\end{df}

\noindent これらの断片を繋ぎ合わせて充足関係を定義する.

\begin{df}
\label{df:充足関係}
\kagi{$
    \mathrm{SR}^\delta
$}は\kagi{$
    \intersect{\classab{z:
        (\mathrm{seq}^{(\mathcal{L}\fap\delta)}\times\mathrm{L})\cap
        (\mathrm{sra}^{\delta} \cup \mathrm{srt}\,z \cup 
        \mathrm{srg}^{\delta}\,z)\subseteq z
    }}
$}を表わす.
\end{df}

\noindent モデル$\delta$における真理は,$\delta$の対象領域のすべての対象列によって充足される論理式である.したがって,

\begin{df}
\label{df:真理集合}
\kagi{$
    \mathrm{T}^\delta
$}は\kagi{$
    \classab{x:
        (s)(
            s\in\mathrm{seq}^{(\mathcal{L}\fap\delta)}\case{1}{1}{1}
            \orp{s,x}\in \mathrm{SR}^\delta
        )
    }
$}を表わす.
\end{df}

\noindent 関連して,論理式の集合が論理式を論理的に含意する関係は,
\begin{df}
\label{df:論理的含意関係}
\kagi{$
    \mathrm{imp}
$}は\kagi{$
    \classab{\orp{x,y}:x\subseteq\mathrm{L}\con{1}y\in\mathrm{L}\con{1}
        (m)(s)(m\in\mathrm{MD}\con{1}s\in\mathrm{seq}^{(\mathcal{L}\fap m)}\case{1}{1}{1}
        \\\hfill
        (l)(l\in x\case{1}{1}{1}\orp{s,l}\in \mathrm{SR}^m)\case{1}{0}{1}\orp{s,y}\in \mathrm{SR}^m
        )
    }
$}を表わす,
\end{df}
\noindent と規定できる.すると妥当式(すべてのモデルのすべての対象列によって充足される論理式)は,$ \brevel{\mathrm{imp}}\img\classab{\Lambda} $のメンバーであり,1個の論理式が含意する関係は,$ \brevel{\lambda_x\classab{x}}\resl\mathrm{imp} $である.

$\mathrm{SR}^\delta$を使って,モデル$\delta$において開放論理式$\beta$が言及するクラス($\delta$における$\beta$の外延)という概念を規定することができる.
まず,変項$ \boldsymbol{v}(\alpha) $が自由出現する論理式のクラスを,
\begin{df}
\label{df:開放論理式のクラス}
\kagi{$
    \mathrm{L}^\alpha
$}は\kagi{$
    \union{
        \classab{z:
            (x)(x\in\mathrm{Atm}\con{1}\alpha\in (\iter{\mathcal{R}}{2}\fap x)\img\univ\case{1}{1}{1}x\in z)\con{1}
            \\\hfill
            (x)(y)(i)(
                x\in z\con{1}y\in\mathrm{L}\con{1}\alpha\neq i\in\mathbb{N}
                \case{1}{1}{1}\mathbf{N}x,\mathbf{C}xy,\mathbf{C}yx,\mathbf{G}_ix\in z
            )
        }
    }
$}を表わす,
\end{df}
\noindent と定義する.次に,論理式$ \beta $に自由出現する変項の「~$'$~」の数を小さい順に並べた系列を導入する.

\begin{df}
\label{df:変項列}
\kagi{$
    \mathrm{var}\,\beta
$}は\kagi{$
    (\imath z)(z\in\mathrm{Seq}\con{1}
        z\img\univ = \mathbb{N}\cap\classab{x:\beta\in\mathrm{L}^x}\con{1}
        z\resl \breve{\mathrm{S}}\resl \breve{z}\subseteq \mathfrak{E}
    )
$}を表わす.
\end{df}

\noindent すると,$\delta\in\mathrm{MD}$において$\beta\in\mathrm{L}$が言及するクラス$\delta\exten\beta$は,

\begin{df}
\label{df:}
\kagi{$
    \delta\exten\beta
$}は\kagi{$
    \classab{\mathcal{O}\fap(s\resl v):\Lambda\neq v= \mathrm{var}\,\beta\con{1}
        \orp{s,\beta}\in\mathrm{SR}^\delta
    }
$}を表わす,
\end{df}

\noindent と規定することができる.$\delta\exten\beta$は,$\delta$で解釈された$\beta$が$x$について真であるような対象$x$のクラスである.

\subsection{因果と蓋然性}
\label{ssec:因果と蓋然性}

因果がそれに相対化される解釈空間の特徴づけを確定しよう.クラス$\epsilon$が解釈空間であるとき,$\epsilon$はモデルに数を与える関数であるが,それが$0$を与える唯一のモデル$\breve{\epsilon}\fap \Lambda$を「開始点」と呼ぶ.単純化のため,任意の$m_1,m_2\in\arg\epsilon$について,それらの対象領域は同一であるとする($\mathcal{L}\fap m_1 = \mathcal{L}\fap m_2$).
これは$\arg\epsilon$の左域が単一クラスであることを意味するが,その唯一のメンバーを指示するために,
\begin{df}
\label{df:トライアングル}
\kagi{$
    \trgl{\alpha}
$}は\kagi{$
    (\imath x)(\alpha\img\univ = \classab{x})
$}を表わす,
\end{df}
\noindent とすると,単に,$ \trgl{\arg\epsilon}\neq \Lambda $ということである.この点と関連して,$m_1,m_2$はいずれも,$2$項述語記号の$0$番目に対して同一の解釈を,対象領域上の要素関係を与えるものとする.すなわち,
\[
    (\mathcal{R}\fap m_1)\fap \orp{2,0} = (\mathcal{R}\fap m_2)\fap \orp{2,0} = \mathfrak{E}\cap\timex{(\trgl{\arg\epsilon})}{2}.
\]
したがって,$ \mathcal{W}\fap 2\subseteq\trgl{\arg\epsilon} $である限り,\kagi{$ \in $}を$ \mathbf{F}^2_0 $に置き換えた\ref{ssec:集合論の体系}のA \ref{axim:外延性}〜A \ref{axim:分出}が,したがって,自然科学に必要な全ての数学的真理(の図式化)が$m_1,m_2$において真となる.

また,「$m_1$の方が$m_2$よりも$\epsilon$の開始点に類似する」を,\kagi{$ m_1 \prec_{\epsilon} m_2 $}と書くことにすると,任意の$m_1,m_2\in\arg\epsilon$について,
\[
   m_1 \prec_{\epsilon} m_2 \case{3}{1}{1} \epsilon\fap m_1 < \epsilon\fap m_2.
\]
つまり,開始点への類似性に関する$\arg\epsilon$のメンバーの大小関係は,$\epsilon$がそれに付与する数の大小関係に帰着する.再び単純化のため,この数は差し当たり自然数であるとする\footnote{
    もし類似性の大小関係$\alpha$を稠密化($\alpha\subseteq\alpha\resl\alpha$)するなら,$\epsilon\img\univ$のメンバーとして,少なくとも有理数が要求される.また,非可算個のモデルに対して非可算個の類似性段階を設定するなら実数が必要になる.
}.したがって,$\epsilon\img\univ\subseteq\mathbb{N}$.

なお,$\epsilon$は有意な蓋然性のレベルを表わす指標を持っており,それは,$n$より小さい数はすべて$\epsilon\img\univ$のメンバーであるが,$n$自体はそのメンバーではないような唯一の数$n$である.すなわち,
\begin{df}
\label{df:有意指標}
\kagi{$
    \indx{\epsilon}
$}は\kagi{$
    (\imath n)(n\in\mathbb{N}\con{1}n\subseteq\epsilon\img\univ\con{1}n\notin\epsilon\img\univ)
$}を表わす,
\end{df}
\noindent とすると,$\indx{\epsilon}\neq \Lambda$.さらに,任意の$i\in\epsilon\img\univ$について,$i$より小さい$\indx{\epsilon}$以外の数はすべて$\epsilon\img\univ$のメンバーとなる.

以上から,$\epsilon$が解釈関数であることの定義は以下のようになる.
\begin{df}
\label{df:解釈空間}
\kagi{$
    \mathrm{sim}\,\epsilon
$}は\kagi{$
    \func \epsilon\con{1}
    \arg\epsilon\subseteq\classab{x:x\in\mathrm{MD}\con{1}(\mathcal{R}\fap x)\fap \orp{2,0} = \mathfrak{E}\cap\timex{(\trgl{\arg\epsilon})}{2}}\con{1}
    \\\hfill
    \trgl{\arg\epsilon}\neq\Lambda \neq \breve{\epsilon}\fap\Lambda \con{1}
    \indx{\epsilon} \neq\Lambda\con{1}
    (\barl{\classab{n}}\uphl\mathfrak{E})\img(\epsilon\img\univ)\subseteq (\epsilon\img\univ)
$}を表わす.
\end{df}

\ref{ssec:改造}の時点では,解釈空間$\epsilon$に相対的に論理式$p_2$が論理式$p_1$に反事実的に依存するのは,\ref{ssec:改造}の(1)(2)の両方が成立するときとされていた.現在の記法で書くと,
\begin{gather}
    (\exists m)[
        m\in\arg\epsilon\con{1}p_1,p_2\in \mathrm{T}^m\con{1}
        (w)(w\in\arg \epsilon\con{1}p_1,\mathbf{N}p_2\in \mathrm{T}^w\case{1}{1}{1}\epsilon\fap m\in\epsilon\fap w)
    ],\tag{1}\\
    (\exists u)(u\in\arg\epsilon\con{1}p_1,\mathbf{N}p_2\in \mathrm{T}^u).\tag{2}
\end{gather}

\noindent さらに省略記法を追加する.$\beta$が真となる$\arg\epsilon$のメンバーの集合$ \epsilon\dagger\beta $を,
\begin{df}
\label{df:真理化する到達可能モデルの集合}
\kagi{$
    \epsilon\dagger\beta
$}は\kagi{$
    \classab{w:w\in\arg\epsilon\con{1}\beta\in\mathrm{T}^w}
$}を表わす,
\end{df}
\noindent と規定する.すると次の定義によって,$\epsilon$に相対的に$t$が$s$に反事実的に依存する$\orp{s,t}$のクラス$\mathfrak{T}^\epsilon$が導入される.

\begin{df}
\label{df:反事実的依存関係}
\kagi{$
    \mathfrak{T}^\epsilon
$}は\kagi{$
    \classab{\orp{s,t}:
        (\exists m)[
            m\in(\epsilon\dagger\mathbf{K}st)\con{1}
            \epsilon\fap m\in\intersect{
                (\epsilon\img(\epsilon\dagger\mathbf{K}s\mathbf{N}t))
            }
        ]\con{1}
        % \\\hfill
        (\epsilon\dagger\mathbf{K}s\mathbf{N}t)\neq \Lambda
    }
$}を表わす.
\end{df}

続いて,モデル$\delta$において$ a\in x $を記述する論理式が,$ \orp{a,x} $に対して持つ関係$ \mathcal{K}^\delta $を導入する.
まず,開放論理式の組$ \orp{p,q} $から結合図式を生成する関数$ \mathcal{S} $を,

\begin{df}
\label{df:結合図式}
\kagi{$
    \mathcal{S}
$}は\kagi{$
    \classab{\orp{z,\orp{p,q}}:p,q\in\mathrm{L}\con{1}
        \mathrm{var}\,p = \classab{2,0}\con{1}\mathrm{var}\,q = \classab{1,0}\con{1}
        \\\hfill
        (\exists k)[k = \classab{\orp{2,0},\orp{1,1}}\con{1}
            z = \mathbf{N}\mathbf{G}_1\mathbf{N}(\mathbf{K}(\mathbf{G}_2 \mathbf{E}\mathbf{F}^{2}_{0}k p) q)
        ]
    }
$}を表わす,
\end{df}

\noindent と規定する.
例えば,$ p = \text{\kagi{$Fx''$}}\con{1}q = \text{\kagi{$Gx'$}} $とすると,
\[
\mathcal{S}\fap\orp{p,q} = \text{\kagi{$ 
    (\exists x')[(x'')(x''\in x' \case{3}{1}{0}Fx'')\con{1}Gx']
$}}
\]
であり,これは\kagi{$ \classab{x''':Fx'''}\in\classab{x'''':Gx''''} $}と等値である.すると,

\begin{df}
\label{df:結合図式と順序対}
\kagi{$
    \mathcal{K}^\delta
$}は\kagi{$
    \classab{\orp{\mathcal{S}\fap\orp{p,q},\orp{x,y}}:
        \delta\exten p = x\con{1}\delta\exten q = y\con{1}p,q\in\arg\mathcal{S}
    }
$}を表わす.
\end{df}

解釈関数$\epsilon$に相対的な因果的依存関係$\mathcal{D}^{\epsilon}$は,$\mathfrak{T}^\epsilon$と$\mathcal{K}^{(\breve{\epsilon}\fap\Lambda)}$に基づいて構成される.すなわち,

\begin{df}
\label{df:因果的依存関係}
\kagi{$
    \mathcal{D}^{\epsilon}
$}は\kagi{$
    \classab{\orp{x,y}:
        \orp{s,x},\orp{t,y}\in \mathcal{K}^{(\breve{\epsilon}\fap\Lambda)}\con{1}
        \orp{s,t},\orp{\mathbf{N}s,\mathbf{N}t}\in\mathfrak{T}^\epsilon
    }
$}を表わす.
\end{df}

\noindent $\orp{\orp{a,x},\orp{b,y}}\in\mathcal{D}^\epsilon$,すなわち,$ \orp{b,y} $が$ \orp{a,x} $に因果的に依存するのは,
$ \breve{\epsilon}\fap\Lambda $において$ a\in x $を記述する論理式$ s $と,$ \breve{\epsilon}\fap\Lambda $において$ b\in y $を記述する論理式$ t $が存在して,
$ t,\mathbf{N}t $がそれぞれ,$ s,\mathbf{N}s $に反事実的に依存するとき,かつ,そのときに限られる.

そして,因果関係は因果的依存関係の(同一性を除外した)祖先関係である.因果関係を指示する記法と,因果関係に属することを記述する記法をそれぞれ導入する.

\begin{df}
\label{df:因果関係}
\kagi{$
    \mathcal{C}^\epsilon
$}は\kagi{$
    \mathcal{D}^\epsilon\resl\ance{(\mathcal{D}^\epsilon)}
$}を表わす,
\end{df}

\begin{df}
\label{df:因果記法}
\kagi{$
    \alpha\to_{\epsilon}\beta
$}は\kagi{$
    \mathrm{sim}\,\epsilon\con{1}\orp{\alpha,\beta}\in \mathcal{C}^{\epsilon}
$}を表わす.
\end{df}

\noindent $ \orp{a,x}\to_{\epsilon}\orp{b,y} $であることは,$\orp{a,b}$が,$x$と$y$を因果的に結合させる背景条件$ x\bkg{\epsilon}y $の要素であることとして記述することもできる.
\begin{df}
\label{df:背景条件}
\kagi{$
    \alpha\bkg{\epsilon}\beta
$}は\kagi{$
    \classab{\orp{a,b}:
        \orp{a,\alpha}\to_{\epsilon}\orp{b,\beta}
    }\cap\timex{(\trgl{\arg\epsilon})}{2}
$}を表わす\footnote{
    $\timex{(\trgl{\arg\epsilon})}{2}$への相対化は,後述する蓋然性もそうであるが,因果の背景条件自体を惹起する因果関係の項として,$\breve{\epsilon}\fap\Lambda$において言及可能にするためである.勿論その都度何らかのクラス$z\in\univ$に制限すればいずれにせよ言及可能であり,場合によってはクラスの大きさを抑えるために,$\timex{(\trgl{\arg\epsilon})}{2}$より小さなクラスに相対化する必要がある.しかし,定義上一般的に相対化しておく方が記法を簡単にできるケースがある.
}.
\end{df}

\noindent 因果の定義により,\ref{ssec:集合論の体系}の体系において以下の基本法則が証明可能である.
\[
    \textbf{因果の基本法則}\qquad 
    \orp{a,x}\to_\epsilon\orp{b,y}\case{1}{1}{2}a\in x\case{3}{1}{1}b\in y.
\]
これによると,$a\in x\con{1}b\notin y$,または,$a\notin x\con{1}b\in y$であるとき,$a\in x$であることは$b\in y$であることを因果的に決定しない.

ところで,ある条件が実際には実現しなかったがそれが実現する蓋然性があり,かつ,そのような蓋然的な状態それ自体が他の条件に因果的に依存している,と言うのが自然なケースがある.
例えば,$a$が$b$を狙撃したが僅かに射線がそれて銃弾が$b$の至近距離に着弾した場合,$b$に銃弾が当たる危険が実在していて,かつ,その危険状況は$a$の行動に因果的に依存している(と言うのが自然である).そこで,解釈空間$ \epsilon $に相対的に$ x $が$ x\in \alpha $である蓋然性を持つという事態を表現するために,記法\kagi{$x\in\alpha^{:\epsilon}$}を用いる.
これは,$ x\in\alpha $であるか,または,
\[
   \text{
        $ \breve{\epsilon}\fap\Lambda $において$ x\in\alpha $であることを記述する論理式$s$が存在して,$\epsilon\dagger s\neq\Lambda$
   }
\]
であることを意味している.すなわち,

\begin{df}
    \label{df:蓋然性}
    \kagi{$
        \alpha^{:\epsilon}
    $}は\kagi{$
        \classab{x:x\in\alpha\case{2}{1}{0}
            (\exists s)(
                \orp{s,\orp{x,\alpha}}\in \mathcal{K}^{(\breve{\epsilon}\fap\Lambda)}\con{1}
                \epsilon\dagger s\neq\Lambda
            )
        }\cap\trgl{\arg\epsilon}
    $}を表わす.
\end{df}

もっとも,$\epsilon\dagger s$には開始点への類似性が極めて希薄なモデルも含まれ得るから,$\prob{\alpha}{\epsilon}$それ自体はまだ有意な蓋然性ではない.
他方,$(\exists n)(\indx{\epsilon}\in n\in\epsilon\img\univ)$であり,それゆえ,$\indx{\epsilon} \neq \epsilon\img\univ$である場合,
$(\indx{\epsilon}\uphl\epsilon)\dagger s$には,開始点への類似性の度合いが一定レベル以上の($\epsilon$による付値が$\indx{\epsilon}$より小さい)モデルのみが含まれる.この意味で,$ \indx{\epsilon} $は$ \epsilon $において有意な蓋然性のレベルを表わしている.
したがって,指標$\indx{\epsilon}$またはより高いレベルの蓋然性を言う場合には,左域を$n\subseteq\indx{\epsilon}$で制限した$n\uphl\epsilon$における蓋然性$ \prob{\alpha}{(n\uphl\epsilon)} $を使う.
なお極限的なケースとして,$ \alpha\subseteq\trgl{\arg\epsilon} $である限り,$ \prob{\alpha}{(1\uphl\epsilon)}=\alpha $である\footnote{
    現に$x\in\alpha$であり,行動がなくても$x\in\alpha$だったであろうケースでは,行動がなくても$ x\in\prob{\alpha}{(\indx{\epsilon}\uphl\epsilon)} $だったであろうから,蓋然性に対しても因果関係はない.ただし,実際には生じていない別の結果,$x$と時間的空間的に近接する$x'$について,$x'\in\prob{\alpha}{(\indx{\epsilon}\uphl\epsilon)}$等に対しては,同一の行動が因果関係を持つかもしれない.
}.
ただし,D \ref{df:蓋然性}の\kagi{$ x\in\alpha\case{2}{1}{0} $}という条件を外すと,$\alpha$が非可算のケースでは($\alpha$に言及不可能な対象が含まれるから)$ \alpha\nsubseteq \prob{\alpha}{(1\uphl\epsilon)} $となり,この同一性は成り立たなくなる.

\subsection{解釈空間}
\label{ssec:解釈空間}

\subsubsection{解釈空間の選択}
\label{sssec:解釈空間の選択}

因果関係は解釈空間に相対的であり,どのような解釈空間を選択するかに応じて,可変的な内容を持つ.この意味で,因果概念は文脈依存的であるが,規制の概念を使用する通常のケースにおいては,下記の1〜4の条件を充たす解釈空間$ \epsilon $が想定されている.すなわち,規制の概念を使用する文脈において,仮に\kagi{$ \beta $}が解釈空間を表わす型文字として現れているなら,\kagi{$ \beta $}は条件1〜4を充たす解釈空間を指示するクラス抽象体の位置を表わしている.
\begin{enumerate}
    \item $ (\exists n)(\indx{\epsilon}\in n \in\epsilon\img\univ). $
    \item $ (\imath n)(2\in n\in\mathbb{N}\con{1}\trgl{\arg{\epsilon}}=\mathcal{W}\fap n)\neq\Lambda. $
    \item $ (x)(y)(n)[n\in\mathbb{N}\con{1}(\exists p)(\exists q)(
        (\breve{\epsilon}\fap\Lambda)\exten p = x\con{1}(\breve{\epsilon}\fap\Lambda)\exten q = y
    )\case{1}{1}{0}
    \\\hfill
    (\exists r)(\exists r')(r,r'\in\mathrm{L}\con{1}(\breve{\epsilon}\fap\Lambda)\exten r = x\bkg{\epsilon}y
    \con{1}(\breve{\epsilon}\fap\Lambda)\exten r' = \prob{x}{n\uphl\epsilon}
    )
    ]. $
    \item ある全体論的概念図式$g$が存在して,$\epsilon$は$g$に適合する.
\end{enumerate}

条件1により,$ \arg\epsilon $のメンバーには,開始点から,蓋然性の有意レベル$\mathfrak{h}\epsilon$を超えて遠ざかるモデルが存在する.この条件の機能は,同一の解釈空間の中で,有意なレベルの蓋然性とそれに至らない蓋然性を区別できるようにするということである.例えば,最初から解釈空間として$ \mathfrak{h}\epsilon\uphl\epsilon $を使う場合はこの区別ができない.

条件2により,$ \arg\epsilon $のメンバーに共通する対象領域は,$ \mathcal{W}\fap 0,\mathcal{W}\fap 1,\mathcal{W}\fap 2,\dots $のある有限段階$\mathcal{W}\fap n$である.これは規制の概念の存在論的前提,すなわち規制の概念の使用を含むような,ある規制体系に関連する言明が真であるために必要な対象を全て含む領域を表わしている.つまり$n\in\mathbb{N}$の値は当の文脈で扱われる規制体系に応じて異なる.これも解釈空間の文脈依存性の一要素であるが,規制類型の階層構造のため自然科学の存在論$\mathcal{W}\fap 2$よりも高階になる.この点については後述するが,$n$は,当該規制体系における階層構造の最大値を超える段階であれば何でもよい.

条件3により,$\breve{\epsilon}\fap \Lambda$で$x,y$が言及可能であるならば,($\Lambda$かもしれない)背景条件$x\bkg{\epsilon}y$と,任意の$n\in\mathbb{N}$に関する蓋然性$ \prob{x}{n\uphl\epsilon} $も言及可能である.それゆえ,背景条件や蓋然性を惹起する因果関係が一般的に可能になる.ただし,$x\bkg{\epsilon}y$が言及可能であれば,それを惹起する因果関係の背景条件が言及可能になる.すると,その背景条件を惹起する因果関係の背景条件が言及可能になる.以下同様に続くから,この条件は少なからず理想化を含んでいる.そして,条件4で$\epsilon$は全体論的概念図式$g$に適合しなければならないから,$g$が充たすべき条件は$\epsilon$が条件3を充たすことと実質的に連動している.

条件4は,ある論理式の集合$g$について,①$g$が全体論的概念図式と言えるための条件と,②$\epsilon$が$g$に適合すると言えるための条件に分解される.

\subsubsection{全体論的概念図式}
\label{sssec:全体論的概念図式}

論理式の集合$g$が全体論的概念図式であるということは,次のことを意味する.まず,\ref{ssec:原初的言語}の原初的言語に述語を(非限定的に)追加して,拡張言語(の文の集合)$\mathfrak{L}$を構成する.ただし,述語はすべて$ n \neq \Lambda $項述語の$ i $番目という形式でリスト化されていると想定する.要素関係を表わす\kagi{$\in$}は$2$項述語の$0$番目である.
次に,$\mathfrak{L}$の文について,対象領域$\kappa$に相対的な充足と真理を定義する.そして,$\trgl{\arg\epsilon}$に相対的に真である文の集合$ \mathcal{T}^\epsilon\subseteq\mathfrak{L} $を考える(真理集合).$g$は$ \mathcal{T}^\epsilon $の図式化,つまりそのメンバーに出現する原子文を,$ n $項述語の$ i $番目を述語記号$ \orp{n,i} $に置換することによって,原子論理式に置換した集合である.
すなわち,この置換条件を充たす変換$ f $が存在して,
\[
    \func f\con{1}\func\breve{f}\con{1}f\img\univ \subseteq\mathrm{L}\con{1}\breve{f}\img\univ = \mathfrak{L}\con{1}
    f\img(\mathcal{T}^\epsilon) = g
\]
であるとき,かつそのときに限り,$ g $は全体論的概念図式である.

まず,以下の要領で\ref{ssec:原初的言語}の原初的言語に述語を追加する\footnote{
    述語が追加されても\ref{ssec:集合論の体系}の公理群は維持される.ただし,A \ref{axim:分出}の\kagi{$ \alpha $}は,変項または$\mathfrak{L}$の開放文で作られたクラス抽象体の位置を表わす型文字として再解釈される.したがって,$\mathfrak{L}$が述語「は人である」を持つならば,再解釈されたA \ref{axim:分出}によって,
    $ \classab{x:x\text{ は人である}}\cap\mathcal{P}(\timex{\mathbb{R}}{4})\in\univ $.
}.ただし,追加される述語は最終的に有限個であると仮定する.

\begin{enumerate}
    \item 受容されている科学理論の公理とその技術的・補助的な前提の他,実験状況の要約や常識的一般化を表現するのに必要な述語を追加する.ただし,同一性述語\kagi{$ = $}と数学的述語は原初的言語で定義可能であるから追加されない.
    \item $\mathfrak{L}$の開放文が($\trgl{\arg\epsilon}$に相対的に)言及する任意の$x,y$と$n\in\mathbb{N}$について,$ x\bkg{\epsilon}y $または$ \prob{x}{n\uphl\epsilon} $を($\trgl{\arg\epsilon}$に相対的に)言及するために必要ならば,さらに述語を追加する.
    \item 2の$ x\bkg{\epsilon}y $のメンバーの左右の成分,または$ \prob{x}{n\uphl\epsilon} $のメンバーになるような巨視的な物理的対象(出来事や人間など)を指示するために必要ならば,さらに(日常的な)述語を追加する\footnote{
        対象$x$を指示する名前は,$x$についてのみ真である述語\kagi{$ A $}を導入して,単称記述\kagi{$ (\imath y)Ay $}で代用できる.あるいは,特別な述語を導入しなくても日付や位置座標等で指定可能な場合もある.
    }.
\end{enumerate}
条件2は,解釈空間$\epsilon$が\ref{sssec:解釈空間の選択}の条件3を充たすために要求される.$\mathfrak{L}$の開放文が(対象領域に相対的に)言及するクラスという概念については後述する.背景条件$ x\bkg{\epsilon}y $を言及する開放文は,構造的な規定であれ何であれ,$x$と$y$の相互連間の持続を説明するような規定条件である.また,蓋然性$ \prob{x}{n\uphl\epsilon} $を言及する開放文は,例えば,確率に関する数的関数$y$について,$ \brevel{(\classab{n}\uphl y)}\img\univ $を言及するものかもしれない\footnote{
    開放文\kagi{$ \orp{a,x}\to_{\epsilon}\orp{b,y} $}が$\mathfrak{L}$の文に還元可能なわけではない.個別の条件$x,y$が特定可能であれば,$ x\bkg{\epsilon}y $を言及する$ \mathfrak{L} $開放文が存在するというだけである.
}.

次に,以下の例のように,追加される述語の分だけ原子文の構成方法を追加する.
\begin{align*}
    &\text{「$\alpha$は人である」の\kagi{$ \alpha $}に変項を代入した結果は文である,}\\
    &\text{「$\alpha$は$\beta$を愛する」の\kagi{$ \alpha $}と\kagi{$ \beta $}に変項を代入した結果は文である.}
\end{align*}
そして,$\mathfrak{L}$の文に対して,対象領域$\kappa$に相対化された充足と真理を(再帰的に)定義する\footnote{
    $\kappa\in\univ$ならば,$\kappa$に相対化された充足と真理を直接的に定義することができる.ただし,追加される述語のリストが確定している必要がある.
}.まず原子文の充足は,以下の例のように各述語ごとに定義される.
\begin{enumerate}[label=(\arabic*)]
    \item 任意の$ i,j\in\mathbb{N} $と$ s\in\mathrm{seq}^{\kappa} $について,$s$が,「$\alpha\in\beta$」の\kagi{$ \alpha $}に変項$ \boldsymbol{v}(i) $を,\kagi{$ \beta $}に$ \boldsymbol{v}(j) $を代入した結果を充足するのは,$ s\fap i\in s\fap j $であるとき,かつそのときに限られる,
    \item 任意の$ i\in\mathbb{N} $と$ s\in\mathrm{seq}^{\kappa} $について,$s$が,「$\alpha$は人である」の\kagi{$ \alpha $}に$ \boldsymbol{v}(i) $を代入した結果を充足するのは,$ s\fap i $は人であるとき,かつそのときに限られる.
\end{enumerate}
次に,複合文の充足を再帰的に定義する.
\begin{enumerate}[label=(\arabic*),start=3]
    \item 任意の$p\in\mathfrak{L}$と$ s\in\mathrm{seq}^{\kappa} $について,$s$が,\kagi{$ \neg P $}の\kagi{$ P $}に$p$を代入した結果を充足するのは,$ s $が$p$を充足しないとき,かつそのときに限られる,
    \item 任意の$p,q\in\mathfrak{L}$と$ s\in\mathrm{seq}^{\kappa} $について,$s$が,\kagi{$ P\supset Q $}の\kagi{$ P $}に$p$を,\kagi{$ Q $}に$q$を代入した結果を充足するのは,$ s $が$p$を充足するならば$s$が$q$を充足するとき,かつそのときに限られる,
    \item 任意の$p\in\mathfrak{L}$と$ i\in\mathbb{N} $,及び$ s\in\mathrm{seq}^{\kappa} $について,$s$が,\kagi{$ (\alpha)P $}の\kagi{$ \alpha $}に$ \boldsymbol{v}(i) $を,\kagi{$ P $}に$p$を代入した結果を充足するのは,すべての$ x\in \alpha $について,$s\tbinom{i}{x}$が$p$を充足するとき,かつそのときに限られる.
\end{enumerate}
$p\in\mathfrak{L}$が対象領域$ \kappa $において真であるのは,すべての$s\in \mathrm{seq}^{\kappa}$について$s$が$p$を充足するとき,かつそのときに限られる.
また,開放文$q\in\mathfrak{L}$が$\kappa$において言及するクラスは,$q$に自由出現する変項の\kagi{$ ' $}の数を小さい順に並べた系列$v$について,
\[
   \classab{\mathcal{O}\fap(s\resl v):
        \Lambda\neq v\con{1}s\in \mathrm{seq}^{\kappa}\text{ は }q\text{ を充足する}
   }.
\]

ところで,真理集合$ \mathcal{T}^\epsilon $には\ref{ssec:集合論の体系}のA \ref{axim:外延性}〜A \ref{axim:分出}が含まれる.したがって,それらの図式化である論理式の集合$a$が存在して,$a\subseteq g$.\kagi{$\in$}は述語記号$\orp{2,0}$に置換され,また,解釈空間の定義上,任意の$m\in\arg\epsilon$について,
\[
    \mathcal{R}\fap m\fap\orp{2,0}=\mathfrak{E}\cap\timex{(\trgl{\arg\epsilon})}{2}.
\]
したがって,$ a\subseteq\mathrm{T}^m $.さらに,$ \brevel{\mathrm{imp}}\img\classab{a}\subseteq\mathrm{T}^m $である.関連して,$ \orp{a,r}\in\mathrm{imp} $なる$r$が$p$を前件$q$を後件とする条件法である場合,反事実的依存の定義により,$q$は$p$に反事実的に依存しない.つまり,数学的法則は因果関係に直結しない.

\subsubsection{適合性の条件}
\label{sssec:適合性の条件}

解釈空間$ \epsilon $が全体論的概念図式$ g $に適合していると言えるためには,まず次の2個の条件を充たさなければならない.
\begin{enumerate}[label=(\arabic*)]
    \item $g\subseteq \mathrm{T}^{(\breve{\epsilon}\fap\Lambda)}$($\breve{\epsilon}\fap\Lambda$ は$g$のモデル),
    \item $g$のどのメンバーにも出現しない任意の述語記号$\orp{n,i}$について,\\\hfill
    $(m)(m\in\arg\epsilon\case{1}{1}{1}(\mathcal{R}\fap m)\fap\orp{n,i}=\Lambda)$.
\end{enumerate}
\ref{sssec:全体論的概念図式}の$g$を構成する仕方によって,(1)は$\breve{\epsilon}\fap\Lambda$が現実を反映したモデルであることを,(2)は$g$に出現しない述語記号は$\breve{\epsilon}\fap\Lambda$の現実性と関係がなく比較類似性の基盤とならないこと,を表現している.

$\epsilon$が$g$に適合していると言えるための他の条件として,$\epsilon$が与える開始点への類似性の尺度は,$g$(がそれの図式化である$\mathcal{T}^\epsilon$)の各要素が改訂圧力を受ける場合(それを含む文の集合が含意する予測が観察により反駁される場合)におけるその耐性値に相関している.

すなわち,実験状況を要約するような一般化$p\in\mathcal{T}^\epsilon$が観察により反駁されたと仮定した場合,整合性を回復するためには,$p$を含意する最小単位$b\subseteq \mathcal{T}^\epsilon$(どの$b'\subset b$も$p$を含意しない)のメンバーのいずれかを偽とみなす必要が生じる.
この場合,体系内の文の相互連間の中で$b$の各メンバーが占める位置が,どれを偽として捨てるかの選択を制御する.体系内の位置の相違によって,ある種類の文は他の種類の文よりも改訂され難い.そのような相違が改訂圧力への耐性の相違である.例えば,論理法則や数学的法則は改訂に対してほとんど免疫がある\footnote{
    論理法則や数学的法則を含めたどの文も,他の文と連合して何らかの予測を含意することに関与するという意味で,経験的内容を分け持っている.原理的には論理法則や数学的法則の改訂もあり得る.クワイン~\cite[pp.\,2--4]{クワインb}を参照.
}.基本的な物理法則の耐性値も非常に高いが,それが改訂されることはあり得る.さらに,物理法則の外側に,体系の周縁に向かって,他の自然科学的法則から経済学的一般化などの中間を経て,常識的な事実的一般化に至る配列が来る.この順番で改訂圧力への耐性値は下がっていく.

そして,$\epsilon$が与える類似性尺度が改訂圧力への耐性値に相関するというのは,他の条件が同じであれば,改訂圧力への耐性値がより高い文(の図式化)がより多く偽となるようなモデルほど,開始点への類似性の度合いが低い,ということを意味している.言い換えれば,ある文が真理集合が表現する包括的体系にとって基本的なものであればあるほど(中心に近ければ近いほど),その図式化が偽となるモデルは開始点への類似性が低くなる.論理法則と数学的法則は体系の中心であり極限的なケースである.すなわち,論理法則は定義上すべてのモデルにおける真理であり,これが偽となる$m\in\arg\epsilon$はない.数学的法則もこれに準じて,解釈空間の定義により,すべての$m\in\arg\epsilon$において真である.

 % 因果
% !TeX root = foundation.tex

\section{規制}
\label{sec:規制}

規制類型$y$が生成する,1個の適用条件と2個の因果的構造,すなわち①$y$の執行可能性と②$y$の制御可能性が,準拠領域$x$について成立しているとき,順序対$\orp{x,y}$は規制である.
\ref{ssec:規制類型}〜\ref{ssec:規制の概念}でこのような規制の構造の細部を展開する.
そして,\ref{ssec:正規集合と規制体系}〜\ref{ssec:規制ルール}では,ある種の構造を持った規制集合を構成して,それが言語的統制と規制システムの工学的妥当性において果たす役割について述べる.
最後に\ref{ssec:存在論}では,規制類型の階層的構造とそれに連動する存在論的前提を明らかにする.

\subsection{規制類型}
\label{ssec:規制類型}

予備的概念として,有限系列$\alpha$の反転$ \tilde{\alpha} $を,
\begin{df}
\label{df:系列の反転}
\kagi{$\tilde{\alpha}$}あるいは\kagi{$\tildel{\alpha}$}は,\kagi{$
    \lambda_x[\alpha\fap(\iter{\breve{\mathrm{S}}}{(\mathrm{S}\fap x)}\fap\arg\alpha)]\uphr \arg\alpha
$}を表わす
\end{df}
\noindent と定義する.$\tilde{\alpha}$は$\alpha$を逆順に並び替えた系列であり,$\tilde{\alpha}\fap n$は$ \alpha $の最後から$n$番目の要素を指示する.
次に,クラス$\alpha$のメンバーである,有限系列とは限らない系列の$\beta$番目の要素のクラス,及びその系列の最後から$\beta$番目の要素のクラスを導入する.

\begin{df}
\label{df:メンバーの系列要素}
\kagi{$
    \msec{\alpha}{\beta}
$}は\kagi{$
    \classab{x\fap\beta:x\in\alpha\cap\mathrm{Seq}\con{1}\beta\in\arg x}
$}を表わす,
\end{df}

\begin{df}
\label{df:メンバーの反転系列要素}
\kagi{$
    \mser{\alpha}{\beta}
$}は\kagi{$
    \msec{(\lambda_x\tilde{x}\img\alpha)}{\beta}
$}を表わす.
\end{df}

さて,規制類型$v$は,独立変項である$3$項以上の系列に対して同じ長さの系列を共通に与える関数である.そして,値となる唯一の系列$\trgl{v}$の最後から2番目の要素$ \tildel{\trgl{v}}\fap 1 $は,空集合であるか,$ \trgl{v}\fap 0 $またはメンバーを制限されたその補クラスと同一である.すなわち,

\begin{df}
\label{df:規制類型}
\kagi{$
    \mathrm{Reg}
$}は\kagi{$
    \classab{v:\trgl{v}\neq\Lambda\con{1}
    (x)(x\in \arg v\case{1}{1}{1}x,\trgl{v}\in\mathrm{Seq}\con{1}2\in \arg x = \arg \trgl{v})\con{1}
    \\\hfill
    \tildel{\trgl{v}}\fap 1 \neq \Lambda\case{1}{1}{2}\tildel{\trgl{v}}\fap 1 = \trgl{v}\fap 0 
    \case{2}{1}{1}
    \tildel{\trgl{v}}\fap 1 = \barl{(\trgl{v}\fap 0)}\cap\msec{(\arg v)}{0}
    }
$}を表わす.
\end{df}
\noindent $v\in\mathrm{Reg}$であるとき,$ \trgl{v}\neq\Lambda $ゆえに,$(\exists y)(v\img\univ=\classab{y})$.したがって,$\func v\con{1}v\neq\Lambda$.

$ \tildel{\trgl{v}}\fap 1 $は$v$の制御方向と呼ばれる.制御方向の違いによって規制類型は以下の3タイプに分けられ,それに応じて制御構造と制御可能性の内容に違いが生じる.
\begin{enumerate}[label=(\arabic*)]
    \item $ \tildel{\trgl{v}}\fap 1 = \trgl{v}\fap 0 $のとき,$v$はP類型,
    \item $ \tildel{\trgl{v}}\fap 1 = \barl{(\trgl{v}\fap 0)}\cap\msec{(\arg v)}{0} $のとき,$v$はS類型,
    \item $ \tildel{\trgl{v}}\fap 1 = \Lambda $のとき,$v$はN類型.
\end{enumerate}
なお,(2)で$ \barl{(\trgl{v}\fap 0)} $ではなく,それと$\msec{(\arg v)}{0}$との合併を用いている理由は,\ref{ssec:集合論の体系}の体系においては,任意の$x$について$\bar{x}\notin\univ$であることによる.すなわち,仮に$ \bar{x}\in\univ $なら,$ x\cup\bar{x} = \univ \in\univ $だから,A \ref{axim:分出}により,
\[
    \classab{x:x\notin x}\cap\univ = \classab{x:x\notin x}\in\univ.
\]
したがって,ある$y$が存在して,$ y = \classab{x:x\notin x} $.定義により,$(z)(z\in y \case{3}{1}{1}z\notin z)$.\kagi{$ y $}で例化すると,
\begin{align*}
    y\in y \case{3}{1}{1}y\notin y
\end{align*}
であり,矛盾が生じる.したがって,$\bar{x}\notin\univ$.

他方,$ \msec{z}{i}\in\univ $であることは次のようにして分かる.$x\in z\con{1}i\in\arg x$なる系列$x$が存在しない場合,$\msec{z}{i}=\Lambda$だから,T \ref{thm:単一クラス}による.他方,そのような系列$x$が存在する場合,$ \msec{z}{i}\subseteq \union{(\union{z})} $だから,A \ref{axim:一対化,和,冪}とT \ref{thm:部分クラス}による.

次に,$\delta\in\mathrm{Reg}$から一義的に生成される構造の1つ,$\delta$の適用条件$ \app{\epsilon}\delta $を構成するために,有限系列に関する予備的概念をさらに追加する.
まず,有限系列に対してその最終要素を落とした系列を与える関数を導入する.
\begin{df}
\label{df:系列の最終要素を除外}
\kagi{$
    \mathrm{E}
$}は\kagi{$
    \lambda_x(x\uphr(\breve{\mathrm{S}}\fap(\arg x)))\uphr\mathrm{Seq}
$}を表わす.
\end{df}

\noindent そして次の定義は,$\alpha,\beta$が同じ長さの有限系列であるとき,$ \alpha = \mathcal{L}\resl z\con{1}\beta = \mathcal{R}\resl z $であるような合成系列$ z $を指示する記法を導入する.
\begin{df}
\label{df:合成系列}
    \kagi{$
        \alpha\diamond\beta
    $}は\kagi{$
        \lambda_n\orp{\alpha\fap n,\beta\fap n}\uphr\arg\alpha
    $}を表わす.
\end{df}

\noindent さらに,系列の先行者が後続者に対して因果関係を持つような有限系列を「因果系列」と言い,そのクラスを
\begin{df}
\label{df:因果系列}
\kagi{$
    \mathrm{Cseq}^\epsilon
$}は\kagi{$
    \classab{z:z\in\mathrm{Seq}\con{1}z\img\univ\subseteq\dotl{\univ}\con{1}
    z\resl\mathrm{S}\resl\breve{z}\subseteq \mathcal{C}^{\epsilon}
    }
$}を表わす
\end{df}
\noindent と規定する.因果関係は因果的依存関係の祖先関係であるから,
\[
    \orp{e,e'}\in\mathcal{C}^\epsilon\case{1}{1}{0}
    (\exists z)(
        z\in \mathrm{Cseq}^\epsilon\con{1}z\fap 0 = e\con{1}\tilde{z}\fap 0 = e'
    ).
\]
他方,$z$が順序対の1項系列である場合,$ z\resl\mathrm{S}\resl\breve{z}=\Lambda $であるから,空虚に$ z\in\mathrm{Cseq}^\epsilon $である.
また,$z\in\mathrm{Cseq}^\epsilon$について,$z\img\univ\subseteq\mathfrak{E}$であるとき,因果系列$z$が「実現している」と言う.$z\fap 0\in\mathfrak{E}$であるなら,因果の基本法則によって,すべての$i\in\arg z$について$z\fap i\in\mathfrak{E}$.

$\app{\epsilon}\delta$は,$x\in \arg\delta$と$ \trgl{\delta} $の合成系列の最後の2個を落とした系列が因果系列となる$ x $のクラスである.
\begin{df}
\label{df:適用条件}
\kagi{$
    \app{\epsilon} \delta
$}は\kagi{$
    \arg\delta\cap\classab{x:\iter{\mathrm{E}}{2}\fap(x\diamond\trgl{\delta})\in\mathrm{Cseq}^\epsilon}
$}を表わす.
\end{df}
\noindent $ \orp{x,y}\in\app{\epsilon} y\times\mathrm{Reg} $であるような$ \orp{x,y} $を「規制要素」と言う.
$ x\in\app{\epsilon}y\subseteq\arg y\subseteq z $について,$z$は後述する修正領域$\tilde{x}\fap 0$に関する条件(他の系列要素との関係的な条件を含む)であり得る\footnote{
    例えば,高次類型である時効援用権の適用条件には,修正領域の規制類型に応じた時効期間の経過という条件が含まれる.また規制表明型では,規制表明で提示されるルールが修正領域の規制類型に関するルールである,という趣旨の条件が含まれる(\ref{ssec:規制ルール}).
}.このような条件を除外して$ \arg y $から修正領域に関与しない条件だけを抽出するには,次の定義による.

\begin{df}
\label{df:純粋条件}
\kagi{$
\mathrm{pur}\,\delta
$}は\kagi{$
\intersect{
    \classab{z:\arg \delta \subseteq z\con{1}
    (x)[
        x\in\arg \delta
        \case{1}{1}{0}(b)(\tildel{(\tilde{x}\tbinom{0}{b})}\in z)
    ]
    }
}
$}を表わす.
\end{df}

\subsection{執行可能性}
\label{ssec:執行可能性}

$x\in\app{\epsilon}\delta$について,$ \iter{\mathrm{E}}{2}\fap (x\diamond\trgl{\delta})\in\mathrm{Cseq}^\epsilon $が実現しているとき,$x$は$\delta$の構成要件$\exe{\epsilon}\delta$に属する.

\begin{df}
\label{df:構成要件}
\kagi{$
    \exe{\epsilon}\delta
$}は\kagi{$
    \classab{x:x\in\app{\epsilon}\delta\con{1}x\fap 0\in \trgl{\delta}\fap 0}
$}を表わす.
\end{df}

\noindent $x\in\exe{\epsilon}\delta$であることは,ある状況における因果系列の実現を意味するから,通常の用法での行動概念に近い\footnote{
    修理領域に関する条件を除外するとさらに近づく.すなわち,$ x\in\mathrm{pur}\,\delta\cap\classab{x:\iter{\mathrm{E}}{2}\fap(x\diamond\trgl{\delta})\in\mathrm{Cseq}^\epsilon}\con{1}x\fap 0\in\trgl{\delta}\fap 0 $.
}.特に,後述の制御可能性の結節点となる$\tilde{x}\fap 1$が,当該因果系列が帰属される対象として$\arg\delta$によって限定されるケースではそうである.例えば,$ \arg\trgl{\delta}=5 $として,厳密な規定条件ではないが,
\begin{gather*}
    \trgl{\delta}\fap 0 = \classab{e:\text{$e$は刺激過程}},\\
    \trgl{\delta}\fap 1 = \classab{\orp{e,a}:\text{$e$は$a$の空間的位置の変化}},\\
    \trgl{\delta}\fap 2 = \classab{\orp{e,a}:\text{$e$は$a$が死亡する出来事}}
\end{gather*}
と置く.そして,任意の$x\in\arg\delta$について,ある$ e,a,e',a'\subseteq\timex{\mathbb{R}}{4} $が存在して,
\begin{gather*}
    x\fap 1 = \orp{e,a}\con{1}x\fap 2 = \orp{e',a'},\\
    x\fap 0\subseteq x\fap 3\con{1}x\fap 3 = a\neq a' \con{1}a,a'\in\classab{x:x\text{ は人間}}
\end{gather*}
とすると,$ x\in\exe{\epsilon}\delta $は,$ x\fap 0\subseteq x\fap 3 $である刺激過程$ x\fap 0 $によって,$ x\fap 3 $の空間的位置が変化し,それによって,$ x\fap 3\neq a' $の死が惹起される,という因果連鎖である.この点,刺激過程は,発火したニューロン(神経細胞)の時間的部分のクラス$d$について$ \union{d} $と見做せる\footnote{
    時間$t_1$で発火したニューロンが,他のニューロンに対して,それの発火を促す刺激かまたはそれの発火を抑制する刺激を送信する.それらを受信したニューロンにおいて刺激の総和が閾値を超えた場合に,そのニューロンが時間$t_2$で発火する.以下同様にして,これらニューロンの時間的部分のクラス$d$の合併$\union{d}$が1個の刺激過程となる.$d$のメンバーには,外部刺激に反応する感覚細胞や,筋繊維に出力する運動神経細胞も含まれ得る.
}.それゆえ,この因果連鎖は結局,$x\fap 3$の身体運動によって$a'$の死が惹起される因果連鎖である.
これに対して,$ \trgl{\delta}\fap 1 = \barl{\classab{\orp{e,a}:\text{$e$は$a$の空間的位置の変化}}}\cap \msec{(\arg\delta)}{1} $等である場合,身体運動の不在を介した不作為による結果惹起である.なお,ある身体運動を惹起する刺激過程とそれの不在を惹起する刺激過程は同一ではない.

次に,$ \tildel{\trgl{\delta}}\fap 0 $を$\delta$の修正条件と言い,(制御可能性において)制御方向の違いに応じて構成要件の実現を促進するか(促進条件),または,それを抑制する機能を持つ(抑制条件).そして,$ x\in\arg \delta $について,$\tilde{x}\fap 0$はそれについて修正条件が実現すべき対象(修正領域)である.
修正条件の内容は規制類型ごとに異なるが,その一般形式を以下のように考えることができる.
前提概念として,$b\in\beta$であるレベル$m$の蓋然性を持つ$\orp{b,m}$のクラスを定義する.
\begin{df}
\label{df:危険測度関係}
\kagi{$
    \beta^{::\epsilon}
$}は\kagi{$
    \classab{\orp{b,m}:b\in\prob{\beta}{m\uphl\epsilon}\con{1}0\in m\in\epsilon\img\univ}
$}を表わす.
\end{df}
\noindent そして,ある$j\in\arg\delta$について,$\trgl{\delta}\fap j$に条件$\beta$の(様々なレベルにおける)蓋然性を設定する場合,
\[
    \trgl{\delta}\fap j = \beta^{::\epsilon}\cap\mser{(\arg\delta)}{0}
\]
とした上で,蓋然性レベルを$ \arg\delta\subseteq\classab{x:\mathcal{R}\fap(x\fap j)\subseteq\indx{\epsilon}} $等,適用条件で制限する\footnote{
    特定の蓋然性レベル$\mu\subseteq\indx{\epsilon}$に限定して,$ \trgl{\delta}\fap j = \prob{\beta}{(\mu\uphl\epsilon)}\cap\mser{(\arg\delta)}{j}$,とすることも考えられる.しかし,$i\in \mu$について,レベル$i$の蓋然性を惹起しているが,レベル$\mu$の蓋然性は先行条件に依存していないケースでは,$\app{\epsilon}\delta$が阻却されるか,または$\enf{\epsilon}\delta$(D \ref{df:執行可能性})が阻却されてしまうから($\trgl{\delta}\fap j$が修正条件の場合),結局は不都合を生じる.
    % 死ぬ直前の人への殺人未遂の事例で,仮に行動がなくても死の有意危険があるケースを考える.当該行動はレベル$\indx{\epsilon}$の危険を因果的に決定していないが,さらに高められた死の危険を因果的に決定している.
}.
次に,$j = \breve{\mathrm{S}}\fap\arg\delta$であり,したがって,$ \trgl{\delta}\fap j $が修正条件である場合,上記の\kagi{$ \beta $}に以下の代入を行って,修正条件の一般形式を得る($\delta$が後述の高次類型であるケースを除く).すなわち,
\[
    (\exists n)(n\in\mathbb{N}\con{1}\union{(\sigma\img\univ)}\subseteq\timex{\univ}{n})
\]
であるような$ \sigma\in\mathrm{Seq} $について,$\beta = \mathfrak{E}\resl\sigma$と置く.

例えば,厳密な規定ではないが,$\sigma$の系列要素を刑罰類型として,
\begin{gather*}
    \sigma\fap 2 = \classab{\orp{e,a,n}:\text{$e$は$a$に対する$n$円の罰金}},\\
    \sigma\fap 3 = \classab{\orp{e,a,n}:\text{$e$は$a$に対する$n$月の禁錮}},\\
    \sigma\fap 4 = \classab{\orp{e,a,n}:\text{$e$は$a$に対する$n$月の懲役}},\\
    \sigma\fap 5 = \classab{\orp{e,a,n}:\text{$e$は$a$に対する死刑}\con{1}n=\Lambda}
\end{gather*}
と置く.すると,$ \orp{\orp{e,a,12},4}\in \mathfrak{E}\resl\sigma $は,$e\subseteq\timex{\mathbb{R}}{4}$が$a$に対する1年間の懲役であることを意味する\footnote{$e$は出来事であると想定されているが,いずれにせよ$ \timex{\mathbb{R}}{4} $の部分クラスであるから,人やその他の物理的対象と出来事の区別は存在論的に重要ではない.}.
次に,修正領域に関して,$ \arg\delta $による以下のような制限を考えることができる.すなわち,
$ x\in\arg\delta $について,$ \mathcal{L}\fap(\tilde{x}\fap 0) = \orp{\orp{e,a,n},i} $なる$ e,a,n,i $と,$\mathcal{R}\fap(\tilde{x}\fap 0) = m $が存在して,
\begin{gather*}
    e,a\subseteq\timex{\mathbb{R}}{4}\con{1}a = \tilde{x}\fap 1\con{1}n\subseteq\mathbb{N},\\
    1\in i \in 6\con{1}
    (i = 2\case{1}{1}{1} n\leq 1000000)\con{1}
    (i = 3\case{1}{1}{1}n\leq 60)\con{1}
    (i = 4\case{1}{1}{2}n\leq 240 \case{2}{1}{1}n = \mathbb{N}),\\
    m\subseteq \indx{\epsilon}.
\end{gather*}
このような適用条件によると,$ \tilde{x}\fap 0\in \tildel{\trgl{\delta}}\fap 0 $は,100万円以下の罰金,5年以下の禁錮,20年以下の懲役,無期懲役($ n = \mathbb{N} $),死刑,のいずれかの(レベル$m$の)蓋然性になる.
% 3年執行猶予は3年以内の$e$が懲役等である蓋然性.起訴猶予の認定も刑罰の蓋然性を修正条件に持つ規制の認定.

さて,$\delta$の執行可能性$\enf{\epsilon}\delta$は,単に,$\delta$の構成要件実現によって$\delta$の修正条件が実現される因果的構造である.すなわち,

\begin{df}
\label{df:執行可能性}
\kagi{$
    \enf{\epsilon}\delta
$}は\kagi{$
    \classab{x:
        \orp{x,\exe{\epsilon}\delta}\to_{\epsilon}\orp{\tilde{x}\fap 0,\tildel{\trgl{\delta}}\fap 0}
    }\cap\trgl{\arg\epsilon}
$}を表わす.
\end{df}

\subsection{制御可能性}
\label{ssec:制御可能性}

制御可能性の前提として,$\delta$の制御構造$ \cs{\epsilon}\delta $は,$\app{\epsilon}\delta$の実現によって制御方向$\tildel{\trgl{\delta}}\fap 1$が実現される因果的構造である.

\begin{df}
\label{df:制御構造}
\kagi{$
    \cs{\epsilon}\delta
$}は\kagi{$
    \classab{x:
        \orp{x,\app{\epsilon}\delta}\to_{\epsilon}\orp{x\fap 0,\tildel{\trgl{\delta}}\fap 1}
    }\cap\trgl{\arg\epsilon}
$}を表わす.
\end{df}

\noindent したがって,$ \tildel{\trgl{\delta}}\fap 1 = \trgl{\delta}\fap 0 $の場合($ \delta $がP類型),
\begin{align*}
    x\in\cs{\epsilon}\delta\con{1}x\in\app{\epsilon} \delta & \case{1}{1}{1}(x\diamond\trgl{\delta})\fap 0\in\mathfrak{E}\\
    &\:\, \case{1}{0}{1}(\iter{\mathrm{E}}{2}\fap(x\diamond\trgl{\delta}))\img\univ\subseteq \mathfrak{E}.
\end{align*}
つまり,$ x\in\app{\epsilon}\delta $によって,$ \iter{\mathrm{E}}{2}\fap(x\diamond\trgl{\delta}) $が実現される(促進的または正の制御構造).

他方,$ \tildel{\trgl{\delta}}\fap 1 = \barl{(\trgl{\delta}\fap 0)}\cap\msec{(\arg\delta)}{0} $の場合($ \delta $がS類型),
\begin{align*}
    x\in\cs{\epsilon}\delta\con{1}x\in\app{\epsilon} \delta & \case{1}{1}{1}(x\diamond\trgl{\delta})\fap 0\in\barl{\mathfrak{E}}\\
    &\:\, \case{1}{0}{1}(\iter{\mathrm{E}}{2}\fap(x\diamond\trgl{\delta}))\img\univ\subseteq\barl{\mathfrak{E}}.
\end{align*}
つまり,$ x\in\app{\epsilon}\delta $によって,$ \iter{\mathrm{E}}{2}\fap(x\diamond\trgl{\delta}) $の実現が阻止される(抑制的または負の制御構造)\footnote{
    適用条件の成立が検出されたときに問題の因果系列を起動する条件を遮断するような,システム$ \tilde{x}\fap 1 $内部のメカニズムを想定できる.$ \tilde{x}\fap 1 $に帰属可能な他の行動を起動するような条件を媒介として,この遮断が起きるケースもあると思われるが,必ずしもそうである必要はない.
}.

次に,$\delta$の制御可能性$\cty{\epsilon}\delta$は,$\tilde{x}\fap 1 = \tilde{w}\fap 1$なる$w$が存在して,
\[
   \text{$ x\in\exe{\epsilon}\delta\cap\enf{\epsilon}\delta$であることによって,$w\in\cs{\epsilon}\delta$である蓋然性が因果的に決定される,}
\]
そのような$x$のクラスである.つまり,

\begin{df}
\label{df:制御可能性}
\kagi{$
    \cty{\epsilon}\delta
$}は\kagi{$
    \classab{x:
        (\exists w)(\exists n)(
            w\in\mathrm{Seq}\con{1}\tilde{x}\fap 1 = \tilde{w}\fap 1\con{1}\Lambda\neq n\subseteq\indx{\epsilon}\con{1}
            \\\hfill
                \orp{x,\exe{\epsilon}\delta\cap\enf{\epsilon}\delta}\to_{\epsilon}
                \orp{w,\prob{(\cs{\epsilon}\delta)}{n\uphl\epsilon}}
        )
    }\cap\trgl{\arg\epsilon}
$}を表わす.
\end{df}

\noindent $ \exe{\epsilon}\delta\cap\enf{\epsilon}\delta $を$ \delta $の「執行随伴性」と言う.すると,$x\in\cty{\epsilon}\delta$であることは,執行随伴性によって結節点$\tilde{x}\fap 1$内部の制御回路が変更される,という学習の過程を表現する.
つまり,$\delta$がP類型ならば,$\delta$の執行随伴性によって促進的制御構造(の蓋然性)が構築される(促進的または正の制御可能性).そして,$\tilde{x}\fap 0\in\tildel{\trgl{\delta}}\fap 0$であることは,$\tilde{x}\fap 1$に対して,$\exe{\epsilon}\delta$を促進する機能を持つ(促進条件).これに対して,$\delta$がS類型ならば,$\delta$の執行随伴性によって抑制的制御構造(の蓋然性)が構築される(抑制的または負の制御可能性).そして,$\tilde{x}\fap 0\in\tildel{\trgl{\delta}}\fap 0$であることは,$\tilde{x}\fap 1$に対して,$\exe{\epsilon}\delta$を抑制する機能を持つ(抑制条件).

さらに,構築される制御構造に関連して,
$ \tilde{x}\fap 1 = \tilde{w}\fap 1 $である$w$と,$ \app{\epsilon}\delta\subseteq y_1,y_2 $について,
\[
    \orp{w,y_1}\to_{\epsilon}\orp{w\fap 0,\tildel{\trgl{\delta}}\fap 1}\con{1}\neg(\orp{w,y_2}\to_{\epsilon}\orp{w\fap 0,\tildel{\trgl{\delta}}\fap 1})
\]
であると仮定する.この場合,$y_1$は統制に関与しているが$y_2$はそうではない\footnote{
    執行可能性によって,修正条件は$ \exe{\epsilon}\delta\subseteq\app{\epsilon}\delta $に依存しているから,原理的には任意の$ \app{\epsilon}\delta\subseteq y $が統制に関与し得る.したがって,主観的条件等の統制に関与し得ない条件は適用条件から除外される.}.
そして,この情報だけでは,$ w\in\cs{\epsilon}\delta $であるとも$ w\notin\cs{\epsilon}\delta $であるとも断定できない.
しかし,$ y_1 $の内容によっては,$ w\in \prob{(\cs{\epsilon}\delta)}{\indx{\epsilon}\uphl\epsilon} $であることは肯定できる.
すなわち,制御構造は結節点であるシステムの外部の環境的条件とその内部の構造的条件との複合的な条件であり,制御構造の蓋然性はそのような複合的条件の不完全な実現であり得る\footnote{
    「不完全な実現」の内容は固定的ではない.一般に,ある$n\in\indx{\epsilon}$について,$ z = \prob{\alpha}{(n\uphl\epsilon)} $である場合,$\alpha\subseteq z$とは限らない.$\alpha\nsubseteq z$であるが$(\exists s)(z\cap s\subseteq\alpha)$のような$z$かもしれないし,$ z\cap\alpha = \Lambda $であるかもしれない.
}.
したがって,ある$n\subseteq\indx{\epsilon}$について,
\begin{align*}
    \gamma & = \classab{x:\orp{x,y_1}\to_{\epsilon}\orp{\tilde{x}\fap 0,\tildel{\trgl{\delta}}\fap 1}}\cap\trgl{\arg\epsilon}\\ & = \prob{(\cs{\epsilon}\delta)}{n\uphl\epsilon}
\end{align*}
とみなすことができるケースがある\footnote{
    有意レベルの蓋然性を認めるためには,$ \app{\epsilon}\delta $の部分条件のうち,どれが$y_1$において保存されていなければならないかという問題がある.すなわち,どの$ z\in\classab{z:\app{\epsilon}\delta\subseteq z} $について,$ y_1\subseteq z $であるかという問題.この点,少なくとも,結節点への行動帰属の条件や,因果系列の構成についての条件$ \classab{x:\iter{\mathrm{E}}{2}\fap(x\diamond\trgl{\delta})\in\mathrm{Cseq}^\epsilon} $が保存されている必要があると思われる.
}.するとこの場合,
\begin{align*}
    \orp{x,\exe{\epsilon}\delta\cap\enf{\epsilon}\delta}\to_{\epsilon}\orp{w,\gamma}
\end{align*}
であれば,$ x\in\cty{\epsilon}\delta $.さらにこのケースで,$\trgl{\delta}=\trgl{v}\con{1}\app{\epsilon}v = y_1$なる$v\in\mathrm{Reg}$を考えると,
\[
    \gamma = \cs{\epsilon}v.
\]
また通常の場合,
\[
    x\in \exe{\epsilon}\delta\cap\enf{\epsilon}\delta\case{1}{1}{1}x\in \exe{\epsilon}v\cap\enf{\epsilon}v.
\]
すると,$ x\in \exe{\epsilon}\delta\cap\enf{\epsilon}\delta $である場合,それと同期している$ v $の執行随伴性によって,$ \cs{\epsilon}v $が構築される,とも言うことができる.

ところで,制御可能性の概念の明確さと,それを立証したり反証する容易さは当然別のことである.
制御可能性の認定プロセスは構成要件該当性のそれと共に,執行可能性が何らかの認知システムにより因果的に媒介されるケースにおいて,その因果連鎖の中に埋め込まれる.
そこにおいて,制御可能性を直接的に立証したり反証することは,構成要件該当性のそれに比べて困難を伴う.当該執行随伴性によって新たに将来の制御構造が構築される(有意なレベルの)蓋然性について,そこに関与している多くの変数を十分に特定化することはできない.仮に関連しそうな全ての情報を取得したとしても,そこから特定の制御構造の蓋然性レベルを導出できるような理論はそもそも存在しない.

そこで,制御可能性の立証及び反証は,通常(日常的にも制度的にも)他の条件の立証及び反証で代替される.すなわち,(抑制条件のデメリットとの関係で制御可能性の認定が特に重視される)S規制について言えば,次の形式の文を制御可能性の認定において前提として使用する証明規則(\ref{sssec:認定システムによる正規性論証}を参照)を考えることができる.
\[
    (x)(y)(\tildel{\trgl{y}}\fap 1 = \barl{(\trgl{y}\fap 0)}\cap\msec{(\arg y)}{0}\case{1}{1}{1}Gxy\case{3}{0}{1}x\in \cty{\epsilon}y).
\]
\kagi{$ Gxy $}は「$ x\in\exe{\epsilon} y $が有責に実現された」に相当する文を表わす文型である.これはS規制の制御可能性を表わす媒介的な概念として,規制類型ごとに構築される.

この点について,\ref{ssec:執行可能性}における規制類型$\delta$の例を再び用いよう.すなわち,$ x\in\exe{\epsilon}\delta $は人の死の惹起であり,\ref{ssec:執行可能性}で示した刑罰類型の系列$\mathfrak{E}\resl\sigma$について,$ \tildel{\trgl{\delta}}\fap 0 \subseteq\classab{\orp{b,m}:b\in\prob{(\mathfrak{E}\resl\sigma)}{m\uphl\epsilon}} $である.また,
\[
   \mathrm{rec}\,\delta = \classab{x:
        \text{$ \tilde{x}\fap 1 $が$ x\fap 0 $の時点で$ x\in\exe{\epsilon}\delta $であることを認知}
   }\cap\app{\epsilon}\delta
\]
と置く.そして,$ k\in\indx{\epsilon} $を有意レベルを超える高度の蓋然性レベルと決める.すると,$ \mathcal{L}\fap(\tilde{x}\fap 0) = \orp{\orp{e,a,n},i} $について,
\begin{gather*}
    x\in \mathrm{rec}\,\delta \con{1} \neg(i = 4\con{1}60\leq n\case{2}{1}{1}i=5) \case{1}{1}{0}\neg Gxy,\\
    x\notin \mathrm{rec}\,\delta \con{1} x\in \prob{(\mathrm{rec}\,\delta)}{k\uphl\epsilon}\con{1}
        \neg( i \in \classab{3,4}\con{1}n\leq 60 \case{2}{1}{1} i = 2\con{1}500000 < n\leq 1000000)\case{1}{1}{0}\neg Gxy,\\
    x\notin \prob{(\mathrm{rec}\,\delta)}{k\uphl\epsilon}\con{1}x\in\prob{(\mathrm{rec}\,\delta)}{(\indx{\epsilon}\uphl\epsilon)}\con{1}\neg(i = 2\con{1}n\leq 500000) \case{1}{1}{0}\neg Gxy.
\end{gather*}
$ x\in\mathrm{rec}\,\delta $は「故意」と言われる状態であり,$ x\in \prob{(\mathrm{rec}\,\delta)}{k\uphl\epsilon} $は高度の過失(予見可能性),$ x\in\prob{(\mathrm{rec}\,\delta)}{(\indx{\epsilon}\uphl\epsilon)} $は通常の過失である.
上記の定式化は,故意・過失の態様に応じて量刑を分割するよく見られる刑法制度の例であるが,実在の制度の記述においては,例外収容条件等により定式化はもっと複雑なものになり得る.この他,不随意的反応によって起動された因果連鎖については,通常の制度において,任意の修正条件との関係で有責性が阻却される\footnote{関連して,複数の認定手続を同時に処理する場合に,そのような手続の集合全体との関係でその要素における量刑が制限される,という法制度が存在し得る.これには,再犯のため前の手続の量刑が足りなかった場合に,その不足を補う手続と現在の手続とを同時に処理するようなケースも含まれる(このように制度を描写することは二重の危険の禁止を持つ規制体系においては問題視されるかもしれない).いずれにせよ,集合の要素の全てに同一タイプの刑罰を課す場合に,その量の合計が特定の値(可能な最も重い刑の値の$1\leq n$倍等)以下であることを要求する制度を想定できる.% 包括一罪・科刑上一罪(牽連犯・観念的競合)なら$n = 1$.併合罪なら$n=2$.
}

\subsection{規制の概念}
\label{ssec:規制の概念}

執行可能性と制御可能性を基にして,規制の概念を構成する.それは規制関係(規制のクラス)の規定条件を構成することを意味する.すなわち,
\begin{df}
\label{df:規制関係}
\kagi{$
    \mathrm{REG}^\epsilon
$}は\kagi{$
    \classab{\orp{x,v}:
        x\in\app{\epsilon} v\cap\enf{\epsilon}v\con{1}
        (\tildel{\trgl{v}}\fap 1\neq\Lambda\case{1}{1}{1}x\in\cty{\epsilon}v)\con{1}
        v\in\mathrm{Reg}
    }
$}を表わす.
\end{df}

\noindent $ \orp{x,v}\in\mathrm{REG}^\epsilon $であるとき,$\orp{x,v}$は1個の規制である.それは適用条件$\app{\epsilon} v$と執行可能性$ \enf{\epsilon}v $を充たす準拠領域$x$と規制類型$v$の順序対である.そして,制御方向$ \tildel{\trgl{v}}\fap 1 $が空でない限り,$ x $は制御可能性$ \cty{\epsilon}v $をも充たしている必要がある.
$v$がP類型なら$\orp{x,v}$はP規制,S類型ならS規制,N類型ならN規制と言われる.
N規制の制御方向は$ \Lambda $であり,そこでは制御構造も制御可能性も意味を持たない.N規制の内容は単に適用条件と執行可能性であり,P規制やS規制のような直接的な規制的機能を持たない.しかし,N規制は,後述の高次規制としてP規制またはS規制の生成消滅に寄与するか(\ref{ssec:正規集合と規制体系}),または,言語的統制を媒介することによって(\ref{sssec:言語的統制における正規性論証}),間接的な規制的機能を持つようになる.

ところで,P類型の制御構造の適用条件が実現すれば,そこで構成要件が実現されることになる.すなわち,
\[
    \tildel{\trgl{y}}\fap 1 = \trgl{y}\fap 0\con{1}x\in \cs{\epsilon}y\con{1}x\in\app{\epsilon} y \case{1}{1}{1}x\in\exe{\epsilon} y.
\]
他方,S類型の場合,同じ状況で構成要件が実現されるのではなく,それの実現が阻止される.つまり,
\[
    \tildel{\trgl{y}}\fap 1 = \barl{(\trgl{y}\fap 0)}\cap\msec{(\arg y)}{0}\con{1}x\in \cs{\epsilon}y\con{1}x\in\app{\epsilon} y \case{1}{1}{1}x\notin\exe{\epsilon} y.
\]
しかし,このS類型のケースで,$ x\in \cs{\epsilon}y $であることが,執行随伴性(またはその言語的代替)によって構築されている場合には,通常,$\orp{x,y}$に対して遮断関係を持つ$ \orp{z,w}\in \app{\epsilon} w\times\mathrm{Reg} $が存在して,
\[
    \orp{x,\app{\epsilon} y}\to_{\epsilon}\orp{z\fap 0,\trgl{w}\fap 0}\con{1}
    \orp{z\fap 0,\trgl{w}\fap 0}\to_{\epsilon}\orp{x\fap 0,\tildel{\trgl{y}}\fap 1}.
\]
あるいは,同一の執行随伴性(またはその言語的代替)によって,$ z\in\cs{\epsilon}w $が構築されるとも考えられる.
遮断関係$\mathrm{IS}$は,一方の因果系列の実現が他方のそれを論理的に阻止するような関係である.すなわち,

\begin{df}
\label{df:遮断関係}
\kagi{$
    \mathrm{IS}
$}は\kagi{$
    \classab{\orp{\orp{z,w},\orp{x,y}}:z\in\app{\epsilon} w\con{1}x\in\app{\epsilon} y\con{1}z\uphr\bar{1} = x\uphr\bar{1}\con{1}w,y\in\mathrm{Reg}\con{1}
    \\\hfill
    \arg w = \arg y\con{1}
    (\iter{\mathrm{E}}{2}\fap\trgl{w})\uphr\bar{1}\subseteq\lambda_n(\barl{(\trgl{y}\fap n)}\cap\msec{(\arg y)}{n})\con{1}
    \trgl{w}\fap 0 = \trgl{y}\fap 0
}
$}を表わす.
\end{df}

\noindent $ \barl{(\bar{x}\cap\alpha)}\cap\alpha = x $であるから,遮断関係は対称的である.つまり,$ \mathrm{IS}=\brevel{\mathrm{IS}} $.

再びS類型$y$について,$ x\in\cs{\epsilon}y $が構築されるケースを考える.すなわち,$ \tilde{e}\fap 1 = \tilde{x}\fap 1 $について,
\[
    e\in\app{\epsilon} y\cap\enf{\epsilon}y\con{1}
    \orp{e,\app{\epsilon} y\cap\enf{\epsilon}y}\to_{\epsilon}\orp{x,\cs{\epsilon}y}.
\]
この場合,おそらく,ある$ \orp{z,w}\in\mathrm{IS}\img\classab{\orp{x,y}} $が存在して,
\[
    \orp{e\in\app{\epsilon} y\cap\enf{\epsilon}y}\to_{\epsilon}\orp{z,\cs{\epsilon}w}.
\]
ただし,$ \tildel{\trgl{w}}\fap 1 = \trgl{w}\fap 0 = \trgl{y}\fap 0 $.
そして,因果系列の構成条件が異なるであろうから,$ \app{\epsilon} y \neq \app{\epsilon} w $であるとしても,事実上,
$ \orp{x,\app{\epsilon} y}\to_{\epsilon}\orp{z,\app{\epsilon} w} $であり得る.また,$ (z\diamond\trgl{w})\fap 0 $と$ (z\diamond\trgl{w})\fap 1 $との間に,
\[
   \orp{z\fap 0,\trgl{w}\fap 0}\to_{\epsilon}\orp{x\fap 0,\tildel{\trgl{y}}\fap 1}\con{1}
    \orp{x\fap 0,\tildel{\trgl{y}}\fap 1}\to_{\epsilon}\orp{z\fap 1,\trgl{w}\fap 1}
\]
という因果連鎖が存在すると考えられる.
さらに,あるレベル$ i\in\indx{\epsilon} $についてならば,$ \prob{(\cs{\epsilon}y)}{i\uphl\epsilon} = \prob{(\cs{\epsilon}w)}{i\uphl\epsilon} $と言えるケースもあり得る.

なお関連して,$ \orp{\orp{z,w},\orp{x,y}}\in\mathrm{IS} $であるとき,$ \orp{x,y} $がS規制ならば,$ x\in\exe{\epsilon} y $は「禁止されている」,$ z\in\exe{\epsilon} w $は「義務づけられている」と言えるかもしれない.
実際このケースで,
\[
    \tildel{\trgl{w}}\fap 1 = \trgl{w}\fap 0\con{1}
    \tildel{\trgl{w}}\fap 0 = \barl{(\tildel{\trgl{y}}\fap 0)}\cap\mser{(\arg y)}{0}
\]
であれば,$ \orp{z,w} $は$ \orp{x,y} $の抑制条件の阻止を促進条件とするP規制であり得る\footnote{
    ただし,これらは機能的に同一であろうから,同一の規制体系に属するなら,冗長性により工学的妥当性に問題を生じる.
}.もっとも,S規制が禁止の概念分析ではないように,このようなP規制も義務の概念分析ではない.

結局,すべての規制は制御方向の違いに応じて3タイプに分けられる.どのタイプであれ準拠領域$x$に対して規制を構築することは,$ \tildel{\trgl{v}}\fap 1 \neq\Lambda \case{1}{1}{1}x\in\cty{\epsilon}v $であるような規制類型$v$について,$ x\in\enf{\epsilon}v $を構築することを意味している\footnote{
    制御可能性を構築することは価値または反価値を創出することであり,規制を構築するのとは異なるプロジェクトである.この点については,
    $ x\in\cty{\epsilon}y \con{1} \orp{\orp{a,b},\orp{\tilde{x}\fap 0,\tildel{\trgl{y}}\fap 0}}\in \mathfrak{E}\uphl\mathcal{C}^\epsilon $であることによって,以下の条件を充たす$ w\in\mathrm{Seq} $と$ z\in\mathrm{Reg} $について,$ w\in \cty{\epsilon}z $が因果的に決定される場合がある.
    \[
        \tilde{x}\fap 1 = \tilde{w}\fap 1\con{1}\mathrm{pur}\,y = \mathrm{pur}\,z\con{1}\tildel{(\tildel{\trgl{y}}\tbinom{0}{b})} = \trgl{z}.
    \]}.
この点,高次類型(\ref{ssec:正規集合と規制体系})ではない規制類型の場合,その執行可能性における構成要件実現と修正条件の実現との間には,通常,
\begin{enumerate}
    \item[①] 検出システムが構成要件実現による環境変化を証拠として収集する,
    \item[②] 認定システムが構成要件該当性と制御可能性(と正規性)を認定する,
    \item[③] 執行システムが修正条件を導入する
\end{enumerate}
という制御構造から成る因果系列が存在する.一方の極である法規制の場合,①〜③の担当システムは通常ある程度独立した異なる機関である.他方の極である個人の内部規制や個人間の規制では,担当システムは同一の人間である.そして,これらの中間的なバージョンが存在する.いずれにせよ①〜③の制御構造もまた執行随伴性またはその言語的代替によって構築される.執行可能性を構築することは,そのような全体的システムを設計・構築することでもある.

さらに執行可能性自体について言えば,\ref{sssec:全体的言語}の全体的言語$\mathfrak{L}$に適合する解釈空間に依拠する限り,因果関係は,それゆえに執行可能性は,最終的に何らかの物理的構造に帰着すると思われる.この意味で規制システムは物理的システムであり,特に法規制はその執行可能性が人工的に創出される工学的な産物である.だとすると,他の工学的産物と同様に,物理的システムとしての規制を記述することと,それの工学的妥当性を検証することは,異なる問題に属する.前者の重要部分は規制類型を言及する言語$\mathfrak{L}$の開放文,つまり,$ (\breve{\epsilon}\fap\Lambda)\exten p = y\in\mathrm{Reg} $である論理式$p\in\mathfrak{L}$を構成する作業であり,これを規制の構造解析と呼ぶ.
例えば,\ref{ssec:執行可能性}においては,「$x$は人間」とか「$e$は$a$に対する$n$月の懲役」といった原初的言語に還元できない述語を使用して,規制類型を部分的に規定した.このような構造解析の例は完全な形ではないが,例証のために今後も必要に応じて追加されるだろう.

構造解析を体系的に行う場合,日常言語の不明瞭な用語を積み重ねて作られた混乱した概念群を明晰化せざるを得ないことが多い.日常言語は定式化ではなくコミュニケーションのために進化してきた構造であり,定式化に無自覚な分野では容易に混乱が生じる.例として法人への行動帰属の問題を考察してみよう.
\ref{ssec:執行可能性}では,$\delta\in\mathrm{Reg}$と$ x\in \app{\epsilon}\delta $について,因果連鎖を起動する刺激過程$ x\fap 0 $が,人間$\tilde{x}\fap 1$の部分クラスであることに基づいて,当該因果連鎖が$\tilde{x}\fap 1$に帰属されていた.これに対して,$ \tilde{x}\fap 1 $が法人の場合,人間$d$が存在して,下記の(1)(2)を充たす場合に,$x\fap 0$が起動する因果連鎖が当該法人に帰属される,と一応言ってみることができる.
\begin{enumerate}[label=(\arabic*)]
    \item $d$は$ \tilde{x}\fap 1 $の業務執行権を持つ.$x\fap 0$は$ x\fap 0\subseteq d $なる刺激過程である.
    \item ある$y\in\mathrm{Reg}$と$z\in\app{\epsilon} y$が存在して,
    \[
        \iter{\mathrm{E}}{2}\fap(x\diamond\trgl{\delta}) = \iter{\mathrm{E}}{2}\fap(z\diamond\trgl{y})
        \con{1}\tilde{z}\fap 1 = d
        \con{1}\text{$z\in\exe{\epsilon} y$は$ \tilde{x}\fap 1 $の業務執行}.
    \]
\end{enumerate}
上記の「業務執行権」と「業務執行」は当然さらなる明確化を必要とする.前者は,$ a\in\exe{\epsilon} b $が業務執行であるP規制$ \orp{a,b} $であるか,または,$ a\in\exe{\epsilon}b $が業務執行であるような$ \mathrm{IS}\img\classab{\orp{a,b}} $のメンバーであるS規制,として定義することが期待できる.
問題は後者であるが,法人の行動を規定するために業務執行の概念を使うなら,業務執行の概念を規定する際には法人の行動の概念,それゆえ法人を結節点に持つ規制の概念は使えない.この点については,
占有等の事実状態としての設立時出資財産について,その不適切な管理運用に対するS規制は,設立者集合によって制定される規制として,法人の行動の概念を使用せずに規定できると思われる\footnote{
    機関の概念も使用しない.逆にそのような規制の結節点が設立時の機関である.
}.すると,そのようなS規制$ \orp{c,f} $と$ \orp{c,f}\in\mathrm{IS}\img\classab{\orp{a,b}} $について,$ a\in\exe{\epsilon}b $(設立時出資財産の適切な管理運用)は当該法人の業務執行であり,また,業務執行の結果としてその執行者が占有取得した財産(設立後出資財産も含まれる)の適切な管理運用も業務執行である,というように再帰的に規定すれば,かろうじて循環を回避できるかもしれない.

また,法人の概念自体についても,それは法人の構成員や機関の概念からは独立に規定する必要がある.なぜなら,法人の構成員や機関は,当該法人との契約等によって,当該法人を準拠領域の内部構造に持つ規制の当事者となるからである.
そこで,法人を,設立者集合$h$,設立時名称$n$,設立時の主事務所の所在地$s$,等々の順序対または系列とみなす方法が考えられる.これによると,法人自体は単なる符号として生成消滅することはなく,生成消滅したり存続期間を云々できるのは,法人を内部構造に持つ規制群であることになる.
  % 規制
% !TeX root = foundation.tex

\subsection{正規集合と規制体系}
\label{ssec:正規集合と規制体系}

\subsubsection{高次類型}
\label{sssec:高次類型}

他の規制類型に係る執行可能性の実現(の蓋然性)を修正条件に持つような規制類型を「制定類型」,逆に執行可能性の阻止(の蓋然性)を修正条件に持つ規制類型を「廃止類型」,両者を併せて「高次類型」と言う.
$ v $が高次類型であるとき,任意の$ u\in\arg v $について,
\[
    \tilde{u}\fap 0 = \orp{\orp{z,y},i}\con{1}z\subseteq\app{\epsilon}y\con{1}y\in\mathrm{Reg}\con{1}0\in i\subseteq\indx{\epsilon}
\]
なる$ z,y,i $が存在する.この事実を表現する省略記法として,
\begin{df}
\label{df:高次類型}
\kagi{$
    \mathrm{Ho}^{\epsilon}
$}は\kagi{$
    \mathrm{Reg}\cap\classab{v:
        \mser{(\arg v)}{0}\subseteq \classab{\orp{\orp{z,y},i}:z\subseteq \app{\epsilon}y\con{1}y\in\mathrm{Reg}\con{1}0\in i\subseteq\indx{\epsilon}}
    }
$}を表わす.
\end{df}
\noindent 次に,$v\in\mathrm{Ho}^{\epsilon}$が制定類型であり,かつ,$ \tilde{u}\fap 0\in \tildel{\trgl{v}}\fap 0 $であるとき,$z\subseteq\enf{\epsilon}y$であるレベル$i$の蓋然性が成立する.
他方,$v$が廃止類型ならば,$z\subseteq\barl{(\enf{\epsilon}y)}$であるレベル$i$の蓋然性が成立する.
したがって,以下の定義により,制定類型と廃止類型それぞれのクラスが導入される\footnote{
    $ \mser{(\arg v)}{0} $への制限は$ \tildel{\trgl{v}}\fap 0 $の存在を確保するためである.
}.

\begin{df}
\label{df:制定類型}
\kagi{$
    \mathrm{Es}^\epsilon
$}は\kagi{$
    \mathrm{Ho}^{\epsilon}\cap
    \classab{v:
        \tildel{\trgl{v}}\fap 0 = 
        (\trgl{\arg\epsilon}\cap\classab{\orp{z,y}:z\subseteq\enf{\epsilon}y})^{::\epsilon}\cap\mser{(\arg v)}{0}
    }
$}を表わす,
\end{df}

\begin{df}
\label{df:廃止類型}
\kagi{$
    \mathrm{Ab}^\epsilon
$}は\kagi{$
    \mathrm{Ho}^{\epsilon}\cap
    \classab{v:
        \tildel{\trgl{v}}\fap 0 = 
        (\trgl{\arg\epsilon}\cap\classab{\orp{z,y}:z\subseteq\barl{(\enf{\epsilon}y)}})^{::\epsilon}\cap\mser{(\arg v)}{0}
    }
$}を表わす.
\end{df}

$\orp{u,v}\in\app{\epsilon}v\times\mathrm{Es}^{\epsilon}$について,$ z = \iter{\mathcal{L}}{2}\fap(\tilde{u}\fap 0)$は$\orp{u,v}$の「制定範囲」と言われる($v\in\mathrm{Ab}^{\epsilon}$なら「廃止範囲」).
$z$が何であるかは適用条件$ \app{\epsilon} v $によって明示的に限定される他,制御可能性と執行可能性によっても限定される.例えば,仮に$\arg v$において$ z\subseteq\app{\epsilon}y $という限定しかないなら,$z$が単一クラスであっても$\app{\epsilon}v$を充たし得る.しかし,後述の規制ルールの付帯条件で明示的に限定されない限り,$z$が小さすぎる場合には$ u\notin\cty{\epsilon}v $であると考えられる.
逆に,$ z = \app{\epsilon} y $である等$z$が大きすぎる場合は,$u\in\exe{\epsilon} v$であっても,おそらく$ z\subseteq\enf{\epsilon}y $である有意なレベルの蓋然性が成立しない.したがって,$ u\notin\enf{\epsilon}v $.一般に$x\in z$が制定時点から時間的にあまりに遠い場合,$x\in\enf{\epsilon}y$である蓋然性は有意レベルを下回る.
ただし,通常の法システムの場合,執行可能性の証明は正規性の証明で代替されるであろうから,少なくとも認定上は($ \orp{u,v} $の正規性が認定される限り),$ u\in\enf{\epsilon}v $であることが前提される.以上のことは必要な変更の上で廃止類型にも当てはまる.

改めて,$ \orp{u,v}\in\mathrm{REG}^\epsilon\uphr\mathrm{Es}^\epsilon $について,$ u\in\exe{\epsilon} v\con{1}\mathcal{L}\fap(\tilde{u}\fap 0) = \orp{z,y}\con{1}x\in z $であると仮定する.
制定関係$ \mathrm{est}^\epsilon $は,この場合における,$ \orp{u,v} $の$ \orp{x,y} $に対する関係である\footnote{
    $\tilde{u}\fap 0\in\mser{(\arg v)}{0}$である限り,$\orp{u,v}$の修正条件の危険測度$\mathcal{R}\fap(\tilde{u}\fap 0)$が何であるかは,制定関係及び廃止関係に影響しない.
}.すなわち,

\begin{df}
\label{df:制定関係}
\kagi{$
    \mathrm{est}^\epsilon
$}は,\\\hfill
\kagi{$
    \classab{\orp{\orp{u,v},\orp{x,(\mathcal{R}\resl\mathcal{L})\fap(\tilde{u}\fap 0)}}:
        \orp{u,v}\in\mathrm{REG}^\epsilon\cap(\exe{\epsilon} v\times\mathrm{Es}^\epsilon)\con{1}x\in\iter{\mathcal{L}}{2}\fap(\tilde{u}\fap 0)
    }
$}を表わす.
\end{df}

\noindent 制定関係と同様にして廃止関係$\mathrm{abo}^\epsilon$を定義する.

\begin{df}
\label{df:廃止関係}
\kagi{$
    \mathrm{abo}^\epsilon
$}は,\\\hfill
\kagi{$
    \classab{\orp{\orp{u,v},\orp{x,(\mathcal{R}\resl\mathcal{L})\fap(\tilde{u}\fap 0)}}:
        \orp{u,v}\in\mathrm{REG}^\epsilon\cap(\exe{\epsilon} v\times\mathrm{Ab}^\epsilon)\con{1}x\in\iter{\mathcal{L}}{2}\fap(\tilde{u}\fap 0)
    }
$}を表わす.
\end{df}

\noindent 制定関係と廃止関係を使って,規制要素の集合$\alpha$に相対的な正規集合を定義することができる.すなわち,規制要素$ k $が$\alpha$の正規集合$ \varSigma^\epsilon\alpha $に属するのは,$ \alpha\subseteq w $であり,
\begin{multline*}
    \text{
        任意の$y$について,
        $ x\in w $が$ y $に対して制定関係を持ち,かつ,}\\
    \text{$y$に対して廃止関係を持つどの$ z\in w $も$x$に依存しないならば,$ y\in w $
       }
\end{multline*}
であるすべての$w$の共通部分に$k$が属するとき,かつそのときに限られる.したがって,

\begin{df}
\label{df:正規集合}
\kagi{$
    \varSigma^{\epsilon}\alpha
$}は\kagi{$
    \intersect{
        \classab{w:
            \alpha\subseteq w\con{1}
            (y)[(\exists x)(
                \orp{x,y}\in w\uphl\mathrm{est}^\epsilon\con{1}\\\hspace*{38mm}
                \neg(\exists z)(\orp{z,y}\in w\uphl\mathrm{abo}^\epsilon\con{1}
                    \orp{\mathcal{L}\fap x,\exe{\epsilon}{(\mathcal{R}\fap x)}}\to_{\epsilon}\orp{\mathcal{L}\fap z,\exe{\epsilon}{(\mathcal{R}\fap z)}}
                )
            )\\\hfill
            \case{1}{0}{1}y\in w]
        }
    }
$}を表わす.
\end{df}

\noindent 定義により,$ \alpha $メンバーが規制要素なら,その正規集合のすべてのメンバーが規制要素である.つまり,
\[
    \alpha\subseteq\classab{\orp{x,y}:x\in\app{\epsilon} y}\case{1}{1}{1}\varSigma^{\epsilon}\alpha\subseteq\classab{\orp{x,y}:x\in\app{\epsilon} y}.
\]
次に,$\alpha$の規制体系$ \mathrm{SY}^{\epsilon}\alpha $は,$\alpha$の正規集合と$ \mathrm{REG}^\epsilon $との共通部分である.

\begin{df}
\label{df:規制体系}
\kagi{$
    \mathrm{SY}^{\epsilon}\alpha
$}は\kagi{$
    \varSigma^{\epsilon}\alpha\cap \mathrm{REG}^{\epsilon}
$}を表わす.
\end{df}

\noindent $ \alpha $のメンバーは,正規集合$ \varSigma^{\epsilon}\alpha $または規制体系$ \mathrm{SY}^{\epsilon}\alpha $における始祖と言われる.始祖の集合として何を選ぶかは文脈依存的であり,それは正規集合や規制体系の概念を使用する目的または機能によって異なることが想定されている.この点,正規集合の概念は,執行可能性の認定や言語的統制において重要な役割を持つ.
また,規制体系の概念は,その部分クラスの工学的妥当性が当該体系内の他の規制の機能を促進したり阻害したりする関係に基づいているという意味で,工学的妥当性を記述する機能を持つ.
さらに,規制体系外の規制要素との関係において規制体系内の機能促進や阻害が生じるケースもあるから,正規集合の概念も(規制体系の前提概念としてだけではなく)工学的妥当性を記述する機能を持ち得る.
なお,高次規制としてのN規制は,制定規制としてP規制またはS規制が始祖に遡及する過程に出現するか,または,廃止規制としてそのような過程を切断することによって,間接的な規制的機能を持つ,と言うことができる.逆に高次規制ではないN規制を持つような規制体系は,部分的にその工学的妥当性が問われることになると思われる.

ところで,$ \mathrm{est}^\epsilon $及び$ \mathrm{abo}^\epsilon $の左域は$ \mathrm{REG}^\epsilon $の部分クラスである.D \ref{df:制定関係}及びD \ref{df:廃止関係}からこの制限を外すと,例えば,$\mathcal{L}\fap z\notin \enf{\epsilon}(\mathcal{R}\fap z)$,つまり執行可能性を持たない外形的な廃止規制$z$について,$\orp{z,y}\in\mathrm{abo}^\epsilon$であり得る.するとこの場合,$y$の正規性が失われる.しかし,$y$はいわば無効な廃止実行の対象であるから,依然として正規集合そして規制体系の成員であるとする方がよい.

次に,D \ref{df:正規集合}で暗に示されているが,制定関係と廃止関係は同一の対象に対して競合し得る.例えば,
$ \orp{u,v},\orp{s,t}\in\varSigma^{\epsilon}\alpha $について,$ \orp{\orp{u,v},\orp{a,b}}\in\mathrm{est}^\epsilon\con{1}\orp{\orp{s,t},\orp{a,b}}\in \mathrm{abo}^\epsilon $とすると,定義により,
\begin{gather*}
    \tilde{u}\fap 0\in \tildel{\trgl{v}}\fap 0\con{1}\tilde{s}\fap 0\in \tildel{\trgl{t}}\fap 0,\\
    \tilde{u}\fap 0 = \orp{\orp{c,b},i}\con{1}\tilde{s}\fap 0 = \orp{\orp{c',b},i'}\con{1}a\in c\cap c'
\end{gather*}
なる$c,c',i,i'$が存在する.さらに,$i = i' = 1$であると仮定すると,$c\subseteq\enf{\epsilon}b\con{1}c'\subseteq\barl{(\enf{\epsilon}b)}$.それゆえ,$a\in\enf{\epsilon}b\cap\barl{(\enf{\epsilon}b)} $となって矛盾が生じる.したがって,競合する制定実行と廃止実行の修正条件の実現について,その少なくとも一方の蓋然性レベルは$1$未満である.

さらに,制定関係と廃止関係のいずれか一方が同一の対象に対して競合するケースも,結論的には認められる.
上記の例を一部変更して,
\[
    \orp{\orp{u,v},\orp{a,b}}\in\mathrm{est}^\epsilon\con{1}\orp{\orp{s,t},\orp{a,b}}\in \mathrm{est}^\epsilon
\]
であると仮定しよう.すると,$c\subseteq\enf{\epsilon}b$であるレベル$i$の蓋然性と,$c'\subseteq\enf{\epsilon}b$であるレベル$i'$の蓋然性とが成立する.
ここで$c = c'$である場合,事実上\footnote{
    $i = i'$のとき:$u\in\exe{\epsilon}v$でなくても,$s\in\exe{\epsilon}t$によって同じ結果が生じるから,$u\notin\enf{\epsilon}v$.同様にして,$s\notin\enf{\epsilon}t$.次に,$i \neq i'$のとき:①$i>i'$ならば,$u\in\exe{\epsilon}v$でなくても,$s\in\exe{\epsilon}t$によって同じ結果が生じるから,$u\notin\enf{\epsilon}v$.②$i<i'$ならば,$s\in\exe{\epsilon}t$でなくても,$u\in\exe{\epsilon}v$によって同じ結果が生じるから,$s\notin\enf{\epsilon}t$.
},
\setcounter{equation}{0}
\begin{gather}
    i = i' \case{1}{1}{1}u\notin\enf{\epsilon}v\con{1}s\notin\enf{\epsilon}t,\\
    i\neq i'\case{1}{1}{2}u\notin\enf{\epsilon}v\case{2}{1}{1}s\notin\enf{\epsilon}t.
\end{gather}
それゆえ,仮定と矛盾するから競合は成立しないように見える.しかし,通常の法体系等の認定システムにおいては,執行可能性の認定は正規性によって代替されると考えられるから,(1)(2)を考慮する余地はない.また,$c\neq c'$である場合は,差し当たって問題は生じない.以下では,制定関係と廃止関係のいずれか一方が競合し得ることを前提とする.

\subsubsection{工学的妥当性}
\label{sssec:工学的妥当性}

クラス$ e\subseteq \mathrm{SY}^{\epsilon}\alpha\cup \varSigma^{\epsilon}\alpha $の工学的妥当性は,
\begin{align*}
    d = \mathcal{P}(\classab{\orp{x,y}:x\in\enf{\epsilon}y})\cap\trgl{\arg\epsilon}
\end{align*}
について,$ e\in d $であることの因果的帰結\footnote{
    $e$メンバーに係る執行可能性の維持コスト等の他,制御構造の惹起を介した因果的帰結が重要である.$e$メンバーに係る構成要件実現がないケースでは,執行可能性を肯定する言語的傾向性を介して,当該準拠領域に制御構造が構築され得る.
}に基づいて測ることができる.
それゆえ,条件$ \orp{e,d} $の工学的妥当性と言い換えてもよい.
この点については,あくまで素描であるが,より一般的な枠組みを作ることができる.すなわち,任意の条件$\orp{e,d}$について,時区間$t$におけるその工学的妥当性その他の重要度を測る構造(評価構造)は,人間等のユーザー集合$\kappa$と評価関数$\gamma $に相対的に構成される.
まず,$ e\in d $であることが因果的に決定する$ \alpha $メンバーの促進条件のクラスを,
\begin{multline*}
    \zeta_1 = \classab{\orp{\tilde{a}\fap 0,\tildel{\trgl{b}}\fap 0}:
    \orp{e,d}\to_{\epsilon}\orp{\tilde{a}\fap 0,\tildel{\trgl{b}}\fap 0}\con{1}a\in\cty{\epsilon}b\con{1}
    a\fap 0\subseteq t\con{1}\tilde{a}\fap 1 \in \kappa\con{1}\\
    b\in\mathrm{Reg}\con{1}
    (\exists x)(\tildel{\trgl{b}}\fap 0 = x^{::\epsilon}\cap\mser{(\arg b)}{0})\con{1}
    \tildel{\trgl{b}}\fap 1 = \trgl{b}\fap 0
    }
\end{multline*}
と置く.同じく抑制条件のクラスを,
\begin{multline*}
    \zeta_2 = \classab{\orp{\tilde{a}\fap 0,\tildel{\trgl{b}}\fap 0}:
    \orp{e,d}\to_{\epsilon}\orp{\tilde{a}\fap 0,\tildel{\trgl{b}}\fap 0}\con{1}a\in\cty{\epsilon}b\con{1}
    a\fap 0\subseteq t\con{1}\tilde{a}\fap 1 \in \kappa\con{1}\\
    b\in\mathrm{Reg}\con{1}
    (\exists x)(\tildel{\trgl{b}}\fap 0 = x^{::\epsilon}\cap\mser{(\arg b)}{0})\con{1}
    \tildel{\trgl{b}}\fap 1 = \barl{(\trgl{b}\fap 0)}\cap\msec{(\arg b)}{0}
    }
\end{multline*}
と置く.
評価関数$\gamma$の独立変項は,手続上包括的に評価される$\zeta_1$の部分クラス,または,同じく$\zeta_2$の部分クラスである.つまり,
\[
    (w)(w\in\breve{\gamma}\img\univ\case{1}{1}{2}w\subseteq\zeta_1\case{2}{1}{1}w\subseteq\zeta_2).
\]
そして$\gamma$は,独立変項に対してある数(評価の値)を与える.この数は差し当たり自然数で足りる.すると,条件$\orp{e,d}$の重要度は,
\[
\tag*{(@)}    (\gamma\img\mathcal{P}(\zeta_1)\text{ のメンバーの総和}) - (\gamma\img\mathcal{P}(\zeta_2)\text{ のメンバーの総和})
\]
で測ることができる.この値がマイナスになる場合,デメリットの方が大きいことを意味する.
$\gamma$の位置に来る関数は評価の企画ごとに異なるが,典型的には,
$\beta\img\univ\subseteq\mathbb{N}\con{1}\breve{\beta}\img\univ\subseteq\zeta_1\cup\zeta_2$なる媒介関数$\beta$に基づいて($\gamma$は)構成される.なお,$\beta$はレベル$1$の蓋然性を基準値として,レベルが加算されるごとに$ 1/\indx{\epsilon} $をマイナスする.つまり任意の$\orp{\orp{b,m},x}\in\breve{\beta}\img\univ$について,
\[
    \beta\fap\orp{\orp{b,m},x} = \beta\fap\orp{\orp{b,1},x} - (\breve{\mathrm{S}}\fap m/\indx{\epsilon}).
\]
そして,$w\in\arg\gamma$について,$\gamma\fap w$は,単純に$\beta\img w$のメンバーの合計値でもよいが,$\gamma$が表わす関数によってはさらに制限される.
例えば,$w$は機能的に関連する修正条件のクラスであり,$\beta\img w$のメンバーの合計値$n$と,$\gamma$に固有の制限値$j$について,
\[
   n\leq j\case{1}{1}{1}\gamma\fap w = n\con{2}n > j\case{1}{1}{1}\gamma\fap w = j
\]
である等\footnote{
    このような評価構造は刑事法の量刑判断の構造と類似する(あるいは後者は評価構造の一種である).媒介関数は個別の構成要件実現に関する量刑に,評価関数は手続上包括されるそれらのクラスに関する量刑に相当する.そして制限値$j$は,例えば,当該企画で許容される($\beta$とは限らない)媒介関数が$w\in\arg\gamma$のメンバーに付与する値のクラスの最大元である.これは上記の量刑手続上包括されるクラスについて,そのメンバーに対して可能な最大の量刑に相当する.
}.$\gamma,\beta$はこれより複雑なものでも単純なものでもあり得る.

さて,規制集合の工学的妥当性に戻ろう.すなわち,
\[
    e\subseteq \mathrm{SY}^{\epsilon}\alpha\cup \varSigma^{\epsilon}\alpha\con{1}d = \mathcal{P}(\classab{\orp{x,y}:x\in\enf{\epsilon}y})\cap\trgl{\arg\epsilon}
\]
である$ \orp{e,d} $の工学的妥当性を測るケースでは,$\kappa$は国籍保持者の集合で,$\gamma$の独立変項は,以下のように,ある$z\times\classab{y}\subseteq\mathrm{SY}^{\epsilon}\alpha$の機能を構成するクラス$w$,または,当該機能の阻止を構成するクラス$w'$になるだろう.後者に関連して,促進条件の阻止は抑制条件であると仮定している.
\begin{gather*}
w = \union{
\classab{x:x\subseteq\zeta_1\con{1}
    \orp{x,\mathfrak{E}\cap\trgl{\arg\epsilon}}\text{ は }z\times\classab{y}\text{ の機能}
}
},\\
w' = \classab{\orp{\tilde{a}\fap 0,\barl{(\tildel{\trgl{b}}\fap 0)}\cap\mser{(\arg b)}{0}}:
    \orp{\tilde{a}\fap 0,\tildel{\trgl{b}}\fap 0}\in w
}\cap\zeta_2.
\end{gather*}

そして,(@)がマイナスである場合,$ \orp{e,d} $は工学的妥当性を欠く.しかし,$ e\subseteq g \subseteq \mathrm{SY}^{\epsilon}\alpha\cup \varSigma^{\epsilon}\alpha $については,プラスサイドも変動するからから工学的妥当性を欠くとは限らない.ただし,$e$の評価値をプラス転換して工学的妥当性を回復した$ e' $は,それが$ g $に埋め込まれた場合に新たなマイナス要因を生成しない限り,元の$g$よりも$ (g\cap\bar{e})\cup e' $の工学的妥当性を高める.
例えば,$ x_1,x_2,r\in\varSigma^{\epsilon}\alpha $について,
\[
    x_1\neq x_2\con{1}\orp{x_1,p},\orp{x_2,p}\in\mathrm{est}^\epsilon\con{1}\orp{r,p}\in \mathrm{abo}^\epsilon
\]
と仮定する.また,制定実行$ \mathcal{L}\fap x_1\in\exe{\epsilon}{(\mathcal{R}\fap x_1)} $及び$ \mathcal{L}\fap x_2\in\exe{\epsilon}{(\mathcal{R}\fap x_2)} $は,廃止実行$ \mathcal{L}\fap r\in\exe{\epsilon}{(\mathcal{R}\fap r)} $より,時間的に先行しているとしよう\footnote{
    一般に,廃止実行は先行する制定実行に依存しているが,時間順序が逆の場合はそうではない.ただし,これは時間遡行的な因果関係が通常は成立しないことの反映であり,制定と廃止の時間順序それ自体によって正規性を規定する先天的な理由はないと思われる.
}.
この場合,$r$の実行が,$x_1$の実行と$x_2$の実行のぞれぞれに依存するような特殊なケースを除いて,$p$の正規性は失われない.すなわち,$x_1$と$x_2$の実行による影響の両方を除去しようとする場合,片方がなくても$r$は実行されていたであろうから,$r$の実行は両者に依存しないことになる.また,一方の実行による影響のみを除去しようとする場合,$r$の実行は他方には依存しない.いずれにせよ,$p$の正規性は失われない.それゆえ,$ \varSigma^{\epsilon}\alpha $が(通常の法体系のように)正規性によって執行可能性を認定するシステムを含む場合には,$p$の執行可能性が前提される.すると,$r$はその機能を果たすことができない.
このケースは当該正規集合または規制体系に関して,以下①②の情報を与える.
\begin{enumerate}
    \item [①] $x_1,x_2$が相まって,$r$の機能を阻害する.
    \item [②] $r$の機能は,$x_1$または$x_2$の制定範囲内の規制($p$など)を廃止する$r$自身の実行によるのではなく,$x_1$または$x_2$それ自体を廃止する規制$r'$によってよりよく果たされる.
\end{enumerate}
プラスサイドを考慮していないから,①だけで$ x_1,x_2 $を含むクラスが工学的妥当性を欠くとは言えない.しかし,$ \varSigma^{\epsilon}\alpha $から$r$を排除し,代わりに$r'$を追加することによって,他の条件が同じであれば,マイナスが減りプラスが増加する.したがって,いずれにせよその工学的妥当性を高めることができる\footnote{
    この他,P規制の構成要件実現がS規制のそれを因果的に決定するような場合にも工学的妥当性に問題を生じる.また,何らかの評価構造に基づく工学的妥当性を充足しないこと自体を構成要件として,端的にその規制集合のメンバーを廃止する,というN規制が体系内に含まれる可能性がある.憲法の人権規制等はそのようなN規制と解する余地がある.
}.

なお,規制の工学的妥当性の判断は規制の機能という概念に頼っているが,この概念については概ね次のように説明できよう.
規制集合の機能は,比喩的に言えば,当該規制集合が廃止されない理由である.
この点,P規制である廃止規制$ \orp{s,t} $について,$ \orp{\orp{u,v},\orp{s,t}}\in \mathrm{IS} $なる$ \orp{u,v} $がP規制である制定規制であるとき,$u\in\exe{\epsilon} v$は,廃止を実行しない不作為による制定行動と言うことができる.極めて図式的に言えば,$ \tilde{s}\fap 1 $が人間であり,$ \orp{s,t} $が刺激過程→身体運動→廃止効果という因果系列を持つならば,$ \orp{u,v} $の因果系列は刺激過程→身体運動の不在→制定効果となる.また,$ \tilde{s}\fap 1 $が議会等で,$ \orp{s,t} $が議事手続→廃止決議→廃止効果という因果系列を持つならば,$ \orp{u,v} $の因果系列は議事手続→廃止決議の不在→制定効果となる.

さて,$b\in k$が,規制集合$ z\times\classab{y}\subseteq\mathrm{SY}^{\epsilon}\alpha $の($ \mathrm{SY}^{\epsilon}\alpha $における)機能であるのは,
\begin{align*}
    \orp{\orp{u,v},\orp{s,t}}\in\mathrm{IS}\cap((\mathrm{REG}^\epsilon\uphr\mathrm{Es}^\epsilon)\times(\mathrm{SY}^{\epsilon}\alpha\uphr\mathrm{Ab}^\epsilon))
    \con{1}\mathcal{L}\fap(\tilde{s}\fap 0) = \orp{z,y}\con{1}\tildel{\trgl{v}}\fap 1 = \tildel{\trgl{t}}\fap 1 = \trgl{t}\fap 0
\end{align*}
である$ \orp{u,v},\orp{s,t} $が存在して,$ z\subseteq\enf{\epsilon}y $が$b\in k$を因果的に決定するということが,$ u\in\exe{\epsilon} v $を因果的に決定しているとき,かつそのときに限られる.
すなわち,
\begin{align*}
    \orp{\orp{\tilde{u}\fap 0,b},(\tildel{\trgl{v}}\fap 0)\bkg{\epsilon}k}
    \to_{\epsilon}\orp{u,\exe{\epsilon}{v}}
    \con{1}u\in\exe{\epsilon}v
\end{align*}
であるとき,かつそのときに限り,$ b\in k $は$ z\times\classab{y} $の機能である.なお,$ \orp{u,v}\in\mathrm{SY}^{\epsilon}\alpha $である必要はない.

\subsection{正規性の認定}
\label{ssec:正規性の認定}

既に述べたように,$ (\mathrm{est}^\epsilon)\img\univ,(\mathrm{abo}^\epsilon)\img\univ\subseteq\mathrm{REG}^\epsilon $であるが,$ \mathrm{est}^\epsilon $及び$ \mathrm{abo}^\epsilon $の右域のメンバーは規制である必要はない(定義上規制要素ではある).つまり,$ \orp{u,v}\in\varSigma^{\epsilon}\alpha $であり,かつ,$ \orp{u,v}\notin \mathrm{REG}^{\epsilon} $ということも一般に可能である.制定関係及び廃止関係がこのように定義されている理由は,
\begin{equation}
    (u)(v)(\orp{u,v}\in\varSigma^{\epsilon}\alpha\case{1}{1}{1}u\in\enf{\epsilon}v) \tag*{(※)}
\end{equation}
を(形式的体系の公理のように)推論の前提とする制御構造が規制システムの運用にとって重要であることによる.すなわち,この制御構造は,(i)認定システムが正規性を認定する推論過程において,または,(ii)執行随伴性の言語的代替を形成する推論過程において出現する.もし定義上$ \varSigma^{\epsilon}\alpha\subseteq \mathrm{REG}^\epsilon $ならば,(※)は同語反復に近い論理的真理になってしまい,システムの運用に寄与するような因果的な重要性を持たなくなる.

\subsubsection{認定システムによる正規性論証}
\label{sssec:認定システムによる正規性論証}

高次類型ではない規制類型$y$について,その執行可能性$ x\in\enf{\epsilon}y $の因果的構造には,通常,上記(i)に関連する制御構造が含まれている.その意味はこうである.$ x\in\enf{\epsilon}y $であるとき,
\[
    z\fap 0 = \orp{x,\exe{\epsilon} y}\con{1}\tilde{z}\fap 0 = \orp{\tilde{x}\fap 0,\tildel{\trgl{y}}\fap 0}
\]
である因果系列$z$が存在する.しかも,通常,ある$i\in\arg z$について,$ \orp{z\fap i,z\fap(\mathrm{S}\fap i)}\in \mathcal{C}^{\epsilon} $は,証拠に基づいて事実認定を行う認定システムの制御構造の実現である.つまり,ある$a$と$b\in\mathrm{Reg}$について,
\[
    z\fap i = \orp{a,\app{\epsilon} b}\con{1}z\fap(\mathrm{S}\fap i)= \orp{a\fap 0,\tildel{\trgl{b}}\fap 1}.
\]
そして,この認定システムの制御構造$ a\in\cs{\epsilon}b $もまた,さらに細分化された因果関係へと分解することができ,その中には以下の3個の制御構造が含まれる.あるいは,$ a\in\cs{\epsilon}b $自体をこれらの制御構造を合成した複合的な事態と捉えることもできる(この場合,多分$b$はN類型になる).
\begin{enumerate}[label=(\arabic*)]
    \item $ x\in\exe{\epsilon} y $であることを認定する制御構造
    \item $ (\tildel{\trgl{y}}\fap 1\neq\Lambda\case{1}{1}{1}x\in \cty{\epsilon}y) $であることを認定する制御構造
    \item $ \orp{x,y}\in\varSigma^{\epsilon}\alpha $であることを認定する制御構造
\end{enumerate}
(1)は,構成要件該当性の情報を検出してそれを執行システムまで運ぶ仕組みの一部である.修正条件の導入を自然的な因果に頼るのでない限りこのような仕組みが必要になる.
(2)は,修正条件の有効性を確認するプロセスであるが,規制体系内の他の機能を促進するために省略されることがある.しかし,抑制機能を持つ修正条件の副作用が意識される場合には,修正条件導入のフロー効率を犠牲にしてでも維持される傾向がある.しかし,(2)をさらに分解すると,省略されない要素として,修正条件の実現$\tilde{x}\fap 0\in\tildel{\trgl{y}}\fap 0$を認定する制御構造が含まれると考えられる.これがないと,$\tilde{x}\fap 0\notin \tildel{\trgl{y}}\fap 0$であっても,(1)と(3)により$\orp{x,y}\in\enf{\epsilon}y$を認定することになって問題が生じる.
(3)は,その構成要件の実現に対して修正条件が導入されるような規制類型の範囲を限定する.規制類型の数が無限であるためこのような限定が必要となる\footnote{
    (3)は,規制を制定するシステムとそれを認定するシステムの分割を前提としているわけではない.仮に認定システムがその都度問題の規制類型の範囲を決定する(非高次類型からなる始祖の集合を決定する)ような仕組みであったとしても,正規性の認定基準を最適化できるかはともかく,(3)の制御構造自体は存在する.
}.この(3)の制御構造をさらに分解すると,そこに(※)を前提として使用する制御構造を見出すことができる.すなわち,系列の前者が後者に対して,関係
\[
    \classab{\orp{\orp{u,v},\orp{x,y}}:
    x\in (\mathcal{L}\resl\mathcal{L})\fap(\tilde{u}\fap 0)\con{1}y = (\mathcal{R}\resl\mathcal{L})\fap(\tilde{u}\fap 0)\con{1}v\in\mathrm{Es}^\epsilon
    }
\]
を持つような規制要素の系列$ k $が存在して,
\[
    k\fap 0 \in\alpha\con{1}\tilde{k}\fap 0 = \orp{x,y}\con{1}u = \mathcal{L}\resl k\con{1}v = \mathcal{R}\resl k
\]
と置く.すると,次のような論証を構成することができる\footnote{
    (※)は,正規性の論証の出発点として,$ \alpha $メンバー(始祖)の執行可能性が単に前提されるということを含意している.
}.
\begin{nom}
    \setcounter{equation}{0}
$ \alpha\subseteq\varSigma^{\epsilon}\alpha $であるから,$ k\fap 0 =\orp{u\fap 0,v\fap 0}\in\varSigma^{\epsilon}\alpha $.すると(※)により,
    $ u\fap 0\in\enf{\epsilon}(v\fap 0) $.それゆえ,
    \begin{align}
        u\fap 0\in\exe{\epsilon}{(v\fap 0)}\con{2}
        \tildel{\trgl{(v\fap 0)}}\fap 1\neq\Lambda\case{1}{1}{1}
        u\fap 0\in\cty{\epsilon}(v\fap 0)
        \tag*{[1]}
    \end{align}
    と仮定すると,$ \orp{k\fap 0,k\fap 1}\in\mathrm{est}^\epsilon $.さらに,
    \begin{align}
        \neg(\exists z)(\orp{z,k\fap 1}\in \varSigma^{\epsilon}\alpha\uphl\mathrm{abo}^\epsilon\con{1}
        \orp{u\fap 0,\exe{\epsilon}{(v\fap 0)}}\to_{\epsilon}\orp{\mathcal{L}\fap z,\exe{\epsilon}{(\mathcal{R}\fap z)}}
        )
        \tag*{[2]}
    \end{align}
    と仮定すると,$ k\fap 1 = \orp{u\fap 1,v\fap 1}\in \varSigma^{\epsilon}\alpha $.すると(※)により,$ u\fap 1\in\enf{\epsilon}(v\fap 1) $.
    
    以上を繰り返すと,[1] [2] $\dots$ という仮定のもとで,$ \tilde{k}\fap 0 = \orp{x,y}\in \varSigma^{\epsilon}\alpha $.
    したがって,仮定[1] [2] $\dots$ が証拠に基づいて認定されれば,$ \orp{x,y}\in \varSigma^{\epsilon}\alpha $であることを認定できる(さらに(※)により,$ x\in\enf{\epsilon}y $であることも認定できる).
\end{nom}

なお,$ \orp{x,y}\in \varSigma^{\epsilon}\alpha $と,$ x\in\enf{\epsilon}y $の構成要件実現と修正条件実現とを連結する因果系列$ z $が存在して,
\[
    (\exists i)(
        i\in\arg z \con{1}z\fap i = \orp{s,\app{\epsilon} t}\con{1}z\fap(\mathrm{S}\fap i)= \orp{s\fap 0,\tildel{\trgl{t}}\fap 1}
    )
\]
であるとしても,$ (\exists u)(\orp{u,t}\in \varSigma^{\epsilon}\alpha) $とは限らない.
上述の認定システムの例で言えば,$ \orp{s,t}=\orp{a,b} $であるか,$ s\in\cs{\epsilon}t $が,$ a\in\cs{\epsilon}b $を分解した(1)〜(3)の制御構造,または,(3)の部分構造である(※)を前提として使う制御構造であるとしても,$ (\exists u)(\orp{u,t}\in \varSigma^{\epsilon}\alpha) $とは限らない.
しかし,$ \varSigma^{\epsilon}\alpha $が法体系等の正規集合ならば,制御構造$ s\in\cs{\epsilon}t $は,通常,$ \varSigma^{\epsilon}\alpha $に属する他の規制類型に係る執行可能性(またはそれに依存する制御構造\footnote{実現した制御構造は執行随伴性またはその言語的代替に依存しているが,通常の状況では,いずれの条件も執行可能性に依存している.})に依存している.
すなわち,$ e\notin\arg d $かもしれない$ e $と$ d\in \brevel{(\varSigma^{\epsilon}\alpha)}\img\univ $が存在して,$ \orp{e,\enf{\epsilon}d}\to_{\epsilon}\orp{s,\cs{\epsilon}t} $.そして,$ d $は例えば以下のようなN類型である.

$ \exe{\epsilon}{d} $は,結節点に帰属可能な行動ではなく,行動の集積としての手続の実現であり\footnote{
    手続を構成する行動について,それがあるパターンに適合しない場合に抑制するS規制もまた,同一体系内にあるかもしれない.
},適用条件$ \arg d $により「適正」な手続であることが要求される.この点,(1)〜(3)の認定行動(文を間主観的に肯定する言語行動)を含まない手続は適正さを欠く.また,手続に含まれる認定行動を惹起している刺激群は,適切な証拠を構成していなければならない(証明規則).例えば次の証明規則が考えられる.
\begin{itemize}
    \item (3)に含まれる認定行動が(※)を前提として使用していない場合,適切な証拠の条件を欠く.関連して,(※)の\kagi{$ \alpha $}には,憲法等の基本法の制定規制及び廃止規制の集合を指示する抽象体を代入しなければならない.
    \item (2)に含まれる認定行動が,
    \[
        (x)(y)(\tildel{\trgl{y}}\fap 1 = \barl{(\trgl{y}\fap 0)}\cap\msec{(\arg y)}{0}\case{1}{1}{1}Gxy\case{3}{0}{1}x\in \cty{\epsilon}y)
    \]
    を前提として使用していない場合,適切な証拠の条件を欠く.ただし,\kagi{$ Gxy $}は$ x\in\exe{\epsilon} y $が有責に実現されたことを記述する文の位置を表わしている.
\end{itemize}
そして,修正条件の実現$ \tilde{e}\fap 0\in \tildel{\trgl{d}}\fap 0 $は,執行規制の制定であって,執行機関の行動を直接起動するような条件(他の規制類型の適用条件)の一部ではない.執行規制は,上位執行機関を結節点に持ち,下位執行機関の実行規制を制定するP規制であるか,裁判を含む執行手続によって実行規制を制定するN規制である.実行規制は,ある$ w\subseteq\mathrm{REG}^\epsilon $について,$ w\subseteq\classab{\orp{p,q}:p\in\exe{\epsilon}q} $であることによって$ \exe{\epsilon}d $で認定される規制の修正条件が直接実現される場合において\footnote{
    例えば,金銭債務(金額占有の移転阻止に対するS規制)の修正条件を債務者の何らかの財産の喪失であるとした場合,それの実行規制は,執行裁判所の売却許可決定を含む執行手続によって制定される(売買の)引渡債務等である.
},P規制$\orp{p,q}\in w$か,$ \mathrm{IS}\img\classab{\orp{p,q}} $のメンバーのS規制である.

\subsubsection{言語的統制における正規性論証}
\label{sssec:言語的統制における正規性論証}

次に,(※)を推論の前提とする制御構造は,執行随伴性の言語的代替を形成する推論過程においても出現する.$ \orp{x,y}\in \mathrm{REG}^\epsilon $について,$ x\notin\exe{\epsilon} y $であると仮定する.しかし,$ \tilde{x}\fap 1 $において($ x\fap 0 $の時間領域に),執行随伴性$ x\in\exe{\epsilon} y\cap\enf{\epsilon}y $であることの言語的代替,つまり,
\begin{gather*}
    \text{$ x\in\enf{\epsilon}y $であることを記述する文を肯定する制御構造}\\
    \text{(その文の真偽を問われたならば,それを肯定するだろう)}
\end{gather*}
があれば,ある$\tilde{w}\fap 1 = \tilde{x}\fap 1 $について$ w\in\cs{\epsilon}y $が成立し得る\footnote{もちろん,制御構造の因果的決定要因が執行随伴性またはその言語的代替に尽きるわけではない.ある条件実現を因果的に決定する条件実現は無数に存在する.}.ただし,$ \tildel{\trgl{y}}\fap 1 \neq \Lambda $ならば,$ x\in\cty{\epsilon}y $でない限り,このような制御構造の構築は起きないと考えられる.
また通常の状況では,$ x\in \enf{\epsilon}y $であることによって,それを肯定する制御構造も形成され得る.他方,$ \tilde{x}\fap 1 $において,(※)が前提として受容されているならば,前述の正規性の論証によって,$ x\in\enf{\epsilon}y $を肯定する制御構造を形成できる.つまり,
\begin{dem}
    正規性論証により,$ \orp{x,y}\in\varSigma^{\epsilon}\alpha $.すると,(※)により,$ x\in\enf{\epsilon}y $.
\end{dem}
\noindent なお,正規性論証がそれについてのものである,始祖から$\orp{x,y}$に至る\ref{sssec:認定システムによる正規性論証}の系列$k$について,その系列要素の中には右成分がN類型である規制要素が含まれ得る.つまりN規制は,言語的統制を媒介することによって間接的な規制的機能を持つことができる.

ところで,$ \tilde{x}\fap 1 $は,正規性の論証を行うために必要な$ \varSigma^{\epsilon}\alpha $に関する情報をどこから得るのかという問題が残る.
前述の認定システムが行う正規性の論証においては,当該システムが正規集合に関する情報を得ていることが前提されていた.なぜなら,認定システムがその機能を果たすためには正規性の情報が不可欠であり,それゆえ,その情報へのアクセス可能性は認定システムの基本設計に含まれると考えられるからである.
他方,任意の$ x\in \arg y $について,$ \tilde{x}\fap 1 $に対して正規集合に関する情報へのアクセス可能性を付与する方法は,別に構築しなければならない.この点については,太古からの方法として,規制ルールの公布によって規制を制定するという仕組みがある.

\subsection{規制ルール}
\label{ssec:規制ルール}

\subsubsection{規制表明}
\label{sssec:規制表明}

任意の$y\in\mathrm{Reg}$と$b$(付帯条件)について,$ \breve{\epsilon}\fap \Lambda $において$ \app{\epsilon} y\cap b \subseteq \enf{\epsilon}y $である蓋然性を記述する$\mathfrak{L}$の論理式を,$b$に制限された$y$の制定ルールと言う.同様に,$ \app{\epsilon} y\cap b \subseteq \barl{(\enf{\epsilon}y)} $である蓋然性を記述する$\mathfrak{L}$の論理式を,$b$に制限された$y$の廃止ルールと言う.そして,制定ルールと廃止ルールを併せて規制ルールと言う.
典型的には,下記(1)(2)の型式の\kagi{$ \delta $}に($ \breve{\epsilon}\fap\Lambda $において)$y$を指示する$\mathfrak{L}$のクラス抽象体\footnote{
    $ (\breve{\epsilon}\fap\Lambda)\exten p = x $である任意の論理式$p\in\mathfrak{L}$について,\kagi{$ \classab{\orp{\phi}:\psi} $}の\kagi{$ \phi $}に,$ \mathrm{var}\,p $の変項を間に\kagi{$ , $}を挟んで順次代入し,\kagi{$ \psi $}に$p$を代入した結果は,$ \breve{\epsilon}\fap\Lambda $において$ x $を指示するクラス抽象体である.
}を,\kagi{$ \beta $}に($ \breve{\epsilon}\fap\Lambda $において)$b$を指示する$\mathfrak{L}$のクラス抽象体を代入して(省略記法を)展開すると,今述べた制定ルールと廃止ルールが得られる.
\setcounter{equation}{0}
\begin{align}
    &\textbf{制定ルールの型式}\qquad \app{\epsilon}\delta\cap\beta\in\mathcal{P}(\prob{(\enf{\epsilon}\delta)}{\indx{\epsilon}\uphl\epsilon}),\\
    &\textbf{廃止ルールの型式}\qquad \app{\epsilon}\delta\cap\beta\in\mathcal{P}(\prob{(\barl{(\enf{\epsilon}\delta)}\cap\trgl{\arg\epsilon})}{\indx{\epsilon}\uphl\epsilon}).
\end{align}
次の2個の定義によって,$\beta$に制限された$\delta$の制定ルールのクラスと,$\beta$に制限された$\delta$の廃止ルールのクラスがそれぞれ導入される.
\begin{df}
\label{df:制定ルール}
\kagi{$
    \beta\stackrel{\epsilon}{\mathrm{er}}\delta
$}は\kagi{$
    \mathcal{K}^{(\breve{\epsilon}\fap\Lambda)}\img\classab{\orp{
        \arg \delta\cap \beta,\mathcal{P}(\prob{(\enf{\epsilon}\delta)}{\indx{\epsilon}\uphl\epsilon})\cap\trgl{\arg\epsilon}
    }}
$}を表わす,
\end{df}

\begin{df}
\label{df:廃止ルール}
\kagi{$
    \beta\stackrel{\epsilon}{\mathrm{ar}}\delta
$}は\kagi{$
    \mathcal{K}^{(\breve{\epsilon}\fap\Lambda)}\img\classab{\orp{
        \arg \delta\cap \beta,\mathcal{P}(\prob{(\barl{(\enf{\epsilon}\delta)}\cap\trgl{\arg\epsilon})}{\indx{\epsilon}\uphl\epsilon})\cap\trgl{\arg\epsilon}
    }}
$}を表わす.
\end{df}

次に,$ \orp{u,v}\in\mathrm{REG}^\epsilon\uphr(\mathrm{Es}^\epsilon\cup\mathrm{Ab}^\epsilon) $が規制表明型の高次規制であるとき,$ u\in\exe{\epsilon} v $を「規制表明」と言う.この場合,
\[
    \tilde{u}\fap 3 = \orp{d,b,y}\con{1}\mathcal{L}\fap(\tilde{u}\fap 2) = h\con{1}
    \mathcal{L}\fap(\tilde{u}\fap 0)=\orp{z,y}\con{1}z\subseteq b
\]
なる$ h,d,b,z,y $が存在する\footnote{
    規制表明は,議会による立法と公布のようなケースに限られない.個人間の契約は当事者への規制ルールの公布によって成立する.また,登記手続 → 登記記録という因果連鎖に代表される物権の公示は,それによって(完全な)物権的規制を制定する規制表明の一種と言える.
}.そして,$ \iter{\mathrm{E}}{2}\fap(u\diamond\trgl{v}) $は,言語的トークンの集合$d$を生成して,公布範囲$h$のメンバーの結節点に対して,ある$c\in d$について$c \in \gamma$であることを蓋然的に認知させる因果系列である.ただし,ここでの$\gamma$は,$b$に制限された$y$の規制ルールを含意する式に解釈される言語的タイプについて,それのトークンのクラスとする.

規制表明型のこの特徴づけは,原初的言語には還元できないにしても,もう少し精密化できる.
まず,型文字\kagi{$ \tau $}で,$e$が言語的タイプ$p$のトークンであるような$ \orp{e,p} $のクラスを指示する抽象体の位置を表わすものとする.ただし,$ \tau\img\univ\subseteq \mathcal{P}(\timex{\mathbb{R}}{4}) $,そして,$ \breve{\tau}\img\univ $のメンバーはある同一の言語に属する.すなわち,$ p\in \breve{\tau}\img\univ $は当該言語の原始記号の有限系列であり,原始記号自体はそのトークンの集合である.また,同じ言語では1個のトークンは唯一のタイプを持つと考えられるから,$ \func\breve{\tau} $である.
次に,$ f $を,\ref{sssec:全体的言語}の言語$\mathfrak{L}$について,$q\in\mathfrak{L}$が,$ l\in\breve{\tau}\img\univ $の適切な解釈(規格化・形式化)であるような$\orp{q,l}$のクラスとする.そして型文字\kagi{$ \theta $}で,$ f $の逆$\breve{f}$を指示する抽象体の位置を表わす.

さて,$ \orp{u,v}\in\mathrm{REG}^\epsilon\uphr\mathrm{Es}^\epsilon $が規制表明型であるとき,
\setcounter{equation}{0}
\begin{gather}
        \tildel{\trgl{v}}\fap 3 = \classab{\orp{d,b,y}:d\subseteq \tau\img\univ\con{1}
        (\exists c)(
            c\in d\con{1}c\in(\tau\resl\theta\resl\brevel{\lambda_x\classab{x}}\resl\mathrm{imp})\img (b\stackrel{\epsilon}{\mathrm{er}}y)
        )
        }\cap\mser{(\arg v)}{3},\\
    \arg v\subseteq\classab{u:\mathcal{L}\fap(\tilde{u}\fap 3)\subseteq \mathcal{P}(\timex{\mathbb{R}}{4})}.
\end{gather}
$ \tilde{u}\fap 3 \in \tildel{\trgl{v}}\fap 3 $であることを因果的に惹起することは,$ d = \mathcal{L}\fap(\tilde{u}\fap 3) $のメンバーが,$\breve{\tau}\img\univ$のメンバーが属する特定の言語の言語的タイプの事例になるように,しかも,ある$ c\in d $が,$b$に制限された$y$の規制ルールを含意する式に解釈される言語的タイプ(例えば複数の条文の連言)の事例になるように,時空領域の集合$ d $を構成することを意味する\footnote{
    $c\subseteq\timex{\mathbb{R}}{4}$が言語的タイプ$r$のトークンである場合,$ q_1\neq q_2 $について,$\orp{r,q_1}\in\theta $であるか,$\orp{r,q_2}\in\theta $であるかによって,$c$の形状や物理的状態は異なり得る.この意味で,$r$が何に解釈されるのかという条件は物理的条件に間接的に言及する.また,$r$が(契約の申込に対する)単純な承諾文言であったとしても,(申込内容を参照して)規制ルールを含意する式に解釈可能である.
}.
なお,規制ルールそれ自体ではなく,それを含意する式への解釈可能性を条件とする理由はこうである.もし規制ルール自体への解釈可能性を要求すると,例えば連言\kagi{$ P\text{ かつ }Q $}が規制ルールに解釈されるが,トークンとしては連言\kagi{$ R\text{ かつ }P\text{ かつ }S\text{ かつ }Q $}のそれしかないケースで,前者の連言のトークンをそこから抽出できるのかという問題が生じる.しかし,後者の連言は規制ルールを含意する式に解釈可能であろうから,それで要求を充たすようにすれば,この問題を回避できる.

次に,文タイプの真偽が質問される状況を特定化する適用条件と,そこにおいてその文を肯定する因果系列を実現する構成要件とを持つある規制類型$ w $を考える.$w$の修正条件はコミュニケーションの成立を示すような質問者の反応を特定化するようなものと想定される.型文字\kagi{$ \eta $}で,このような$w$を指示するクラス抽象体の位置を表わすことにする.すなわち,
\begin{gather*}
    \arg \eta \subseteq\classab{e:(\exists k)(\exists r)(\tilde{e}\fap 2 = \orp{k,r}\con{1}k\subseteq \timex{\mathbb{R}}{4}\con{1}r\in\breve{\tau}\img\univ)},\\
    \tildel{\trgl{\eta}}\fap 2 = \classab{\orp{k,r}:r\in\breve{\tau}\img\univ\con{1}(\exists q)(\orp{k,q}\in\tau\con{1}\text{ $q$は$r$の肯定 })}.
\end{gather*}
$ e\in\arg \eta\con{1}\tilde{e}\fap 2 = \orp{k,r} $とすると,$ \tilde{e}\fap 2\in\tildel{\trgl{\eta}}\fap 2 $であるとき,$r$の肯定であるような言語的タイプのトークン$k$が生成される.例えば,質問「$\phi$ですか」に対する「$\phi$です」.ただし,\kagi{$ \phi $}に$r$を代入する.

$ \tildel{\trgl{v}}\fap 2 $は,$ h\subseteq\cs{\epsilon}\eta $であるレベル$i$の蓋然性を持つ$\orp{h,i}$のクラスである.つまり,
\begin{align}
    \tildel{\trgl{v}}\fap 2 = (\mathcal{P}(\cs{\epsilon}\eta)\cap\trgl{\arg\epsilon})^{::\epsilon}\cap\mser{(\arg v)}{2}.
\end{align}
$ \tilde{u}\fap 2\in \tildel{\trgl{v}}\fap 2 $であるとき,公布範囲$h$と蓋然性レベル$i$が存在して,$\tilde{u}\fap 2 = \orp{h,i}$.そして,公布範囲に属する任意の準拠領域$e\in h$について,ある文$ \mathcal{R}\fap(\tilde{e}\fap 2) $を肯定する制御構造の蓋然性(レベル$i$)が成立する.そして,$ \mathcal{R}\fap(\tilde{e}\fap 2) $は,ある$ c\in \mathcal{L}\fap(\tilde{u}\fap 3) $について,
\[
    c\in(\tau\resl\theta\resl\brevel{\lambda_x\classab{x}}\resl\mathrm{imp})\img (b\stackrel{\epsilon}{\mathrm{er}}y)
\]
であることを記述する論理式(を含意する式)に解釈される文である.すなわち,
\begin{multline}
    \arg v \subseteq \classab{u:(\exists d)(\exists b)(\exists y)(\exists t)[
        \tilde{u}\fap 3 = \orp{d,b,y}\con{1}
        t = (\tau\resl\theta\resl\brevel{\lambda_x\classab{x}}\resl\mathrm{imp})\img (b\stackrel{\epsilon}{\mathrm{er}}y)\con{1}\\
        (e)(e\in \mathcal{L}\fap(\tilde{u}\fap 2)\case{1}{1}{0}
            (\exists c)(\orp{c,e\fap 0}\in d\uphl\vec{\varphi}\con{1}
                \mathcal{R}\fap(\tilde{e}\fap 2)\in (\theta\resl\brevel{\lambda_x\classab{x}}\resl\mathrm{imp}\resl \mathcal{K}^{(\breve{\epsilon}\fap\Lambda)})
                \img\classab{\orp{ c,t }}
            )
        )
    ]
    }.
\end{multline}
$ \orp{c,e\fap 0}\in d\uphl\vec{\varphi} $であることは,$ c\in d $が$ e\fap 0 $よりも時間的に後でないこと,つまり,$e\in\cs{\epsilon}\eta$が将来の状況に関する予測的な認知でないことを意味している.
型文字\kagi{$ \varphi $}は,物理理論に適合する座標変換$f$,すなわち,
\[
    f\img\univ = \timex{\mathbb{R}}{4}\con{1}\breve{f}\img\univ = \timex{\mathbb{R}}{4}\con{1}\func f\con{1}\func\breve{f}
\]
である特定の$f$を指示するクラス抽象体の位置を表わす\footnote{物理理論に適合する座標変換であるための条件が,数学的概念だけで定式化可能であるとしても,現在選ばれている特定の座標系を指示するには他の原始的述語が必要になると思われる.}.
そして,任意の$ x\in \timex{\mathbb{R}}{4} $について,$ \varphi $に相対的に$x$と同時である$y$のクラスは,$ \varphi $に相対的な1個の時点である.そのような時点の集合を,
\begin{align*}
    \hat{\varphi} = \lambda_x\classab{y:y\in\timex{\mathbb{R}}{4}\con{1}\mathcal{L}\fap(\varphi \fap y)=\mathcal{L}\fap(\varphi \fap x)}\img(\timex{\mathbb{R}}{4})
\end{align*}
と規定する.今,$ \timex{\mathbb{R}}{4} $上の大小関係が$ \classab{\orp{x,y}:x\subset y} $となるように$\mathbb{R}$が定義されているとしよう.すると,
$t,t'\in\hat{\varphi}$の(時間的)前後関係は,$ \trgl{t}\subset\trgl{t'} $,あるいは$ \trgl{t}\subseteq\trgl{t'} $に帰着する.
また,$ \classab{\union{x}:x\subseteq\hat\varphi} $の任意のメンバーを「時間領域」と言う\footnote{連続した時点からなる時間領域は「時区間」と言われる.時点$t'$と$t''$との間の時区間は,$ \union{\classab{t:t\in\hat{\varphi}\con{1}\trgl{t'}\subseteq \trgl{t}\subseteq \trgl{t''}}} $.}.
次に,任意の$ x $に対してそれに含まれる時点の集合を与える関数を,
\begin{align*}
    \check{\varphi} =  \lambda_x\classab{t:t\in\hat{\varphi}\con{1}x\cap t\neq\Lambda}
\end{align*}
と置く.すると,内部構造に時空領域を含む$x,y$について,ある$t\in\check{\varphi}\fap(\trcl x)$がどの$t'\in\check{\varphi}\fap(\trcl y)$よりも時間的に後ではない,という意味での時間的前後関係は,
\begin{align*}
    \vec{\varphi} = \classab{\orp{x,y}:
    (\exists t)(t\in\check{\varphi}\fap(\trcl x)\con{1}
    (t')(t'\in\check{\varphi}\fap(\trcl y)\case{1}{1}{1}\trgl{t}\subseteq\trgl{t'}))
    }
\end{align*}
である.

次に,型文字\kagi{$ \varrho $}で,ある規制集合$ m $について,その正規集合$ \varSigma^{\epsilon}m $のメンバーに対して,
\[
    t\in\classab{\union{x}:x\subseteq\hat\varphi}\con{1}s\subseteq\mathcal{P}(\timex{\mathbb{R}}{4})
\]
なる$\orp{t,s}$を付与する関数$ g $を指示するクラス抽象体の位置を表わす.
すると次の条件は,$x\fap 0$に後行しない生成トークンを持つ任意の$ x\in (\mathcal{L}\resl\mathcal{L})\fap(\tilde{u}\fap 0) $と$ (\mathcal{R}\resl\mathcal{L})\fap(\tilde{u}\fap 0) $に相関して,起動領域$e\fap 0$の時間的位置と,結節点$ \tilde{e}\fap 1 $が決定されるような公布範囲のメンバー$ e\in \mathcal{L}\fap(\tilde{u}\fap 2) $が存在することを要求する.
\begin{multline}
    \arg v \subseteq \classab{u:(\exists d)(\exists b)(\exists y)(\exists z)[
        \tilde{u}\fap 3 = \orp{d,b,y}\con{1}\mathcal{L}\fap(\tilde{u}\fap 0) = \orp{z,y}\con{1}z\subseteq b \con{1}\\
        (x)(c)(x\in z\con{1}\orp{c,x\fap 0}\in d\uphl\vec{\varphi}\case{1}{1}{0}
        (\exists e)(\exists t)(\exists s)(e\in \mathcal{L}\fap(\tilde{u}\fap 2)\con{1}\\
            \varrho\fap\orp{x,y} = \orp{t,s}\con{1}e\fap 0\cap t\neq\Lambda\con{1}\tilde{e}\fap 1\in s
        )
        )
    ]
    }.
\end{multline}
$ \tilde{u}\fap 3 = \orp{d,b,y}\con{1}\mathcal{L}\fap(\tilde{u}\fap 0) = \orp{z,y} $について,任意の$ x\in z $に相関する公布範囲$ \mathcal{L}\fap(\tilde{u}\fap 2) $のメンバーを要求するのではなく,$ \orp{c,x\fap 0}\in d\uphl\vec{\varphi} $である$x$に限定する理由はこうである.
制定/廃止範囲$ z $は,$ \arg v $または付帯条件$b$((5)により$ z\subseteq b $)によって,例えば,$ (x)(x\in z\case{1}{1}{1}\orp{u\fap 0,x\fap 0}\in\vec{\varphi}) $といった時間的限定が可能である.しかし,このような限定がない場合,因果系列$ \iter{\mathrm{E}}{2}\fap(u\diamond\trgl{v}) $の始点となる時間領域に完全に先行するような$ x\in z $が存在し得る.それでも,\ref{sssec:認定システムによる正規性論証}の正規性論証によって,$ x\in\enf{\epsilon}y $が認定され得る.いわゆる遡及的な制定や廃止と呼ばれる現象はこうして成立する.このことは$\orp{u,v}$が規制表明型でないケースでも同様である.
しかし,規制表明型である場合,この遡及的な$ x\in z $に相関する公布範囲のメンバー$e$を要求すると,$e\fap 0$より後でない$ c\in d $が存在しないため,(4)の条件を充たさないことになる.仮に(4)の時間的条件を外したとしても,遡及的な$ x\in z $に相関する$e$は,$ \cs{\epsilon}\eta $である有意な蓋然性を持たないと考えられる.

次の条件は,(5)と逆に,公布範囲の任意のメンバー$ e\in \mathcal{L}\fap(\tilde{u}\fap 2) $について,$e\fap 0$の時間的位置と$ \tilde{e}\fap 1 $が,それに相関して決定されるような$x\in (\mathcal{L}\resl\mathcal{L})\fap(\tilde{u}\fap 0)$が存在することを要求する.
\begin{multline}
    \arg v\subseteq\classab{u:
    (e)[e\in \mathcal{L}\fap(\tilde{u}\fap 2)\case{1}{1}{0}(\exists x)(\exists t)(\exists s)(
        x\in(\mathcal{L}\resl\mathcal{L})\fap(\tilde{u}\fap 0)\con{1}\\
        \varrho\fap\orp{x,(\mathcal{R}\resl\mathcal{L})\fap(\tilde{u}\fap 0)} = \orp{t,s}\con{1}e\fap 0\cap t\neq\Lambda\con{1}\tilde{e}\fap 1\in s
    )
    ]
    }.
\end{multline}

関数$ \varrho $の主要なパターンは次のような構成である.$\varrho\fap\orp{x,y} = \orp{t,s}$について,$ x\fap 0\subseteq\timex{\mathbb{R}}{4} $である場合,$ t = \union{(\check{\varphi}\fap (x\fap 0))} $.それ以外の場合(特に$ y $がN類型の場合),$t$は因果系列$ \iter{\mathrm{E}}{2}\fap(x\diamond\trgl{y}) $の始点と言えるような何らかの時間領域である.
他方,$s$については,
\[
    x\fap 0\subseteq \tilde{x}\fap 1\subseteq\timex{\mathbb{R}}{4}\case{1}{1}{1}s = \classab{\tilde{x}\fap 1}.
\]
ただし,$ y $が私法的規制類型の場合で,$s$が(法人を含む)当事者(権利者と義務者)の集合となるケースが考えられる\footnote{例えば,相殺による対立債権の廃止のケースが考えられる.}.これ以外の場合,例えば$ \tilde{x}\fap 1 $が法人なら$s$はその業務執行機関の集合である.また,特に$y$がN類型の場合,$ s $は$ t\cap a\neq \Lambda $であり,一定の条件(国籍要件等)を充たす$a$の集合等であり得る.
さらに,\kagi{$ \varrho $}の位置に来る抽象体が指示する関数$g$は,$ v\in\mathrm{Reg} $が何であるかに応じて,上記の主要パターンと異なる構成を持ち得る.例えば,$ \orp{u,v} $が契約の承諾権限の場合,多分,$ \trgl{g} = \orp{t',s'}\con{1}t' = \union{[\check{\varphi}\fap\union{(\mathcal{L}\fap(\tilde{u}\fap 3))}]} $.また,$s'$は$\orp{u,v}$を制定する申込権限の適用条件によって,その申込者の単一クラスに限定されると考えられる.

なお,これまでに構成された規制表明型の条件(1)〜(6)は,$ \orp{u,v} $が制定規制である場合の条件であるが,(1)(4)の\kagi{$ b\stackrel{\epsilon}{\mathrm{er}}y $}を\kagi{$ b\stackrel{\epsilon}{\mathrm{ar}}y $}に交換すれば,$\orp{u,v}$が廃止規制である場合の条件になる.

\subsubsection{付帯条件}
\label{sssec:付帯条件}

さて,先程$ \varrho $を例示する際に私法的規制類型や契約に言及したが,これらに関しては規制ルールの付帯条件の機能に関連して,もう少し言うべきことがある.
再び規制表明型$ \orp{u,v}\in\mathrm{REG}^\epsilon $について,
\[
    \tilde{u}\fap 3 = \orp{d,b,y}\con{1}\mathcal{L}\fap(\tilde{u}\fap 0) = \orp{z,y}
\]
であるとしよう.D \ref{df:制定類型}またはD \ref{df:廃止類型}によって,$ z\subseteq \app{\epsilon} y $である.また,$ z $は$ \arg v $によって更なる制限を受けるかもしれない.
ここまでは高次類型一般に当てはまる状況であるが,規制表明型の場合,(5)によりさらに$ z\subseteq b $という条件が追加される.
付帯条件$b$は,$\app{\epsilon} y$や$\app{\epsilon} v$に還元できない条件によって,制定/廃止範囲$z$をさらに限定する.
それが適用条件に還元できない理由はこうである.構造解析を行う際,単一クラスを言及するような開放文をなるべく使用しない,という方針を認めることができる.もちろん,議会の立法権などを特定化するには,特定の国の議会の単一クラスを言及する開放文を使用するであろうから,この方針は絶対的ではない.しかし,この大まかな方針は,構造解析を体系的に行うことを促進するほか,適用条件から以下の意味での個別的条件を可及的に除外して,制御構造や執行可能性の構造的条件としての内実を保持する機能を持つ.
すなわち,$ \arg\delta\subseteq w $が個別的な条件であるのは,それが特定の対象を任意の$ x\in \arg\delta $に共通の内部構造とするような条件であるとき,言い換えれば,$ \trcl x $の所定の要素を規則的に取り出す関数$ \gamma $について,$ w = \classab{x:\gamma\fap x = a} $となるときである.例えば,$ w = \classab{x:\tilde{x}\fap 1 = a} $.このような$w$に言及するには$a$の単一クラスを言及する開放文を使用するであろうから,上記の方針によって適用条件から除外される.
適用条件が個別的であればあるほど,それの反事実的仮定は空想的になり(現実のモデルから大きく乖離したモデルにおける真理を問題とすることになり),制御構造や執行可能性の構造的条件としての内実は希薄化していくと考えられる\footnote{他の条件が同じであれば,現実には人間ではない時空領域$s$について,$s$が人であることを肯定するモデルより,$s$が特定の人物$a$であることを肯定するモデルの方が現実から遠い.}.
以上のような方針に基づく構造解析で特定化される適用条件から除外される条件を,付帯条件が収容する.それは多分,通常それを言及する開放文に単一クラスを言及する開放文が出現するようなクラスである.

付帯条件は,類型的な引渡債務や金銭債務等の執行可能性を特定の当事者や時間について構築する,という私法規制の制定場面でよく見られるが,それに限られない.例えば,法律の施行日を2025-03-01に設定するのは付帯条件である.例えば,
\[
    v\in\mathrm{Es}^\epsilon\con{1}b\subseteq \classab{x:\orp{\text{2025-03-01},\union{(\check{\varphi}\fap(x\fap 0))}}\in\vec{\varphi}}.
\]
ここで「2025-03-01」は特定の時区間を指示する記述(\kagi{$ (\imath x)Fx $}の\kagi{$ F $}に単一クラスを言及する開放文を代入)を省略したものと考えられる\footnote{なお,法律の施行日を政令に委任するケースは,特定化された規制類型集合に関する制定権限を政令に与えるケースである.そして,政令の付帯条件において施行日が設定される.}.
私法領域での始期の設定もこれと同様である.逆に終期は例えば,
\[
    v\in\mathrm{Es}^\epsilon\con{1}b\subseteq \classab{x:\orp{\union{(\check{\varphi}\fap(x\fap 0))},\text{2025-03-01}}\in\vec{\varphi}}.
\]
これに対して,いわゆる停止条件と解除条件は\kagi{$ (\imath x)Fx $}への代入結果の指示対象が存在しない可能性があるケースである\footnote{いわゆる条件と期限は,規制表明の付帯条件だけでなく,規制表明型ではない高次類型の適用条件によっても規定され得る.}.
また,$ v\in\mathrm{Ab}^\epsilon $の始期と終期の設定,つまり,ある特定の時点$ t $以降について廃止するか,または,$t$以前を廃止することも想定できる.しかし,既に\ref{ssec:正規集合と規制体系}において示唆されているが,同一の規制要素に対して制定関係を持つ規制が競合し得る場合には,微妙な問題が生じる.例えば,
\[
    \orp{u',v'},\orp{u'',v''}\in\mathrm{est}^\epsilon\img\classab{\orp{x,y}}
\]
について,$\orp{x,y}$が100万円の金銭債務の規制要素で,$ \orp{u',v'} $が売買契約の代金債務を制定する承諾権限,$ \orp{u'',v''} $が(不法行為や債務不履行等の)利益侵害によって損害賠償義務を制定するS規制,であるようなケースを想定できる.
さらに極端なケースとして,同一当事者間で金額を含めて同一の条件の金銭消費貸借債務を2口制定する場合,$ v'=v'' $であり,$ u' $と$ u'' $も,以下のように人工的なやり方で区別された付帯条件においてのみ異なる.
\[
   (\mathcal{L}\resl\mathcal{R})\fap(\tildel{(u')}\fap 3) = b\cup\classab{1}\con{1}
   (\mathcal{L}\resl\mathcal{R})\fap(\tildel{(u'')}\fap 3) = b\cup\classab{2}.
\]
いずれにせよ,この状況で$ \mathcal{L}\fap(\tilde{u}\fap 0) = \orp{z,y}\con{1}x\in z $なる$ \orp{u,v}\in\mathrm{REG}^\epsilon\uphr\mathrm{Ab}^\epsilon $を正規化するような体系は工学的妥当性に問題を生じる.
他方,
\[
    \mathcal{L}\fap(\tilde{u}\fap 0) = \orp{w,v'}\con{1}u'\in w\case{2}{1}{1}\mathcal{L}\fap(\tilde{u}\fap 0) = \orp{w,v''}\con{1}u''\in w
\]
であるような$ \orp{u,v}\in\mathrm{REG}^\epsilon\uphr\mathrm{Ab}^\epsilon $は正規化されてよい.そして,$ u\in\exe{\epsilon} v $によって,$ t $以降の$ x'\in\app{\epsilon}y $を含む制定範囲を持ち,かつ,
$ \mathrm{E}\fap k = \mathrm{E}\fap u' $である任意の制定規制$ \orp{k,v'} $を廃止する.すなわち,($t$の特定化を含む)付帯条件$ (\mathcal{L}\resl\mathcal{R})\fap(\tilde{u}\fap 3) $により,
\[
   w = \classab{k:
        \mathrm{E}\fap k = \mathrm{E}\fap u'\con{1}
        (\exists z)(\exists x)(
            \tilde{k}\fap 0 = \orp{z,y}\con{1}x\in z\con{1}\orp{t,\union{(\check{\varphi}\fap(x\fap 0))}}\in\vec{\varphi}
        )
   }\cap\app{\epsilon}v'.
\]
あるいは,同様の条件を充たす任意の$ \orp{k,v''} $を廃止する.
なお,前提として,
\[
\arg v'\subseteq\classab{u':
    (\exists z)(\exists y)(\exists t)
    [
        \mathcal{L}\fap(\tildel{(u')}\fap 0) = \orp{z,y}\con{1}
        t\in\zeta\fap y\con{1}
        z\subseteq\classab{x:\union{(\check{\varphi}\fap(x\fap 0))}\subseteq t}
    ]
}.
\]
ただし,型文字\kagi{$ \zeta $}に代入されるクラス抽象体は,$ u'\in\arg v' $について,被制定類型$ (\mathcal{R}\resl\mathcal{L})\fap(\tildel{(u')}\fap 0) $に応じて,ある時点以降の時間領域を一定の長さを持つ時区間ごとに分割した集合を与えるような関数を指示するものとする.例えば,1873-01-01から1日ごとに区切られた時区間の集合.関数$\zeta$によって1個の制定範囲$ (\mathcal{L}\resl\mathcal{L})\fap(\tildel{(u')}\fap 0) $の時間的な幅が決定される.なお,始期や終期等はこれとは別に,適用条件$ \app{\epsilon}v' $か,($\orp{u',v'}$が規制表明型である場合の)付帯条件$ (\mathcal{L}\resl\mathcal{R})\fap(\tildel{(u')}\fap 3) $によって定められる.
いずれにせよ,上記の廃止によって,$ t $より前の$x''\in\arg y$からなる制定範囲を持ち,かつ,$ \mathrm{E}\fap j = \mathrm{E}\fap u' $である任意の制定規制$ \orp{j,v'} $だけが残存することになる.

さて,規制類型が共通で因果系列と結節点も同一であるが,異なる制定範囲を持つ制定規制(要素)の集合について,その一部を廃止するケースは他にもある.上記の例では,当該集合を制定範囲のメンバーが属する時区間によって分割した.次の例では,当該集合を,制定範囲のメンバーの利益侵害の量的指標によって分割する.
この点,利益侵害の量的指標とは,$ x\in\exe{\epsilon} y $が適用条件下での利益侵害の惹起であるような私法規制$ \orp{x,y} $について,
$ i\in \trcl{(\tilde{x}\fap 2)} $なる$i$であって,利益侵害の量を表示する機能を持つものである.
例えば,$y$が金銭債務の規制類型で,以下のように構造解析されるとしよう.
\begin{gather*}
    \tilde{y}\fap 2 = \barl{\classab{\orp{i,e,a}:\text{$e$は$a$が$i$円の金額占有を取得する出来事}}}\cap\mser{(\arg y)}{2},\\
    \tilde{y}\fap 3 = \barl{\classab{\orp{i,e,a}:\text{$e$は$a$が$i$円の金額占有を喪失する出来事}}}\cap\mser{(\arg y)}{3},\\
    \arg y\subseteq \classab{x:
        \iter{\mathcal{R}}{2}\fap(\tilde{x}\fap 3) = \tilde{x}\fap 1\con{1}
        \mathcal{L}\fap(\tilde{x}\fap 2) = \mathcal{L}\fap(\tilde{x}\fap 3)\in\mathbb{N}
    }.
\end{gather*}
すると,$ x\in\arg y $については,金額$ \mathcal{L}\fap(\tilde{x}\fap 2) $が利益侵害の量的指標である.今,型文字\kagi{$ \chi $}を使用して,
\[
    \func l\con{1}l\subseteq\classab{\orp{z,y}:
    \func z \con{1}\breve{z}\img\univ = \arg y\con{1}y\in\mathrm{Reg}
    }
\]
なる$l$を指示するクラス抽象体の位置を表わす.ただし,$ y\in\arg l $について,$ l\fap y $は$\arg y$のメンバーの利益侵害の量的指標を取り出す関数とする.
すると今の例では,$ (\chi\fap y)\fap x = \mathcal{L}\fap(\tilde{x}\fap 2) $.

次に,再び制定規制$\orp{u',v'}$について,
\[
    \arg v'\subseteq\classab{u':
    (\exists z)(\exists y)(\exists i)
    [
        \mathcal{L}\fap(\tildel{(u')}\fap 0) = \orp{z,y}\con{1}
        i\in\mathbb{N}\con{1}
        z\subseteq\classab{x:(\chi\fap y)\fap x = i}
    ]
}
\]
であるとしよう.これにより,任意の$ u'\in\arg v' $について,制定範囲$ (\mathcal{L}\resl\mathcal{L})\fap(\tildel{(u')}\fap 0) $は,そのメンバーに共通する利益侵害の量的指標$ i $を持つ.
なお,最大指標$n\in\mathbb{N}$はこれとは別に,$ \app{\epsilon}v' $か,($\orp{u',v'}$が規制表明型である場合の)$ (\mathcal{L}\resl\mathcal{R})\fap(\tildel{(u')}\fap 3) $によって定められる.例えば,
\[
    (\mathcal{L}\resl\mathcal{R})\fap(\tildel{(u')}\fap 3)\subseteq\classab{x:(\chi\fap y)\fap x \leq n}.
\]
$ u\in\exe{\epsilon} v $によって,その量的指標が$ i $($u$の付帯条件で特定化される)より大きい$ x'\in\arg y $を含む制定範囲を持ち,かつ,
$ \mathrm{E}\fap k = \mathrm{E}\fap u' $である任意の制定規制$ \orp{k,v'} $が廃止されると仮定する.ここで,
\[
    n = 1000000\con{1}i = 700000
\]
とすると,これは100万円の金銭債務の一部30万円分が廃止されることを意味する.債務の一部履行や一部消滅と言われる現象はこのようにして起きる.

\subsubsection{当事者}
\label{sssec:当事者}

最後に,制定規制の付帯条件によって規制の当事者が限定されるパターンと,それに関連する契約と呼ばれる構造について考える.
まず,規制要素$ \orp{x,\delta} $の法益主体という概念を規定する.
そのために,準拠領域とそれが制御可能性を充たす規制類型との関係$\beta$と,独立変項の内部構造を取り出す関数$\gamma$を考える.つまり,
\begin{gather*}
    \beta\neq\Lambda\con{1}(g)(h)(\orp{g,h}\in\beta\case{1}{1}{1}\orp{g,h}\in(\cty{\epsilon}h)\times\mathrm{Reg}),\\
    \gamma\neq\Lambda\con{1}(a)(a\in\arg \gamma \case{1}{1}{1}\gamma\fap a\in \trcl a).
\end{gather*}
すると,任意の$ a\in\arg \delta $について,$ \tilde{a}\fap 2\in\tildel{\trgl{\delta}}\fap 2 $であることが$ \gamma\fap(\tilde{a}\fap 2)\neq\Lambda $に対する利益侵害であるとき,すなわち,
\begin{multline*}
    \arg \delta \subseteq \classab{a:
        (g)(h)(\orp{g,h}\in\beta\case{1}{1}{1}
            \tilde{g}\fap 1 = \gamma\fap(\tilde{a}\fap 2)\neq\Lambda\con{1}\tilde{g}\fap 0 = \tilde{a}\fap 2\con{1}\\
            \tildel{\trgl{h}}\fap 0 = \tildel{\trgl{\delta}}\fap 2\con{1}\tildel{\trgl{h}}\fap 1 = \barl{(\trgl{h}\fap 0)}\cap\msec{(\arg h)}{0}
        )
    } \tag{i}
\end{multline*}
であるとき,かつそのときに限り,$ \gamma\fap(\tilde{x}\fap 2) $は$ \orp{x,\delta} $の法益主体である.
(i)は,$ \tilde{a}\fap 2\in\tildel{\trgl{\delta}}\fap 2 $であることが,$ \gamma\fap(\tilde{a}\fap 2) $のある種の行動パターン($\beta$の要素)に対して抑制条件になっていることを表現している.(i)が真となるような\kagi{$ \beta $}と\kagi{$ \gamma $}への代入は,\kagi{$ \delta $}への代入によって異なる.
そして,規制要素$ \orp{x,\delta} $の結節点$ \tilde{x}\fap 1 $と法益主体$ \gamma\fap(\tilde{x}\fap 2) $の組を,$ \orp{x,\delta} $の当事者と言う.

さて,ある規制体系において契約と呼ばれる構造が成り立つには,以下のような規制表明型$v\in\mathrm{Es}^\epsilon $と規制表明型$t\in\mathrm{Es}^\epsilon\cup\mathrm{Ab}^\epsilon$が必要である.いずれもP類型である.まず,任意の$u\in \arg v$について,被制定類型$ (\mathcal{R}\resl\mathcal{L})\fap(\tilde{u}\fap 0) $は$t$であり,制定範囲$ (\mathcal{L}\resl\mathcal{L})\fap(\tilde{u}\fap 0) $の任意のメンバーの公布範囲は,$\tilde{u}\fap 1$を結節点に持つ.すなわち,
\begin{align*}
    \arg v\subseteq\classab{u:
    (\mathcal{R}\resl\mathcal{L})\fap(\tilde{u}\fap 0) = t\con{1}
    (\mathcal{L}\resl\mathcal{L})\fap(\tilde{u}\fap 0)\subseteq \classab{s:
                \mathcal{L}\fap(\tilde{s}\fap 2) \subseteq\classab{e:\tilde{e}\fap 1 = \tilde{u}\fap 1}
                }
    }.
\end{align*}
次に,付帯条件$ (\mathcal{L}\resl\mathcal{R})\fap(\tilde{u}\fap 3) $によって,すべての$s\in (\mathcal{L}\resl\mathcal{L})\fap(\tilde{u}\fap 0)$に共通する内部構造が特定される\footnote{
    $(u)[u\in\arg v\case{1}{1}{1}(\mathcal{L}\resl\mathcal{L})\fap(\tilde{u}\fap 0)\subseteq(\mathcal{L}\resl\mathcal{R})\fap(\tilde{u}\fap 3)]$であるから(\ref{sssec:規制表明}),結局,$\union{\classab{(\mathcal{L}\resl\mathcal{L})\fap(\tilde{e}\fap 0):e\in\arg v\con{1}\mathrm{E}\fap e = \mathrm{E}\fap u}}$のメンバーに共通する内部構造である.
}.
共通化される内部構造は,結節点$ \tilde{s}\fap 1 $,付帯条件と被制定類型の組$ \mathcal{R}\fap(\tilde{s}\fap 3) $,制定範囲$ (\mathcal{L}\resl\mathcal{L})\fap(\tilde{s}\fap 0) $のメンバーの結節点と法益主体である.
すなわち,$\gamma$に相対的な$\alpha$の特定化系列$ \alpha\bullet\gamma $を,
\begin{multline*}
    \alpha\bullet\gamma = (\imath k)[
        k\in\mathrm{Seq}\con{1}\arg k = 5\con{1}
        \alpha\subseteq\classab{s:\tilde{s}\fap 1 = k\fap 0\con{1}
        \mathcal{R}\fap(\tilde{s}\fap 3) = \orp{k\fap 1,k\fap 2}\con{1}\\
        (\mathcal{L}\resl\mathcal{L})\fap(\tilde{s}\fap 0)\subseteq \classab{x:k\fap 3= \tilde{x}\fap 1\con{1} k\fap 4 = \gamma\fap(\tilde{x}\fap 2)\neq\Lambda}
        }
    ]
\end{multline*}
と定義する.$ (\alpha\bullet\gamma)\fap 2 $は(i)が成り立つような規制類型である.あるいは,準制定関係$\sigma$と結節点を$\mu$に制限された$\sigma$を,
\begin{gather*}
    \sigma = \classab{\orp{\orp{u,v},\orp{x,y}}:
    \orp{u,v}\in\app{\epsilon} v\times \mathrm{Es}^\epsilon\con{1}
    x\in (\mathcal{L}\resl\mathcal{L})\fap(\tilde{u}\fap 0)\con{1}y = (\mathcal{R}\resl\mathcal{L})\fap(\tilde{u}\fap 0)
    },\\
    \mu \ddagger\sigma = \classab{\orp{u,v}:\tilde{u}\fap 1 \in\mu}\uphl\sigma
\end{gather*}
と規定する.そして,(i)が成り立つ規制類型$y$に関する規制要素から$\mu \ddagger\sigma$を反復して遡及可能な規制要素にまで拡張して,
\begin{multline*}
    \alpha\bullet\gamma = (\imath k)[
        k\in\mathrm{Seq}\con{1}\arg k = 5\con{1}
        \alpha\subseteq\classab{s:\tilde{s}\fap 1 = k\fap 0\con{1}
        \mathcal{R}\fap(\tilde{s}\fap 3) = \orp{k\fap 1,k\fap 2}\con{1}\\
        (\mathcal{L}\resl\mathcal{L})\fap(\tilde{s}\fap 0)\subseteq\classab{x':
            (\exists x)(\exists y)(
            \orp{x',k\fap 2}\in\ance{(\classab{k\fap 3,k\fap 4}\ddagger\sigma)}\img\classab{\orp{x,y}}\con{1}\\
            k\fap 3= \tilde{x}\fap 1\con{1} k\fap 4 = \gamma\fap(\tilde{x}\fap 2)\neq\Lambda
            )
        }
        }
    ]
\end{multline*}
と定義してもよい\footnote{
    例えば,契約によって直接債務を制定するのではなく,債務を制定する予約権を制定するケース等が想定されている.
}.いずれにせよ,
\[
   \arg v \subseteq \classab{u:(\exists b)(
    b = (\mathcal{L}\resl\mathcal{R})\fap(\tilde{u}\fap 3)\con{1}
    b\bullet\gamma\neq \Lambda
   )}.
\]

以上の条件を充たす$v,t$について,$ \orp{u,v}\in\mathrm{REG}^\epsilon $を申込権限,$ \orp{s,t}\in \brevel{(\mathrm{est}^\epsilon)}\img\classab{\orp{u,v}}\cap\mathrm{REG}^\epsilon $を承諾権限と言う.特定の規制体系においてこれらを実装する方法については,例えば以下のように考えられる.
すなわち,立法規制の適用条件か付帯条件によって,$\enf{\epsilon}v$が構築される準拠領域$u\in\app{\epsilon} v$が,下記(1)(2)のいずれかを充たすものに限定される.ただし,$ k = (\mathcal{L}\resl\mathcal{R})\fap(\tilde{u}\fap 3)\bullet\gamma $とする.
\setcounter{equation}{0}
\begin{equation}
    \tilde{u}\fap 1\in\classab{k\fap 3,k\fap 4}\con{1}k\fap 0\in\classab{k\fap 3,k\fap 4}\cap\barl{\classab{\tilde{u}\fap 1}}.
\end{equation}
\begin{multline}
    (\exists u')(\exists k')(\exists u'')(\exists k'')[
        \orp{u',v},\orp{u'',v}\in\mathrm{REG}^\epsilon\con{1}
        \check{\varphi}\fap(u\fap 0)\cap\check{\varphi}\fap(u'\fap 0)\neq\Lambda\con{1}
        \tilde{u}\fap 1 = \tildel{(u')}\fap 1 \con{1}\\
        k\fap 0 = \tildel{(u'')}\fap 1\notin\classab{k\fap 3,k\fap4}\con{1}
        k' = (\mathcal{L}\resl\mathcal{R})\fap(\tildel{(u')}\fap 3)\bullet\gamma\con{1}k'' = (\mathcal{L}\resl\mathcal{R})\fap(\tildel{(u'')}\fap 3)\bullet\gamma\con{1}\\
        k'\fap 0 ,k''\fap 0 \in \classab{k\fap 3,k\fap4}\con{1}
        k\uphr\bar{1} = k'\uphr \bar{1} = k''\uphr\bar{1}
    ].
\end{multline}

(1)を充たす$u$について,$ \orp{u,v}\in\mathrm{REG}^\epsilon\con{1}k = (\mathcal{L}\resl\mathcal{R})\fap(\tilde{u}\fap 3)\bullet\gamma $と仮定する.すると契約の申込$ u\in \exe{\epsilon} v $によって,
$ (\mathcal{L}\resl\mathcal{L})\fap(\tilde{u}\fap 0)\subseteq\enf{\epsilon}t $となるレベル$\mathcal{R}\fap(\tilde{u}\fap 0)$の蓋然性が成立する.
今,ある$ s\in(\mathcal{L}\resl\mathcal{L})\fap(\tilde{u}\fap 0) $について,$ \orp{s,t}\in\mathrm{REG}^\epsilon\uphr\mathrm{Es}^\epsilon $と仮定する.すると契約の承諾$ s\in\exe{\epsilon} t $によって,$ \tilde{s}\fap 0 = \orp{\orp{z,y},m} $について,$z\subseteq \enf{\epsilon}y$となるレベル$m$の蓋然性が成立する.言い換えれば,$ \tilde{u}\fap 1,\tilde{s}\fap 1\in \classab{k\fap 3,k\fap 4} $の合意によって,そのような蓋然性が生じる.

これに対して,(2)は,申込権限の代理権$ \orp{u'',v} $について,$ \tildel{(u'')}\fap 1 $を承諾権限の結節点とするような申込権限$\orp{u,v}$を制定するものである.$\orp{u,v}$の結節点は,既存の申込権限$\orp{u',v}$のそれと同一である.代理権自体は(1)(2)とは別に,下記(3)の条件を充たす規制表明型$ p\in\mathrm{Es}^\epsilon $について,ある$ \orp{e,p}\in\mathrm{REG}^\epsilon $が規制体系に含まれる場合に,その実行によって制定される.
\begin{multline}
    \arg p\subseteq \classab{e:
    (\exists u)(\exists u')[
        \orp{u,v}\in\mathrm{REG}^\epsilon\con{1}u'\in (\mathcal{L}\resl\mathcal{L})\fap(\tilde{e}\fap 0)\cap\app{\epsilon} v\con{1}(\mathcal{R}\resl\mathcal{L})\fap(\tilde{e}\fap 0) = v\con{1}\\
        \tilde{u}\fap 1\neq \tildel{(u')}\fap 1\con{1}
        \tilde{u}\uphr\classab{0,2,3} = \tildel{(u')}\uphr\classab{0,2,3}\con{1}
        \check{\varphi}\fap(u\fap 0)\cap\check{\varphi}\fap(u'\fap 0)\neq\Lambda
    ]
    }.
\end{multline}
承諾権限の代理権についても同様である.契約に関する基本的な規制は,当該規制体系において,(1)〜(3)により再帰的に実装される.


\subsection{存在論}
\label{ssec:存在論}

規制類型のメンバーの内部構造には再び規制類型が含まれ得る\footnote{なお,他の規制類型の制御構造を惹起することが構成要件であったり,他の規制類型の構成要件実現を阻止することを抑制するS規制を想定できるから,その内部構造に他の規制類型を含むような規制類型は高次類型だけではない.}.つまり,$y\in\mathrm{Reg}$について,
\[
    (\exists z)(z\in y\con{1}y' \in \trcl z\cap \mathrm{Reg}).
\]
さらに,
\begin{gather*}
    (\exists z')(z'\in y'\con{1}y'' \in \trcl z'\cap \mathrm{Reg}),\\
    (\exists z'')(z''\in y''\con{1}y''' \in \trcl z''\cap \mathrm{Reg})
\end{gather*}
と階層が続く可能性がある.また,D \ref{df:規制類型}による限定は非常に乏しいため,規制類型の集合は必要以上に大きくなりがちである.\ref{ssec:集合論の体系}の体系では,$\univ$や$\bar{x}$など大きすぎるクラスは存在できない.以上のことから,規制類型の存在論を,その内部構造に含まれる規制類型の階層構造を踏まえて,体系的に明確化する必要が生じる.
以下では,正規集合$\varSigma^{\epsilon}\alpha$について,制定階層とは異なる階層構造を定義して,D \ref{df:階層構造}の無限系列$ \mathcal{W} $と連動させる形で正規集合内の規制類型の存在論を確定する.

実際的な規制体系におけるどの規制類型も,物理的対象及び古典的な数学的対象とこれらのクラス,そのようなクラスからなるクラス等々によって構成されているであろうから,事実上は$ \mathcal{W}\fap 2 $の範囲内に収まると考えられる.すなわち,内部構造に規制類型を含まない規制類型のうち,実際上必要なものはどれもある$i\in\mathbb{N}$について,$ \iter{\mathfrak{P}}{i}\fap(\mathcal{W}\fap 1) $のメンバーになると期待できる.すると,そのような規制類型を内部構造に持つ規制類型も,$i\in j\in\mathbb{N}$について,$ \iter{\mathfrak{P}}{j}\fap(\mathcal{W}\fap 1) $のメンバーになると期待できる.以下同様にして,その体系内の全ての規制類型は$ \mathcal{W}\fap 2 $のメンバーになると考えられる.
しかし,規制体系内の全ての規制類型について,それが$ \mathcal{W}\fap 1,\iter{\mathfrak{P}}{1}\fap(\mathcal{W}\fap 1),\iter{\mathfrak{P}}{2}\fap(\mathcal{W}\fap 1),\dots $のどの段階に属するかを体系的に確定することはできない.それゆえ,個々の規制類型の構造解析において,上記の$i$や$j$を個別的に特定する負担が生じる.

このような負担を回避して,規制類型の存在論を体系的に確定できるようにするには,$\mathcal{W}\fap 2$を超えて階層を上昇させる必要がある.
この点に関して,$y\in\mathrm{Reg}$と$0\in i\in\mathbb{N}$について,$\arg y\subseteq \mathcal{W}\fap i$であると仮定しよう.すると,$\trgl{y}$の系列要素はそれぞれが$\mathcal{W}\fap i$の部分クラスとなるような対象であることが想定できる.すなわち,$ \trgl{y}\img\univ\subseteq \mathcal{P}(\mathcal{W}\fap i) $.すると$\trgl{y}$自体は,$\mathcal{W}\fap i$の部分クラスと自然数との順序対の有限クラスであるから,ある$n\in\mathbb{N}$について,$ \trgl{y}\in\iter{\mathfrak{P}}{n}\fap(\mathcal{W}\fap i) $.したがって,$\trgl{y}\in \mathcal{W}\fap(\mathrm{S}\fap i)$.そして,$ y $のメンバーは,その1個の$\trgl{y}$と$\mathcal{W}\fap i$のメンバーとの順序対になるから,$ n\subseteq m\in\mathbb{N} $について,$ y\subseteq \iter{\mathfrak{P}}{m}\fap(\mathcal{W}\fap i) $.それゆえ,$y\in \mathcal{W}\fap(\mathrm{S}\fap i)$.
ところで,\ref{ssec:集合論の体系}の体系で,次の(1)(2)が証明可能である.
\setcounter{equation}{0}
\begin{gather}
    (z)(x)(z\subseteq\mathcal{P}(z)\con{1}x\subseteq z\case{1}{1}{1}\trcl x\subseteq z),\\
    (n)(n\in\mathbb{N}\case{1}{1}{1}\mathcal{W}\fap n\subseteq \mathcal{P}(\mathcal{W}\fap n)).
\end{gather}
すると,仮定と(2)により,
\begin{align*}
    x\in\arg y & \case{1}{1}{1}x\in \mathcal{W}\fap i\\
    &\:\,\case{1}{0}{1}x\subseteq \mathcal{W}\fap i.
\end{align*}
また既述の通り,$ x\in \trgl{y}\img\univ\case{1}{1}{1}x\subseteq \mathcal{W}\fap i $.したがって,(1)により,
\begin{align*}
    x\in\arg y\case{2}{1}{1}x\in \trgl{y}\img\univ & \case{1}{2}{1}x\subseteq\mathcal{W}\fap i\\
    &\:\,\case{1}{0}{1}\trcl{x}\subseteq  \mathcal{W}\fap i.
\end{align*}
それゆえ,$v\in\trcl x$である規制類型$v$について,$v\in\mathcal{W}\fap i$.上述したところから,$ v\subseteq \mathcal{W}\fap i\times \mathcal{W}\fap(\breve{\mathrm{S}}\fap i) $ならば,$v\in\mathcal{W}\fap i $となる.

このような考察を踏まえて,$\mathcal{W}$を拡張した規制類型の階層構造を導入する.
\begin{df}
\label{df:規制類型の階層}
\kagi{$
    \mathcal{X}
$}は\kagi{$
    (\imath w)[
        \func{w}\con{1}\arg w = \mathbb{N}\con{1}
        (n)(n\in\arg{w}\case{1}{1}{1} \\\hfill 
        w\fap n =(\mathcal{W}\fap(\mathrm{S}\fap n)\times \mathcal{W}\fap n)\cap\classab{x:\trcl x \cap \mathrm{Reg} \subseteq \mathcal{P}(w\fap(\breve{S}\fap n))}
        )
    ]
$}を表わす.
\end{df}
\noindent D \ref{df:階層構造}により,$ \mathcal{W}\fap\Lambda = \Lambda $であるから,$ \mathcal{X}\fap\Lambda = \Lambda $である.
次に,正規集合$\varSigma^{\epsilon}\alpha$について,内部階層$ \mathrm{inh}_\epsilon\alpha $を定義して,それを$\mathcal{X}$の階層と連動させる.
まず,$y$が$v$のメンバーの内部構造に含まれる$\orp{v,y}\in \timex{(\mathrm{Reg})}{2}$のクラスである内部関係を,
\begin{df}
\label{df:内部関係}
\kagi{$
    \mathrm{in}
$}は\kagi{$
    \classab{\orp{v,y}:(\exists z)(z\in v\con{1}y\in\trcl z)}\cap\timex{(\mathrm{Reg})}{2}
$}を表わす
\end{df}
\noindent と規定する.明らかに$ (\mathrm{in}\resl\mathrm{in})\subseteq \mathrm{in} $.内部階層は反復の最大値で定義されるため結局同じことになるが,推移性を排除した次の関係(直接の内部関係)を使う方が分かりやすい.
\begin{df}
\label{df:直接の内部関係}
\kagi{$
    \mathrm{int}
$}は\kagi{$
    \mathrm{in}\cap\barl{(\mathrm{in}\resl\mathrm{in})}
$}を表わす.
\end{df}

\noindent 制定関係または廃止関係に立つ規制$ \orp{u,v} $と$\orp{x,y}$について,$ \orp{v,y} $は内部関係に属する.つまり,
\[
    \classab{\orp{v,y}:(\exists u)(\exists x)(
        \orp{\orp{u,v},\orp{x,y}}\in\mathrm{est}^\epsilon\cup\mathrm{abo}^\epsilon
    )}\subseteq\mathrm{in}.
\]
次の定義は,正規集合$\varSigma^{\epsilon}\alpha$の包括類型の集合$ \mathrm{anc}_\epsilon\alpha $を導入する.
\begin{df}
\label{df:包括類型}
\kagi{$
    \mathrm{anc}_\epsilon\alpha
$}は\kagi{$
    \classab{v:v\in\brevel{(\varSigma^{\epsilon}\alpha)}\img\univ\con{1}v\notin\brevel{(\mathrm{in})}\img(\brevel{(\varSigma^{\epsilon}\alpha)}\img\univ)}
$}を表わす.
\end{df}
\noindent 包括類型は正規集合内の規制類型であり,他の正規集合内のどの規制類型もそれに対して内部関係を持たないような規制類型である.すると,
\begin{align*}
    \orp{x,y}\in\bar{\alpha}\cap\varSigma^{\epsilon}\alpha &\case{1}{1}{0}(\exists z)(\orp{z,\orp{x,y}}\in(\varSigma^{\epsilon}\alpha)\uphl\mathrm{est}^\epsilon )\\
    &\:\,\case{1}{0}{1} y\notin \mathrm{anc}_\epsilon\alpha.
\end{align*}
つまり,$\mathrm{anc}_\epsilon\alpha\subseteq\breve{\alpha}\img\univ$.包括類型は全て始祖の規制類型である.

次に,内部階層は,正規集合内の規制類型とは限らない$y\in\mathrm{Reg}$に対して,それが直接の内部関係を最大$n$回反復して,$\varSigma^{\epsilon}\alpha$の包括類型に到達するような$n$を与える関数である.すなわち,
\begin{df}
\label{df:内部階層}
\kagi{$
    \mathrm{inh}_\epsilon\alpha
$}は\kagi{$
    \classab{
        \orp{n,y}:n = \union{\classab{i:i\in\mathbb{N}\con{1}(\exists v)(\orp{v,y}\in\mathrm{anc}_\epsilon\alpha\uphl\iter{\mathrm{int}}{i})
        }}
    }
$}を表わす.
\end{df}

\noindent 自然数の大小関係は$\classab{\orp{x,y}:x\in y}$または$\classab{\orp{x,y}:x\subset y}$であるから,任意の$ z\subseteq\mathbb{N} $について,$z$の最小元は,$x\subseteq\intersect{z}$である$x\in z$,すなわち,$\intersect{z}\in z$であるときの$ \intersect{z} $である.他方,$z$の最大元は,$ \union{z}\subseteq x $である$x\in z$,すなわち,$\union{z}\in z$であるときの$ \union{z} $である.この点,仮に$ m\in\mathbb{N} $が存在して,
\[
    m = \union{((\mathrm{inh}_\epsilon\alpha)\img\univ)}\in (\mathrm{inh}_\epsilon\alpha)\img\univ
\]
ならば\footnote{実際の物理的システムとしての規制体系で,内部階層が無限に上昇していくようなものは想像できない.},$ \tildel{(\mathcal{X}\uphr (\iter{\mathrm{S}}{3}\fap m))} $によって,$\iter{\mathrm{S}}{2}\fap m$から始めて,順番に内部階層を下っていくことで,個々の規制類型に特定化された階層を割り当てることができる\footnote{
    $ y\in\mathrm{Reg} $について,$ \trgl{y}\fap i = \bar{z}\cap\mathcal{W}\fap n $とすると,$ \trgl{y}\fap i $の中には,$y$に対して直接の内部関係を持つ規制類型が無限定に入り込む.その結果,事実上内部階層を辿ることができなくなるように思われる.高次類型の修正条件なども同様の問題を持つ.そのため,$\mathcal{W}$のある段階ではなく,$ \msec{(\arg y)}{i} $等で制限する方がよい.
}.すなわち,
\[
    y\in\brevel{(\varSigma^{\epsilon}\alpha)}\img\univ\case{1}{1}{1}
   y\subseteq \tildel{(\mathcal{X}\uphr(\iter{\mathrm{S}}{3}\fap m))}\fap((\mathrm{inh}_\epsilon\alpha)\fap y)
\]
とみなす.
$ \iter{\mathrm{S}}{2}\fap m $から始める理由は,$ \mathcal{X}\fap 0 = \Lambda $であり,$ \mathcal{X}\fap 1 $でも実際上有意味な規制類型を決定できないであろうから,$ \orp{m,y}\in\mathrm{inh}_\epsilon\alpha $であるような$y$であっても,$ y\subseteq \mathcal{X}\fap 2 $ではある,と考えられるからである.

さらに,$\epsilon$の不特定性を逆手にとって,$ \varSigma^{\epsilon}\alpha $の内部階層の最大値の特定化を実質的に不要にできる.
\begin{df}
\label{df:内部階層の最大値}
\kagi{$
    \mathfrak{w}\epsilon
$}は\kagi{$
    (\imath n)(2\in n \in\mathbb{N}\con{1}\trgl{\arg\epsilon}=\mathcal{W}\fap n)
$}を表わす
\end{df}
\noindent とすると,単に,ある$ m\in\mathbb{N} $について,
\[
    m = \union{((\mathrm{inh}_\epsilon\alpha)\img\univ)}\con{1}
    \iter{\mathrm{S}}{2}\fap m \in \mathfrak{w}\epsilon
\]
とみなすだけでよい.そして,$ x\in y\in\brevel{(\varSigma^{\epsilon}\alpha)}\img\univ $についての以下の記述を,$y$を言及する規定条件の一部として使用する.ただし,予め$(\mathrm{inh}_\epsilon\alpha)\fap y$の特定化が必要である.
\[
    x\in\tildel{(\mathcal{X} \uphr\mathfrak{w}\epsilon)}\fap((\mathrm{inh}_\epsilon\alpha)\fap y).
\]

関連して,$\alpha\bkg{\epsilon}\beta$や$\prob{\alpha}{\beta}$も(定義上$\trgl{\arg\epsilon}$等に制限されてはいるものの)そのサイズが大きくなりがちであり,これらを規制類型の内部構造に配置する場合は,そこに含まれる規制類型と同様に$ \mathcal{X}\fap i $等に制限する必要がある.あるいは,規制類型の中ではこれらをなるべく使用せずに,次のようなN類型$y$に還元する方がよいかもしれない.すなわち,
\[
   \arg\trgl{y} = 4\con{1}\arg y = \mathcal{W}\fap(\breve{\mathrm{S}}\fap i)
\]
ならば,$ \exe{\epsilon}y $によって,実質的に単なる因果的構造を表現することができる.または,$ \arg\trgl{y} = 3 $として,$\enf{\epsilon} y$で表現してもよい.

なお,既に述べたように,$\mathrm{anc}_\epsilon\alpha\subseteq\breve{\alpha}\img\univ$であるが,始祖集合の選択によっては逆は成り立たない.
例えば,始祖の実行を妨害する行動に対するS規制が体系内にあり,かつ,その規制類型の内部構造に始祖類型が含まれる場合,始祖類型であっても包括類型でないものが存在することになる.
しかし,そのような結果を生む当のS規制の構造解析が適切かどうかは問題になる.おそらく,妨害対象となる行動を規定するのに,規制自体の特定化は必要ないであろうから,それは存在量化して準拠領域の内部構造から除外すべきだと思われる.つまり,
\[
    (\exists x)(\exists y)(\exists i)(\orp{x,y}\in \mathrm{SY}^{\epsilon}\alpha\con{1}i\in\arg \trgl{y}\con{1}
    e = x\fap i\con{1}f = \trgl{y}\fap i)
\]
などとして,$ e\in f $であることを阻止する行動が構成要件になるようにする.
こうした点を踏まえると,始祖の集合$\alpha$を選ぶ場合,$ \mathrm{anc}_\epsilon\alpha = \breve{\alpha}\img\univ $となるように$\alpha$を構成した方が見通しは良くなるだろう.
$\alpha$がそのようなものである限り,$ y\in\breve{\alpha}\img\univ\case{1}{1}{1}(\mathrm{inh}_\epsilon\alpha)\fap y = 0 $.
そして,$ \alpha $のメンバーに対して制定/廃止関係を持つような規制は$ \alpha $のメンバーではなく,$ \varSigma^{\epsilon}\alpha $のメンバーでもない.
この方針によると,憲法規制と憲法改正権を混在させた始祖集合は適切ではなく,憲法改正権と憲法制定権を始祖とする正規集合を考えるべきであることになる.

最後に,$ x\in y\in\mathrm{Reg} $について,$ \trcl x $に他の規制類型が含まれ得るが,正則性公理(A \ref{axim:正則性})によって,$y$自身がそれに含まれることはない.
一般化すると,$ x\notin\trcl x $.なぜなら,
\begin{pfx}
$ x\in\trcl x $と仮定すると,ある$n$が存在して,$ x\in \iter{(\lambda_z\union{z})}{n}\fap x $.

①$n=\Lambda$である場合,$x\in x$だから,$\neg(\exists y)(y\in\classab{x}\con{1}y\cap\classab{x}=\Lambda)$.したがって,A \ref{axim:正則性}と矛盾する.

②$ \Lambda\in n $である場合.$\arg y = n+1\con{1}y\fap 0 = x $である系列$y$が存在して,任意の$ 0\in i\subseteq n $について,
\[
    (\exists j)(
        j = \iter{\breve{\mathrm{S}}}{i}\fap n\con{1}
        y\fap(\breve{\mathrm{S}}\fap i)\in y\fap i \in \iter{(\lambda_z\union{z})}{j}\fap x
    ).
\]
すなわち,循環的な配列 $ x\in y\fap 1\in y\fap 2 \in \dots y\fap n\in x $ が出来上がる.したがって,
\[
    \neg(\exists z)(z\in(y\img\univ)\con{1}z\cap(y\img\univ)=\Lambda)
\]
となりA \ref{axim:正則性}と矛盾する.
\end{pfx}

\noindent したがって,いかなる規制も自分自身を制定することはできず,それ自身によって廃止されることもない.さらに,それ自身から制定関係を反復して遡及可能などの規制についても,それを制定することも廃止することもできない.

\section{結論}
\label{sec:結論}

$ \orp{x,y} $が規制であるということは,$ \orp{x,y}\in\mathrm{REG}^\epsilon $であるということである.それは解釈空間に相対的な因果的構造であり,これまでに定義された通りの構造である.原初的言語への還元によって,規制の概念それ自体には曖昧さは存在しない.すべての文脈依存性は,解釈空間を指示するクラス抽象体の代わりとなる型文字\kagi{$ \epsilon $}に吸収される.
このような構造としての規制は結局は物理的システムであり,当然ではあるがそれの工学的妥当性とは区別される.あるいは,規制類型の規定条件を構成する構造解析の作業と,規制の工学的妥当性を検証することは,異なるプロジェクトに属する.後者は当該規制が属する規制体系または正規集合に相対的な評価構造によって測られる.
また,規制体系または正規集合には2種類の階層構造が設定される.制定階層はこれらの集合を定義する条件であり,その階層を辿ることで個別の規制の執行可能性の認定を代替し,言語的統制を実現する.もう一つの階層構造である内部階層は,集合の累積的階層と連動することによって,規制類型の存在論を確定する.そして,解釈空間$\epsilon$が設定する対象領域の階層は,規制体系の階層の深さを常に上回るものと想定される.
 % 規制体系

\bibliographystyle{jplain}
\bibliography{refs}

\end{document}