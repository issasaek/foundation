% !TeX root = foundation.tex

\begin{textblock*}{0.1\linewidth}(510pt, 60pt)
    \small \version
\end{textblock*}

\title{規制の概念とその論理的還元}
\author{佐伯 一冴\\\small\url{https://github.com/issasaek}}
\date{}
\maketitle

\begin{abstract}
工学的設計の対象としての規範が実はいかなる対象なのか明らかでない.規範の同一性の基準も物理的因果的世界における規範の位置も不明であるため,規範的システムの解析と設計は最適化が十分でない.その解決策として,規範をある種の因果的な構造とみなして,その構造を数学的意味での集合に還元する.
まず,因果の概念について,$a\in x$であることが$b\in y$であることを因果的に決定するような$\orp{a,x}$の$\orp{b,y}$に対する関係を,ある解釈空間に相対的に定義する.
次に,行動の環境的条件$y_1$,$y_1$における行動惹起$y_2$,行動の後続条件$y_3$を一義的に特定するような構造$y$について,
①$y$の執行可能性と②$y$の制御可能性という二つの因果的な構造を定義する.そして,$y_1$を充たす領域$x$について①と②が成立しているとき,順序対$\orp{x,y}$を1個の規制とみなす.工学的設計の対象としての規範はこの意味での規制である.
\end{abstract}

\section{序論}

規範的システムの設計は典型的には法システムにおいて現れるが,応用倫理学が倫理的ルールが表現する構造の設計に多分関わっているように,倫理的システムも設計対象になり得る.しかし,規範的システムを設計・構築する際,実はいかなる対象が設計・構築されているのかが明らかでないことが問題となる.つまり,規範の同一性の基準が不明であるから,1個の規範の内部構造を解析することができず,規範の集合が持つ全体的構造(例えば法規範の集合が持つであろう階層構造)の把握も覚束ない.また,物理的因果的世界における規範の位置が不明であり,規範がそれの成否による物理的相違を示唆するように記述されることもない.そのため,設計される規範的システムの因果的影響を予測して,その工学的妥当性を検証するという戦略が進展しない.

この問題の解決策として,規範をそれを表現するルール等の言語的対象と同一視することが考えられる.言語的トークンは通常の物理的対象であり,言語的タイプはトークンの集合かそのような集合の系列と見做せるから,言語的対象それ自体の同一性に問題はない.しかし,いかなる文または文の集合が1個の規範であるのか,という別の個別化の問題が直ちに生じる.仮にルールの記述的な性格を否定しつつ,その構文論的または意味論的な特徴によってこの個別化に対処しようとするなら,その困難は手に負えないものになるだろう\footnote{例えば,民法の約1000個の条文を眺めて,それらが何個の規範なのかを確定する作業を想像するとよい.}.むしろ必要なことは,規範をルール等の言語的対象から独立した構造として位置づけること,そして,ルールは端的にそのような構造に関する事実を記述すると認めることである.

本稿では,規範をある種の(主として物理的な)因果的構造として捉えた上で,その構造を数学的意味での集合に還元するという解決策を提案する.この因果的構造を表す理論的概念を拵えるが,その定式化は標準的な公理的集合論の言語に還元可能な形式で行われる.それによって,規範的システムの工学的妥当性の基準を構築することに方向づけを与え,規範的システムを解析的に記述する枠組を提供する.また,法的または倫理的規範から完全に個人的な規範に至るまで,それらが共通に持つ構造を特徴づけることによって,統合的な理解を促進する.

\section{描像}

これ以降,工学的な解析と設計の対象としての規範を特に「規制」と呼ぶ.第 \ref{sec:論理}節以降で定式化される概念は規制の概念である.それは規範の解析と設計を最適化する工学的理論を準備するために開発される人工的な言語的装置であり,既存の規範表現の概念分析ではない.したがって,日常的な「規制」等との用法の一致は,規制の概念に付与された機能を果たすのに必要な範囲でのみ要求される.また,規制の概念の定式化は,日常言語の規範演算子を識別してそれに関わる論理を定式化しようとするものではない.それは,工学的設計の対象としての物理的システムを記述する言語的装置であり,規範表現の論理形式とは関係がない.単にそのような記述を明晰化するために形式化された言語が使用されるにすぎない.
以上を確認した上で,ここでは定式化の前段階として,規制の概念の漠然としたイメージを提示する\footnote{規制の概念はそれなりに込み入っており,冗長にならずに日常言語で完全に展開することは困難である.}.

\subsection{一般形式}

規範的システムは,人やその他のシステムの行動を制御すること,すなわちそれを促進したり抑制したりする機能を持つ.犯罪に対する刑罰の仕組みや,道徳的な行動を賞賛する(または不道徳を非難する)等の事例では明白であるが,そこでは学習理論において時に弁別学習と呼ばれるようなプロセスが利用されている.
弁別学習とは,一定の条件$F$における行動$G$に報酬または罰$H$を(何度か)後続させると,同一の行動主体について$F$における$G$の出現確率が高まる/低下する,というプロセスである\footnote{レイノルズ~\cite[pp.\,9--12]{レイノルズ}.}.行動主体が言語的能力を持つ場合,$F+G$に$H$が後続するという経験を積まなくても,「$F+G$の実現によって$H$が実現される」という因果関係の情報を言語的な経路で取得すれば,$F$における$G$の出現確率が同様に変化し得る\footnote{長谷川~\cite[pp.\,4]{長谷川}.}.

規制の概念は,この弁別学習のプロセスに含まれる因果的構造からその一般形式を抽象して得られる.
まず,条件$F$,行動$G$,後続条件$H$を一義的に特定する何らかの構造(規制類型)$X$を考える.今の段階では単に3つ組$\orp{F,G,H}$でもよい.
次に,規範的システムの対象は特定の有機体の再現性を持つ行動だけではないため(同一の個体について$F$や$G$の再現性がないケースがあり得る),$F$における$G$の出現確率の変化の代わりに,次のいずれかの制御構造(の蓋然性)に言及する.
\begin{enumerate}[label=(\arabic*)]
    \item $F$の実現によって$G$が実現する,という因果的関係($X$の正の制御構造),
    \item $F$の実現によって$G$の実現が阻止される,という因果的関係($X$の負の制御構造).
\end{enumerate}
また,$F+G$に$H$が(何度か)後続するという事態は,$F+G$($X$の構成要件)が実現し,かつ,次の因果的関係($X$の執行可能性)が成立しているという事態に置き換える.
\begin{align*}
    \text{$F+G$の実現によって$H$(の蓋然性)が実現される,という因果的関係.}
\end{align*}
さらに,$X$の執行可能性を充たす領域で$X$の構成要件が共に実現すること($X$の執行随伴性)によって,同一の行動主体について$X$の正/負の制御構造(の蓋然性)が生じる,という因果的関係を$X$の正/負の制御可能性と言う\footnote{この因果的関係を担うメカニズムは行動主体の種類や個体ごとに異なり得る.メカニズムの相違は行動科学ではともかく規範的システムの解析と設計においては必ずしも重要ではない.行動主体が人間等の有機体であっても,法人のようなシステムであっても,規制の概念の一般形式は等しく適用できる.}.

$F$を充たす領域$s$について$X$の執行可能性と$X$の制御可能性が成立しているとき,順序対$\orp{s,X}$は1個の規制である.このとき,
$s$について成立する制御可能性が正の制御可能性であるなら,$s$における$H$の実現は$s$における$X$の構成要件の実現に対して報酬の機能を持つ.他方,負の制御可能性なら罰の機能を持つ.
そして,前者の$\orp{s,X}$をP規制,後者のそれをS規制と呼ぶ.P規制は伝統的なカテゴリーで言う許可や権利の領域を,S規制は禁止や義務の領域を主としてカバーする.

\subsection{言語的統制}

引き続き$X=\orp{F,G,H}$と置き,行動主体$a$を含む領域$s$について,$F$と$X$の制御可能性が成立していると仮定する.このとき,$s$について$X$の執行可能性を構築しておけば,$s$で$G$が実行された場合に,将来の$a$について$X$の制御構造(の蓋然性)を形成することができる\footnote{規制類型$X$に関する規制を設計し構築するということは,$X$の執行可能性を構築することを意味する.}.
しかし,$s$における$G$の実行がなくても,$s$における$X$の執行可能性を記述する($a$の言語の)文$\phi$について,$a$において$\phi$を肯定する言語的傾向性があれば,同一の制御構造を形成できる可能性がある.このような言語的統制が機能するための典型的なモデルを以下のように構成できる.

まず,他の規制の執行可能性を後続条件(報酬/罰)とする規制を考える.この場合,その構成要件実現によって他の規制の執行可能性(の蓋然性)が構築される.すなわち,$X'=\orp{F',G',H'}$について,$s'$における$H'$の実現が$s$における$X$の執行可能性の実現であるような規制$ \orp{s',X'} $を考える.$s'$において$X'$の構成要件が実現している場合,$ \orp{s',X'} $は$ \orp{s,X} $に対して制定関係を持つ(制定規制).
これに対して,$s'$における$H'$の実現が$s$における$X$の執行可能性の阻止である場合は,廃止関係を持つ(廃止規制).
$s'$における制定類型$X'$の構成要件の実現が,$X$の執行可能性を記述する規制ルール\footnote{ここでは,$F$を充たす任意の領域$s$について$X$の執行可能性が成立する,ということを記述する文を意味する.}を当事者に提示する行動である場合,言語的統制との関係で特に重要なカテゴリーに属する(規制表明型)\footnote{例えば,議会の立法権や契約により私法的権利義務を創出する権利は,制定規制であるP規制であり,かつ,規制表明型である.他方,不法行為により損害賠償義務を負う関係は,制定規制であるS規制の執行可能性であるが,規制表明型ではない.}.

次に,ある規制集合$\alpha$に相対的な正規集合の概念を構成する.すなわち,$\orp{s,X}$が制定関係を反復して$\alpha$のメンバーに遡及可能であり,かつ,その反復のどの段階についても,それに対して廃止関係を持つ正規集合のメンバーが存在しないとき,$\orp{s,X}$は$\alpha$の正規集合に属する\footnote{$\alpha$のメンバーとして何を選ぶかは文脈依存的であるが,差し当たって,憲法その他の基本法の一群かまたはそれらを制定する権限を想定すればよい.}.

さて,$r_1\in\alpha$であり,かつ,系列の前者が後者に対して制定関係を持つような規制の系列$r_1,r_2,\dots,r_n$を考える.ただし,$r_n =\orp{s,X}$とする.
そして,$s$の行動主体$a$について,次の①②を仮定する.
\begin{enumerate}
    \item[①] $\alpha$の正規集合の主要部分が規制表明型である等により,$a$は,$r_1,r_2,\dots,r_n$及び関連する正規集合メンバーの構成要件該当性を記述する文を肯定する傾向性を持つ.
    \item[②] $a$は,$r_1\in\alpha$の執行可能性を記述する文を肯定する傾向性を持つ.
\end{enumerate}
このとき,以下のようにして,$s$における$X$の執行可能性を記述する文$\phi$を肯定する$a$の言語的傾向性を形成できる.
すなわち,②により,$r_1$の執行可能性は前提される.したがって,$r_1$の構成要件該当性を証拠として,$r_2$の執行可能性を認定できる.次に,$r_2$の構成要件該当性を証拠として,$r_3$の執行可能性を認定できる.以下同様にして,$r_n$の執行可能性を認定できる.ただし,各$r_i(1\leq i\leq n)$の執行可能性が,$\alpha$の正規集合に属する規制によって廃止されていないことも前提しなければならない.
